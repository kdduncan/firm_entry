\documentclass{article}

\usepackage{amsmath}

\begin{document}

\section{Spatial Ergoticity}

\textbf{Can spatial DGP's be ergotic in location given a valid measure space?}

Let us denote a spatial plane by the real line. Then there is a mapping from the spatial plane by a data generating process, let us denote it;

$F: R \to R$

Then "Spatial Ergoticity" implies that an individual observed data point $y_{i}, i \in R$ is represented by some closed interval on $R$, denoted $C_{i}$. Thus, we can find some $j \in R$ such that $j \not \in C_{i}$

Thus, we can find (given that we have a single dimensional space), a $M$ and $N$ such that

$E(y_{i}|y_{j},y_{k}) = 0 \forall j > M, k < K$

\section{Shared Labor Markets for Two Counties}


\end{document}
\documentclass{article}

\usepackage{hyperref}
\usepackage{amsmath}

\begin{document}

\section{Variables}

\subsection{Dependent Variables}

A variety of dependent variables are used in my paper. The primary object of interest is first start up rates, as a result, three main variables are used.

The first, measures the relative difference in firm start up rates, where $sub$ denotes the subject county, and $nbr$ the county it borders.

$$log(births_{sub}) - log(births_{nbr})$$

I also use a birth \textit{rate} by normalizing each county's firm start ups by the existing number of firms at the start of each time period, denoted $base$. Then,

$$\frac{births_{sub}}{base_{sub}} - \frac{births_{nbr}}{births_{sub}}$$

Finally, I include an unadjusted difference between the firm start up rates,

$$base_{sub} - base_{nbr}$$

My data set includes a robust set of possible variables to utilize, including firm closing figures, and whether or not a firm expanded employment, contracted employment, or stayed the same. Equivalent dependent variables to using the firm start up figures can easily be constructed.

Further, the census data provides a robust set of results for a broad set of general industries.

\subsection{Policy Variables}

Following Orazem, McPhail, and Singh (2010), my primary tax variables are pulled from the National Bureau of Economic Research estimates of state marginal income tax and long-term capital gains tax rates. When applicable, I pull from the highest marginal tax rates available, as this is the rate most applied to small business and S corporations.\footnote{\url{http://users.nber.org/~taxsim/allyup/} \url{http://users.nber.org/~taxsim/marginal-tax-rates/} \url{http://users.nber.org/~taxsim/state-marginal/}}

Corporate and sales tax rates were compiled from The Council of State Governments \textif{Book of States}, where marginal rates again are the highest state tax rates on business corporations. Where rates differ between banks and non-banks, I use the non-bank rate. The sales tax rates used are those levied on general merchandise, and non on food, clothing, and medicine.

Property taxes are calculated from household level data provided by the Minnesota Population Center's INtegrated Public Use Microdata Series (IPUMS)
\subsection{Government Expenditures}

\subsection{Controls}

\end{document}
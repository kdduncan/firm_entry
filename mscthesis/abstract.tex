\documentclass[12 pt,a4paper]{article} %
\usepackage{graphicx,amssymb} %

\textwidth=15cm \hoffset=-1.2cm %
\textheight=25cm \voffset=-2cm %

\pagestyle{empty} %

\date{} %

\def\keywords#1{\begin{center}{\bf Keywords}\\{#1}\end{center}} %

\def\titulo#1{\title{#1}} %
\def\autores#1{\author{#1}} %

% Please, do not change any of the above lines

\begin{document}
\title{Do Taxes Impact Firm Location on the Margin? A Matched County-Pair Study}
\author{Kevin D. Duncan \\ Iowa State University \\ 978-390-3459}
\maketitle


The impacts of taxes on business activity and growth is a heated topic in growth and public economics. Topics include Gordon and Lee (date), 

States and counties try to find ways to incentivize new business start-ups in their locale over alternative choices. The most visible cases include temporary reprieve from tax burdens, or deals to help build infrastructure to support new start-ups. There has been a growing literature addressing the efficiency of these exemptions both to state revenue and public welfare. (examples). 

However, states may also take a longer run approach to incentive new firm start-ups by altering their tax and regulatory codes for all market participants. There is comparably less literature examining both the causes and effects of these alterations. This paper examines the impact of state policies on entry decisions of firms into markets that are split between policy regimes. I attempt to discern what policies give locations higher numbers of firm start-ups over neighbors when inputs have high levels of mobility such that small differences in taxes and government expenditures can incentivize firm decisions. The paper builds off of the existing Tiebout-style public finance literature which claims that people sort into counties that have the optimal bundle of prices and public goods by extending it into firms' decision process over taxes and private goods. Further, it extends existing public economics literature on impacts of policies on individuals' and firms location choice with respect to policy discontinuities. The results of this study can lead to better understanding of how structuring taxes and expenditure programs can attract new firms, and to discern what borders have the highest potential for such policy alterations.

The identification strategy used in this paper takes the difference in variables between two matched counties on either side of a state border, where the data includes 105 state-pairs and 1213 matched county pairs from 1998 to 2010. This method generates a local regression discontinuity that is estimated with pooled OLS. The vector of policies includes an array of top marginal tax rates for property, sales, capital gains, corporate, income, workers compensation, and unemployment insurance taxes, as well as minimum wages and right to work status. I also include state expenditures on highways, welfare, and education to test the impact on policy variation on firm entry decisions.

Preliminary results show that coefficients are statistically identical across counties such that altering the unit of observation does not change estimated values. Secondly, the pooled OLS estimates indicated that property taxes, income taxes, and sales taxes significantly impact firm start-up rates with predicted negative elasticities. Secondly, the pooled OLS estimates indicated that the percentage decrease in new firms entry from a 1 percentage point increase in the marginal tax rate was -0.4\% for marginal property tax rates, -0.08\% for marginal income tax rates, and \\ -0.1\% for marginal sales tax rates. Corporate and capital gains tax rates are indistinguishable from zero, which might be due to the fact that many new start-ups are S-corporations and therefore pay income tax; that capital gains taxes are only important if firms expect to be atypically successful. The disincentive effect generated by property, income, and sales taxes plausibly comes from their more immediate impact on firm profits and start-up costs. 

I conclude by providing an index of where the tax differentials are the largest, and an estimate of the aggregate effect of taxes on deterring firm entry. This is calculated by multiplying my estimated coefficients by the existing marginal taxes in those counties. This should allow readers to better visualize the full impact of taxes on firm entry is, as well as what borders currently have the largest amount of firm start up differential imposed by their regulatory choices.
\\
\\
JEL Classifications: H71, H73

\end{document}

I provide several reduced form estimates for the differences in firm start up rates. First, I prove that the coefficients are identical for both counties, then I provide estimates of the coefficients imposing that the coefficients are the same. Next I provide random coefficient estimates allowing state-pair coefficients to be different, and again test if empirical estimates are largely stable across policy regimes.
\section{Variables and Data}

\subsection{Matching Process}

Our matching procedure is as follows. We first obtained Census county adjacency files.\footnote{\url{https://www.census.gov/geo/reference/county-adjacency.html}}, then used it to construct county-pairs by generating all pairs of counties that have adjacent counties in a neighboring state. From this matching we also tracked state FIPS codes to create a list of state pairs. For each state-pair we assigned one side of a border to be either a subject ($sub$) or neighbor ($nbr$) side of the border, which we use in our data construction. This matching generates 1213 matched county-pairs with 107 state-pairs in each year. Throughout we will index each state-pair by $g = 1,...,107$, and the set of matched county pairs for each state pair by $i_{g} = 1,...,N_{g}$, where $N_{g}$ is the number of pairs for each border.

We then generated an extended border match. For this process we matched each subject county to each of its neighbor's neighbor, then excluded from any county in the original matching set. We provide a graphical representation of these matching processes in Figure \ref{rb}. This extended match connects 1549 county-pairs across 107 state pairs each year.

\subsection{Firm Entry Data}

Our primary variable of interest were county level firm start up rates for all firms in a year. This data set was procured at the Census Bureau's Business Dynamic Statistics program.\footnote{\url{http://www.census.gov/ces/dataproducts/bds/overview.html}} The data included the number of firm births, deaths, expansions, and contractions for each year from 1999 to 2013. It also provided these figures for  broad NAICS coded industries. As a result, our main variable of interest, $births\_ratio$ is calculated for each matched counties along state pair $(A,B)$ in time $t$ as,

\begin{equation} births\_ratio_{i,g,t} = \ln(n_{sub,A,t})-\ln(n_{nbr,B,t})\end{equation}

\subsection{Tax Data}

We included top marginal tax rates of seven taxes from 1977 to 2008. In all cases we used a one period lagged difference in top marginal values. For each tax rate $\tau$ and state pair $g = (A,B)$, at time $t$ the tax ratio was calculated 

\begin{equation} \tau\_ratio_{g,t} = \tau_{A,t}-\tau_{B,t} \end{equation}

State marginal income tax and long term capital gains tax rates were obtained from The National Bureau of Economic Research. For income tax rates we used the highest marginal tax rates available, as this is the rate most applied to small business and S corporations. When not available, we calculated the highest implied tax rate. \footnote{\url{http://users.nber.org/~taxsim/allyup/} \url{http://users.nber.org/~taxsim/marginal-tax-rates/} \url{http://users.nber.org/~taxsim/state-marginal/}}

Corporate and sales tax rates were compiled from The Council of State Governments Book of States\footnote{\url{http://knowledgecenter.csg.org/kc/category/content-type/content-type/book-states}}. We used the highest marginal state tax rates on business corporations. Where rates differ between banks and non-banks, we use the non-bank rate, and we restrict to sales tax rates levied on general merchandise, rather than food, clothing, or medicine. 

Property taxes were calculated from household level data provided by the Minnesota Population Center's Integrated Public Use Micro-data Series (IPUMS).\footnote{\url{https://usa.ipums.org/usa/}} Workers compensation was calculated from Thomason et al (2001) between 1977 and 1995, with data afterwards provided by the Oregon Department of Consumer and Business Services. 

Finally, the top marginal unemployment insurance tax rates were provided by the US Department of Labor. To calculate, they multiplied the top marginal tax rate, $\tau_{u,it}^{max}$, by the maximum wage level to which the rate is applied, $W_{it}^{max}$. They normalized this figure by the average wage in a state in a current year, $\bar W_{it}^{max}$. Then the unemployment insurance tax was calculated for each state as;
\begin{equation} \tau_{A,t} = \frac{\tau_{A,t}^{max}W_{A,t}^{max}}{\bar W_{A,t}^{max}}\end{equation}

\subsection{Government Expenditures}

We compiled log state governments expenditures on highways, education, and welfare per capita using Census data on State Government Finances.\footnote{\url{https://www.census.gov/govs/state/}} We used expenditures on "Education" for our education value, welfare sums up expenditures on "Public Welfare", "Hospitals," and "Health," while highways is calculated from "Highways" expenditures pulled from annual historical data accounts. To calculate per capita terms we divided each figure by Census state population estimates\footnote{\url{http://www.census.gov/popest/}} and then took logs. For each of our expenditure figures ($exp\_percap$), the state differenced variable for two states and time t was calculated as,

\begin{equation} exp\_percap\_{g,t} = \log(exp_{A,t}/pop_{A,t}) - \log(exp_{B,t}/pop_{B,t}) \end{equation}

\subsection{Additional Controls}

As a final series of controls, we included state level variables for percent of workforce unionized, log real fuel prices, population density, percent of industry manufacturing, and percent of population with high school education. This data was collected from, "Union Membership and Coverage Database from the CPS."\footnote{\url{http://www.unionstats.com/}}

Lastly, amenity data was acquired from the USDA.\footnote{\url{http://www.ers.usda.gov/data-products/natural-amenities-scale.aspx}} We used normalized values of hours of sunlight in January, temperature in July, humidity in July, topology score, and percent of county that is water. After normalization each amenity variable is normal with approximate mean zero and standard deviation 1. As a result, interpretation of these terms should be done in terms of deviations from the mean. Again, we take difference in county level Z-scores, and it is the only county level data we include in our empirical estimates.

\subsection{Preliminary Analysis}

Summary statistics are provided in Table (\ref{--summary}).

We test the hypothesis that states use taxes jointly to accomplish policy goals. We plot simple cross correlations between our differenced tax variables in Table \ref{pairs} as a heuristic test. Between 1998 and 2008, income tax and capital gains tax rates exhibit highly positively correlation, the simple correlation between values is 0.64. We further see that sales, payroll, workers compensation, and unemployment insurance tax rates have low rates of correlation with other tax rates. 

The presence of simple correlations indicate policy makers might have shifted taxes jointly to accomplish policy goals and tried to advantageously shift tax incidence. Thus, modeling firm entry using a larger set of top marginal tax rates will improve estimates of tax incidence on firm start up rates.
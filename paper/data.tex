\section{Variables and Data}

Our primary variable of interest is county level firm start up rates for all firms in a year. This data set is generously provided by the Census Bureau's Business Dynamic Statistics program.\footnote{\url{http://www.census.gov/ces/dataproducts/bds/overview.html}} The data also includes how many firms expanded or contracted employment, a variety of broad NAICS coded industries, and the number of firms that shut down in a given year, however we omit these variables in practice. Since many counties in our sample do not have all the sub coded, for completeness our results rely on the summation across different industries. 

For our matched county pair estimation, our data set includes 105 state-pairs, with 1213 matched counties over 10 years. Additional borders could be added, namely New Mexico - Oklahoma, and Arizona - Oklahoma. However, due to the number of regressors we exclude the pairs for degrees of freedom.

As part of our empirical design we drop counties not on the border of a state to ensure that as we linearly approach the border from either side we have a single limit. We match counties that share borders by arbitrarily assigning subject ($sub$) and neighbor ($nbr$) classification. As a result, our main variable of interest, $births\_ratio$ takes on the form, 

\begin{equation} births\_ratio_{t} = \ln(n_{sub,t})-\ln(n_{nbr,t})\end{equation}
We also include estimations for the raw number of implied firm difference, 
\begin{equation} births\_diff = n_{sub,t} - n_{nbr,t} \end{equation}
We also include a variable $allstarts$, that is a single column vector of all observations for the $sub$ and $nbr$ counties stacked on top of each other. This later variable is what we use to estimate our count data models. In the future we will extend our count data model data set to include all counties in the US between 1998 and 2009.

Equivalently to Orazem, McPhail, and Singh (2010) our tax variables are provided by the following sources. The National Bureau of Economic Research estimates of state marginal income tax and long-term capital gains tax rates. When applicable, we pull from the highest marginal tax rates available, as this is the rate most applied to small business and S corporations, or calculate the highest implied tax rate.\footnote{\url{http://users.nber.org/~taxsim/allyup/} \url{http://users.nber.org/~taxsim/marginal-tax-rates/} \url{http://users.nber.org/~taxsim/state-marginal/}}

Corporate and sales tax rates were compiled from The Council of State Governments Book of States, where marginal rates are the highest state tax rates on business corporations. Where rates differ between banks and non-banks, we use the non-bank rate, and we restrict to sales tax rates levied on general merchandise, and rather than food, clothing, and medicine. Property taxes are calculated from household level data provided by the Minnesota Population Center's Integrated Public Use Micro-data Series (IPUMS). The top marginal unemployment insurance tax rates were provided to Orazem, McPhail, and Singh by the US Department of Labor. To calculate, they multiply the top marginal tax rate, $\tau_{u,it}^{max}$, by the maximum wage level to which the rate is applied, $W_{it}^{max}$. They then normalize by the average wage in a state in a state in a current year, $\bar W_{it}^{max}$. Then the unemployment insurance tax is calculated as;
\begin{equation} \tau_{u,it} = \frac{\tau_{u,it}^{max}W_{it}^{max}}{\bar W_{it}^{max}}\end{equation}
Workers compensation is provided between Thomason et al (2001) for between 1977 and 1995, with data afterwards provided by the Oregon Department of Consumer and Business Services. In all cases we use the lagged difference in their top marginal values, such that for a tax rate $\tau_{i}$ we get that for each pair of states $sub, nbr$ and time $t$ the tax ratio is calculated 
$$tax\_ratio_{i,t} = \tau_{i,sub,t}-\tau_{i,nbr,t}$$
All of our tax variables are scaled to be between 0 and 100, where a 100\% tax rate would be 100. This creates an intuitive interpretation of our estimated coefficients later.

We include two major non-tax variables that we expect to have a major impact on firm entry, right to work status and minimum wage rates. The former is the difference in a states right to work status coded as a binary variable with $1$ if the state has right to work laws, and $0$ otherwise. Right to work is a law that enables workers to exempt themselves from joining unions. For minimum wages we simply take the difference between the minimum wage for each state in a given year, 
$$min\_wage\_ratio_{t} = min\_wage_{sub,t} - min\_wage_{nbr,t}$$
The Tax Policy Center also provides historical state minimum wage, education spending per capita, highway expenditure per capita, and welfare spending per capita data from 1983-2014.\footnote{\url{http://www.taxpolicycenter.org/taxfacts/displayafact.cfm?Docid=603}} Right to work status was compiled from the National Conference of State Legislators.\footnote{\url{http://www.ncsl.org/research/labor-and-employment/right-to-work-laws-and-bills.aspx}} s

Lastly, amenity data was acquired from the USDA.\footnote{\url{http://www.ers.usda.gov/data-products/natural-amenities-scale.aspx}} Since we care less about the coefficients of these variables, we use the normalized values of hours of sunlight in January, temperature in July, humidity in July, topology coefficient, and percent of county that is water. In all cases, the amenity variables are coded to be normal with mean zero and standard deviation 1. As a result, interpretation of these terms should be done in terms of deviations from the mean. 

As a final series of controls, we also include percent of workforce unionized, log real fuel prices, population density, percent of industry manufacturing, and percent of population with high school education. This provides a robust set of location modifiers to explain a lot of non-economic preference for geographic amenities. Table \ref{table:summary} provides a series of summary statistics for each of our differenced terms.

It should be noted that for our data set we see some state-pairs more often than others. As a result, the range of observations range between 20 for Delaware and New Jersey, Delaware and Pennsylvania, and Arizona and Nevada, and 350 between Oklahoma and Texas, out of 12130 total observations available before any additional transformations are taken. We graph all the borders in Figure 1, where red represents the subject county, and blue is its matched neighbor.

Firm start ups look like exponential decay with an incredibly fat tail. From zero there is a large spike up, followed by decreasing frequency that exhibits extreme values in its tail. We can see this initial spiking behavior much more clearly as we truncate the data set to only show data points where the number of new firm start ups was less than 1000.

Taking the difference between our subject and neighbor counties we can see that the difference exhibits the exponential decay split across sides, and the log form looks like a slightly skewed normal distribution. This is expected from how subject and neighbor status of counties is arbitrarily assigned.

We plot the aggregate un-weighted tax differential by summing up the difference between tax rates for each matched county pair. The difference almost looks like two normal distributions with opposite skews. Further, similar to the firm start up data, we see that the right hump appears to have higher variance toward it's tail .

Finally we produce correlative graphs between all the variables. Along the top row we see the correlation between the difference in log number of firm start up rates with each of our tax variables. At first glance there does not appear to be any correlation between firm start up rates and any of the individual tax rates. As a result, most of the correlations appear to be "black boxes." However in my sample income tax and corporate tax rates are highly positively correlated (.64). Using the data set of all taxes from 1977-2008 the correlation is (.55), thus it appears as if over time states have generally streamlined their tax and corporate tax rates to be related to each other on purpose. Further, corporate tax rates are also correlated with income tax and capital gains tax rates. 

Interestingly, we see the presence of  clusters in the unemployment insurance tax rate with every other variables as immediate outliers. This is present in the right most column by by a few data points appearing in the right most part separate from the remaining states' unemployment insurance tax rates. This suggests there are some states that have an abnormally high unemployment insurance tax rate. Through analysis, it appears as is Delaware has a significantly higher unemployment insurance tax rate than any other state in our sample.

Broadly we infer the following; Taking the log number of firm start ups, especially by plotting their difference, makes the data look incredibly normal. Further, simple correlations between our dependent variable and any single of our independent variables does not immediately seem to indicate any expected results, though the independent variables themselves clearly have some minor dependence on each other. Thus, utilizing more powerful conditioned regression models will hopefully tell a better story than what is often done in many empirical tax works by only utilizing one or a small sample of tax variables.
\section{Variables and Data}

\subsection{Matching Process}

Our theory section showed that as the location choice of firm entrants approaches a state border the  difference in location-specific attributes on either side of the border approaches zero.  Thus, an advantage of the border design is that these location-specific factors are differenced away in a specification that considers the difference in expected profits on either side of the border.  We estimate a “closeness to the border” bandwidth at the county level. The average county in our data set is 1,260 square miles, or about 35 miles per side if it is approximately square. This distance is slightly longer than more refined approaches such as Rohlin, Rosenthal, and Ross (2014). 

Our matching procedure is as follows. We first obtained the Census' County Adjacency File\footnote{\url{https://www.census.gov/geo/reference/county-adjacency.html}} to construct county-pairs by generating all pairs of counties that have adjacent counties in a neighboring state. This process is outlined in Table \ref{gensubnbr}. We use the file to track match each county with every adjacent county in a different state. The assignment of subject and neighbor status is derived from their ordering in the County Adjacency File. From this matching we track state FIPS codes to create a list of matched state pairs. This matching generates 1,213 matched county-pairs with 107 state-pairs in each year. Throughout we will index each state-pair by $g = 1,...,107$, and the set of matched county pairs for each state-pair by $i = 1,...,N_{g}$, where $N_g$ is the number of pairs for each border.


\subsection{Firm Entry Data}

Our primary variable of interest is county-level firm start up rates for all firms in a year. These data were procured from the Census Bureau’s Business Dynamic Statistics program.\footnote{\url{http://www.census.gov/ces/dataproducts/bds/overview.html}} The data include the number of firm births, deaths, expansions, and contractions for each year from 1999 to 2009. Data are reported in total number of firm births, and for broad NAICS coded industries. Our main variable of interest, $births\_ratio$, is calculated for each matched county-pair for each state pairs $(A, B)$ in time $t$ as, 

\begin{equation} births\_ratio_{i,g,t} = \ln(n_{sub,A,t})-\ln(n_{nbr,B,t})\end{equation}

where $n_{sub,A,t}$ is the number of new firm entrants in the state A's current subject county at time t and  $n_{nbr,B,t}$ is the corresponding number of firm births in the state B's neighboring county.

\subsection{Tax Data}

We include the top state marginal tax rates of seven taxes from 1998 to 2008 in our analysis.\footnote{We omit local tax rates because there is no existing database with county level tax rates. This leads to mild omitted variable bias that exists in the previous literature as well. This downwards biases our estimates as shown by Argawal (2015).} We use a one period lagged difference in the top marginal values due to time costs to opening, procuring permits, zoning, and building infrastructure. For each tax rate $\tau$ and state pair $g = (A,B)$, at time $t$ the tax ratio was calculated 

\begin{equation} \tau\_ratio_{g,t} = \tau_{A,t}-\tau_{B,t} \end{equation}

State marginal income tax and long term capital gains tax rates were obtained from The National Bureau of Economic Research. For income tax rates we use the highest marginal tax rates available, as this is the rate most applied to small business and S corporations. When not available, we calculate the highest implied tax rate.\footnote{\url{http://users.nber.org/~taxsim/state-marginal/}}

Corporate and sales tax rates were compiled from \textit{The Council of State Governments Book of States}.\footnote{\url{http://knowledgecenter.csg.org/kc/category/content-type/content-type/book-states}} We use the highest marginal state tax rates on business corporations. Where rates differ between banks and non-banks, we use the non-bank rate, and we restrict sales tax rates to those levied on general merchandise, rather than food, clothing, or medicine.

Property taxes are calculated from household level data provided by the Minnesota Population Center’s Integrated Public Use Micro-data Series (IPUMS).\footnote{\url{https://usa.ipums.org/usa/}} Workers compensation are calculated from Thomason et al (2001) between 1977 and 1995, with data afterwards provided by the Oregon Department of Consumer and Business Services. 

Finally, the top marginal unemployment insurance tax rates are provided by the US Department of Labor. To calculate these rates, they multiply the top marginal tax rate, $\tau_{g,t}^{max}$, by the maximum wage level to which the rate is applied, $W_{it}^{max}$. They normalize this figure by the average wage in a state in a current year, $\bar W_{it}$. Then the unemployment insurance tax is calculated for each state as: 

\begin{equation} \tau_{A,t} = \frac{\tau_{A,t}^{max}W_{A,t}^{max}}{\bar W_{A,t}}\end{equation}

\subsection{Government Expenditures}

We compiled log state governments’ expenditures on highways, education, and welfare per capita using annual historical Census data on State Government Finances.\footnote{\url{https://www.census.gov/govs/state/}} We use expenditures on Education” for our education value, the sums of expenditures on ”Public Welfare”, ”Hospitals,” and ”Health,” for the welfare measure, and ”Highways” expenditures for highway spending.  We divide each figure by Census state population estimates and then take logs.\footnote{\url{http://www.census.gov/popest/}} The difference between state A and state B, for each of our expenditure figures is calculated:

\begin{equation} exp\_percap\_{g,t} = \log(exp_{A,t}/pop_{A,t}) - \log(exp_{B,t}/pop_{B,t}) \end{equation}

\subsection{Additional Controls}

We include state level variables for percent of workforce unionized, log real fuel prices, population density, percent of employment in manufacturing, and percent of population with high school education. This data is compiled a mix of the Bureau of Economic Analysis, the Current Population Survey, the EIA, and the Census.

Finally, county level geographic amenity data were acquired from the USDA.\footnote{\href{http://www.ers.usda.gov/data-products/natural-amenities-scale.aspx}{USDA Natural Amenities Rankings}} These measures are the only county level data we include in our empirical estimates. We use the normalized values of hours of sunlight in January, temperature in July, humidity in July, topology score, and percent of county that is water. After normalization each amenity variable is normally distributed with approximate mean zero and standard deviation 1. These terms should be interpreted  as deviations from the mean. Again, we difference these county level Z-scores.

\subsection{Preliminary Analysis}

Summary statistics are provided in Table (\ref{--summary}). Of note is the fact that for all the taxes, the standard deviations are quite large relative to their means. Thus, there should be plenty of variation to provide identification of the impacts of taxes on firm entry rates.

We further plot simple cross correlations between our differenced tax variables in Figure \ref{pairs} as a heuristic test that states use taxes jointly to accomplish policy goals. Between 1998 and 2008, income tax and capital gains tax rates exhibit strong positive correlation; the simple correlation between values is 0.64. Sales, payroll, workers compensation, and unemployment insurance tax rates are only weakly correlated with other tax rates. The presence of simple correlations indicate studies that do not include a larger array of taxes, may suffer from omitted variable bias. Thus modeling firm entry using a larger set of top marginal tax rates will improve estimates of tax incidence on firm start up rates.

We also plot cross correlations between the differenced tax variables for each state in table \ref{pairsL1}. Due to the differenced nature of the data we are looking for co-movement between tax variables, which we see in a non-zero number of cases between all of our different tax variables. Of note is that the workers compensation tax seems to have more variation in the difference then some of our more traditional tax rates.
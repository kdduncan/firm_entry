\documentclass[12 pt,a4paper]{article} %
\usepackage{graphicx,amssymb} %

\textwidth=15cm \hoffset=-1.2cm %
\textheight=25cm \voffset=-2cm %

\pagestyle{empty} %

\date{} %

\def\keywords#1{\begin{center}{\bf Keywords}\\{#1}\end{center}} %

\def\titulo#1{\title{#1}} %
\def\autores#1{\author{#1}} %
% Please, do not change any of the above lines

\begin{document}

\title{Impacts of Taxes on Firm Entry Rates Along State Borders}
\author{Kevin D. Duncan \\ Iowa State University \\ 978-390-3459}
\maketitle

Economists have long been interested in the role that barriers to entry play into firm start up rates. The effects of tax rates are particularly appealing as they offer policy makers a means to bring about change in their communities. Recent work in estimating the effects of taxes on firm entry either do not use top marginal tax rates, such as Gabe and Bell (2004) and Ojede and Yamarik (2012), or do not include a full array of top marginal tax rates, such as Rathelot and Sillard (2008) and Brulhart et al (2012). Theory indicates that marginal rates are what matter to entrepreneurs looking to start firms, and that policy makers may use several different taxes in conjunction to accomplish policy goals. As a result, these papers may suffer from either poor identification or omitted variable bias.

We fix this gap in the literature by creating a data set that covers a larger set of top marginal tax rates than existing studies, and then use a matched county-pair techniques to control for location specific determinants of firm entry and endogenous components in our policy variables. This is done by taking the difference in firm start up rates and taxes between adjacent counties on either side of a state border. Variations on this technique have become increasingly common in studying the role government policies play on firm entry, including many of the cited studies. 

Our work is very similar to Rohlin, Rosenthal, \& Ross (2014), who utilize top marginal tax rates for corporate, income, and sales tax rates, to estimate a linear probability model of firm entry along state borders using GIS coded data. Comparably, we use county level data on firm births from the US Census Bureau to look at the impact of seven top marginal tax rates, including property, sales, income, corporate, capital gains, workers compensation, and unemployment insurance tax to study their impact on relative firm start up rates. Our data matches 1202 counties across 107 state-pair borders between 1999 and 2009. The result of our work shows that property, sales, and income taxes have a negative and statistically significant impact on firm start up rates. 

Our estimates indicate that a 1\% increase in income and sales tax differentials decreases relative firm start up rates by 0.1\%, while a 1\% increase in the relative property tax differential decreases firm start up rates by 0.3\%. These findings coincide with characteristics of new firms, where the majority of entrants are both small in size, with a short expected life time. As a result, the majority of firms are S corporations where the owners still by income taxes, and property and sales taxes may account for the largest share of tax incidence on their profits.

A final output of our study is an index of where the tax differentials are the largest. This is calculated by multiplying our estimated coefficients by the existing marginal tax rates in each state. This allows readers to better visualize the aggregate impact of taxes on firm entry, as well as what borders currently have the largest amount of firm start up differential imposed by their regulatory choices. We further generate a comparison between the mean difference in firm start up rates and the tax differential, to compare how often the side with the preferred tax differential also has higher firm start up rates.

In conclusion, we create a larger data set of top marginal tax rates for researchers to utilize in studying the effects of taxes. We further use a regression discontinuity technique along state borders to estimate the impacts of top marginal tax rates on firm entry rates. We find that sales and income taxes impact firm entry rates, results that are consistent with the characteristics of new firm entrants.
\\
\\
JEL Classifications: H73, L26, R30
\end{document}
\documentclass[12 pt, a4paper]{article}

\begin{document}

\title{Impacts of Taxes on Firm Entry Rates Along State Borders}
\author{Kevin D. Duncan}
\maketitle

Recent research on the imacts of firm entry have focused on the role that complementary human capital plays in firm entry such as Zucker and Darby (2014) and Brulhart et al (2012), and in turn, the impacts that firm entry can have on relative total factor productivity gains as shown in Greenstone, Hornbeck, and Moretti (2010). These studies show that local agglomeration economies, in the form of complementary labor and firms, are a recurring driver of new firm entry which in turn give local returns. However, these papers lack solid identification of traditional policy drivers of firm entry. Policies such as taxes might heavily impact entrepreneurial decisions, and might provide governments a way to encourage new firm entry.

The lack of papers focusing on identifying the impacts of taxes is attributable to the lack of variation in tax and regulatory policies. Many of these variables feature prolonged periods of stability across political regimes, and changes might be endogenous to policy action in response to economic activity (Romer and Romer (2010), Mertens and Ravn (2013)). Our paper fills this gap by using a regression discontinuity technique in order to estimate the policy effect of seven top marginal tax rates on firm entry behavior. 

We take the difference in firm start up rates on either side of a state border to control for location specific determinants of firm entry, such as labor market or agglomeration characteristics, and estimate a local policy effect. We propose this as a valid estimator, since as we approach the border in the limit the policy discontinuity in taxes is strictly exogenous, lending this area prone for exploration. Similar techniques have been used in Holmes (1998), Rohlin (2011), and Dube et al (2010) to study right to work status and minimum wages. Equivalently, this method allows us to estimate the local average treatment effect that tax policies impose on entrepreneurs.

We use data on firm births from the US Census Bureau, along with a compiled data set of seven top marginal tax rates that includes property, sales, income, corporate, capital gains, workers compensation, and unemployment insurance tax rates. We further include log expenditures per capita on highways, education, and welfare as an additional determinant of the sorting behavior of entreprenuers into preferred neighborhoods following McKinnish (2007). We then estimate a model that takes the difference in log firm start ups and independent variables between randomly assigned subject and neighbor counties, as well as an average firm start up rate along borders as proposed by Donald and Lang (2007).

We find that property, sales, and income taxes have a negative and statistically significant impact on firm start up rates, and that as the bandwidth increases the effect becomes indistinguishable from zero. These results remain significant even when run on NAICS subcodes and on single time period cross section analysis, and a border average estimator proposed by Donald and Lang (2007). Across most models, we find that a 1\% increase in the relative income and sales tax differentials decreases relative firm start up rates by 0.1\%. Property taxes seem to be a significant driver of firm entry, where a 1\% increase in the relative property tax differential decreases firm start up rates by 0.3\%. Finally, wee find evidence that relative welfare spending per capita is related to firm entry rate, while highway and education spending per capita is not.

Our findings that property, income, and sales tax rates have the largest impact on firm entry rate appear consistent with the fact that mthe mority of firms that enter in a given year are small, short lived firms. Comparably we would expect capital gains and corporate tax rates to have a higher impact on large firms that expect to have a long life. As a final output of our paper we calculate what borders have the highest discrepency in weighted tax differential, which multiplies our coefficients by the top marginal tax rate of each state. We then show which states have space to adjust their tax code in order to attract new firm entrants. We find that despite the low fit of our models, most of the states with the largest difference in firm start up rates feature a preferential weighted tax differential. In conclusion, by using regression discontinuity along state borders we are able to estimate the impacts of top marginal tax rates on firm entry rates, and find that sales and income taxes impact firm entry rates, results that are congruent with the structure of firm entry.

\end{document}

\begin{thebibliography}{9}
\bibitem{moretti2004}
Moretti, Enrico. “Workers’ Education, Spillovers, and Productivity: Evidence from Plant-level Production Functions.” American Economic Review (June 2004): 656-690.
\bibitem{moretti2010}
Greenstone, Michael, Hornbeck, Richard, and Moretti, Enrico. 2010. “Identifying Agglomeration Spillovers: Evidence from Winners and Losers of Large Plant Openings.” Journal of Political Economy 118(3):536-598.
\bibitem{ZuckerDarby}
Lynne G. Zucker and Michael R. Darby, "Movement of Star Scientists and Engineers and High-Tech Firm Entry," Annals of Economics and Statistics, No. 115-116, SPECIAL ISSUE ON KNOWLEDGE CAPITAL IN NANOTECHNOLOGY AND OTHER HIGH TECHNOLOGY INDUSTRIES (December 2014), pp. 125-175
\bibitem{djankov}
Simeon Djankov, Tim Ganser, Caralee McLiesh, Rita Ramalho and Andrei Shleifer, "The Effect of Corporate Taxes on Investment and Entreprenuership" American economic Journal: Macroeconomics. Vol. 2, No. 3 (July 2010), pp. 31-64
\end{thebibliography}
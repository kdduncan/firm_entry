\documentclass[12 pt, a4paper]{article}

\begin{document}

\title{Impacts of Taxes on Firm Entry Rates Along State Borders}
\author{Kevin D. Duncan}
\maketitle

Current research on the determinants of firm entry has focused at the use of complementary human capital in determining the impacts of agglomeration economies nr firm entry as shown in Moretti (2004) and Zucker and Darby (2014). Alternatively many studies focus on testing the incidence of one tax on firm entry such as Brulhart et al (2012) and Gjankov et al (2010). Currently there are no papers that try to estimate the impacts of an array of taxes on firm entry rates. Tax rates might be heavily related to both entrepreneurial decisions, and across tax types, such that focusing on one tax rate might bias estimates due to ommitted variables. 

This current gap is attributable to the lack of variation in tax and regulatory policies, which feature prolonged periods of stability across political regimes, or that might be endogenous to policy action in response to economic activity (Romer and Romer (2010), Mertens and Ravn (2013)). Bridging this gap in knowledge can provide researchers and policy makers a better understanding of how to attract new businesses and entrepreneurs, as well as the limitations of policy action.

Our paper fills this gap by using a regression discontinuity technique in order to estimate the policy effect of seven top marginal tax rates on firm entry behavior. We take the difference in firm start up rates on either side of a state border to control for location specific determinants of firm entry, such as labor market characteristics, and estimate a pure policy effect. Further, as we approach the border in the limit the policy discontinuity in taxes is strictly exogenous, lending this area prone for exploration. Similar techniques have been used in Holmes (1998), Rohlin (2011), and Dube et al (2010) to study right to work status and minimum wages. The method allows us to estimate the local average treatment effect that tax policies impose on entrepreneurs.

We use data on firm births from the US Census Bureau, along with a compiled data set of seven top marginal tax rates, including property, sales, income, corporate, capital gains, workers compensation, and unemployment insurance tax rates. We further include log expenditures per capita on highways, education, and welfare as an additional determinant of the sorting behavior of entreprenuers into preferred neighborhoods following McKinnish (2007). Next we estimate a model that takes the difference in log firm start ups and independent variables between randomly assigned subject and neighbor counties.

At the border we find that property, sales, and income taxes have a negative and statistically significant impact on firm start up rates, and that as the bandwidth increases the effect becomes indistinguishable from zero. These results remain significant even when run on NAICS subcodes and singe time period cross section analysis. 




\begin{thebibliography}{9}
\bibitem{moretti2004}
Moretti, Enrico. “Workers’ Education, Spillovers, and Productivity: Evidence from Plant-level Production Functions.” American Economic Review (June 2004): 656-690.
\bibitem{moretti2010}
Greenstone, Michael, Hornbeck, Richard, and Moretti, Enrico. 2010. “Identifying Agglomeration Spillovers: Evidence from Winners and Losers of Large Plant Openings.” Journal of Political Economy 118(3):536-598.
\bibitem{ZuckerDarby}
Lynne G. Zucker and Michael R. Darby, "Movement of Star Scientists and Engineers and High-Tech Firm Entry," Annals of Economics and Statistics, No. 115-116, SPECIAL ISSUE ON KNOWLEDGE CAPITAL IN NANOTECHNOLOGY AND OTHER HIGH TECHNOLOGY INDUSTRIES (December 2014), pp. 125-175
\bibitem{djankov}
Simeon Djankov, Tim Ganser, Caralee McLiesh, Rita Ramalho and Andrei Shleifer, "The Effect of Corporate Taxes on Investment and Entreprenuership" American economic Journal: Macroeconomics. Vol. 2, No. 3 (July 2010), pp. 31-64
\end{thebibliography}
\end{document}
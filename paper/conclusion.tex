\section{Conclusion}

This paper estimates the average impact of taxes on firm start up rates. Using a model that relies on the similarity of locations on either side of a state border, we are able to effectively control for location-specific determinants of firm entryin our empirical design, and more precisely isolate the effects of policy  that do vary on opposite sides of a state border. 

We find that counties with higher property, income, and sales taxes relative to a neighboring county in another state, have lower firm start-up rates. On average, a 1\% increase in the property tax differential decreases firm start up rates by 0.3\%, while a 1\% increase in income and sales tax differentials decreases firm start up rates by 0.01\%.   These results are generally consistent across industry groups, and time periods in our sample.  They are also largely robust to the addition of added controls.

Our estimated model's inability to find significant corporate and capital gains taxes follows from characteristics of new firm entrants. Lacking firm-level characteristics, our model approximates an average firm from the joint distribution of firm characteristics. However, most new firms are small S corporations, meaning that owners pay top marginal income taxes rather than corporate taxes, and firm employment and output is relatively low. Sales, income, and property taxes may play a significantly larger role on their profits than capital gains and corporate tax rates. Moreover, most new firms have a relatively short life span, such that investments in the company will probably not be recouped, and that capital gains tax rates are not likely to impact the majority of small new firm entrants.

Government expenditure variables do not seem to impact firm start up rates. This might be due to the fact that individuals can live in one county that has a preferred public expenditure bundle and set up a businesses in a neighboring county that has a preferred regulatory policy. Our robustness tests using Rohlin, Rosenthal, and Ross (2014) reciprocal agreements mirror our main model and estimates, indicating that this impact is largely not large.

Based on our estimates, we calculate a weighted tax differential, showing that the impact of taxes on firm entry rates remain small, only accounting for about 0.2\% of the difference in firm start up behavior across borders. Despite this, the side with the preferred taxation policy had more firm startups 62\% of the time in our sample. Therefore while taxes might have a small impact at the margin, their adjustment may still be beneficial to communities and states. 

Future work on this issue could benefit from more disaggregated, firm-level data with firm specific characteristics. This would help establish better estimates of tax incidence on firm start up and life cycle behavior. Our current estimates are limited by our set of covariates. Lacking firm specific data, our estimates rely on a proxy "average firm", which is most likely small and not paying corporate taxes, nor have venture capital backing. Thus taxes that may have impacts based on firm characteristics may be ommitted from our model.
\section{Conclusion}

Our paper tests the impact of taxes on firm start up rates. We present a model illustrating that when firm entry locations are close together and split across state borders, location specific determinants of firm entry become insignificant. This allows researchers to estimate policy effects. We estimate this model by taking the difference in county firm start up rates on opposite sides of a state border. This allows us to examine firm entry behavior around state borders by an approximate bandwidth.

In our empirical results, we included property, income, corporate, capital gains, workers compensation, and unemployment insurance top marginal tax rates. We further include log expenditures per capital on education, highways, and welfare. We control for state level agglomeration averages such as population density, fuel prices, union rate, and percent of population with a high school degree, as well as and county level geographic amenities, such as January temperature, July humidity, and log area with water.

Our estimated model shows that property, income, and sales taxes have the strongest determining factor on firm start up rates. On average, a 1\% increase in the difference in property taxes decreases firm start up rates by 0.3\%, while a 1\% increase in income and sales tax differentials decreases firm start up rates by 0.01\%. 

The fact that corporate and capital gains taxes are not significant follows from characteristics of new firm entrant. Generally many new firms are small S corporations, meaning that owners pay top marginal income taxes rather than corporate taxes, and that sales and property taxes may play a significantly larger role on their profits than capital gains and corporate tax rates. Also, the average new firm has a relatively short timeline, such that investments in the company will probably not be recouped, and that capital gains tax rates are not likely to impact the majority of small new firm entrants.

In our sensitivity tests, we found that coefficients are the same across counties. We also show that the sign, size, and significance of property and sales taxes remain consistent for each time period in our sample, while income taxes gain significant over time. Finally, we show that when we include an array of state-pair specific fixed effects all of our estimates become insignificant, but our tax variables remain the largest, keeping their sign and relative importance.

Government expenditure variables do not seem to impact firm start up rates. This might be due to the fact that individuals can live in one county that has a preferred public expenditure bundle and still set up a businesses in a neighboring county that has a preferred regulatory policy. Rohlin, Rosenthal, and Ross (2014) control for this by including reciprocal agreements in their specification, which require workers to pay their income tax in the state of residence rather than the state of employment. This may control for some of the sorting of entrepreneurs we observe.

We finally provided a weighted tax differential, showing that the impacts of taxes on firm entry rates remain small, only accounting for about 0.2\% of the difference in firm start up behavior across borders. Despite this, the side with the preferred taxation policy had more firm startups 62\% of the time in our sample. Therefore while taxes might have a marginally small impact, their adjustment may still be beneficial to communities and states. 

Going forward obtaining firm specific characteristics will help establish better estimates of tax incidence on firm start up and life cycle behavior. Generating county level agglomeration figures and testing their impacts might also be a way of estimating the interior impacts of tax differentials on firm start up rates. Finally, looking to test the welfare impacts of new firm entrants is important. Current theory is agnostic about the impacts of firm entry on welfare, and ensuring that policy changes improve lives required for program efficacy.
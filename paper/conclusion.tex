\section{Conclusion}

Our paper tests the impact of taxes on firm start up rates, and if firm entry seems to be dependent on government expenditures. We present a model illustrating when using regression discontinuity techniques around the border may provide identification for the impacts of government policies on firm start up rates. We then estimated both count data models and a model where we took the difference in county firm start up rates on opposite sides of a state border in a pseudo-regression discontinuity design. 

In our empirical results, we included an array of state top marginal tax rates, right to work status, and minimum wage as costs, and counterbalance it with spending per capital on education, highways, and welfare. We also included a variety of controls, such as geographic amenities, population density, fuel prices, union rate, and percent of population with a high school degree.

Our Count Data Model estimates show that property, capital gains, and corporate taxes impose a burden on the number of firm start up rates. Surprisingly, both higher minimum wage, and a lack of right to work status, imply higher firm start ups. Also, education and highway spending per cap lower firm start up rates, but welfare spending does not. Thus it is not directly clear when cutting or raising taxes pays for itself in increased public expenditures. The inclusion of scaled amenity variables does not strongly impact the significance or sign of terms, but does tend to drive coefficients to be lower by some margin. These results may be biased by the lack of constricting our current choice to just counties on the border, rather than all counties.

In these models we see a high fit and significance for almost all variables. There may be issues both in endogeneity, as well as unobserved characteristics that entice firms to enter into one market over another. Our border discontinuity design may correct for some of these obstacles. In this model we take the difference between two counties on either side of a state border. In these estimates property taxes, income taxes, and sales taxes have the strongest determining factor on firm start up rates. This coincides with the observation that many companies are small S corporations, such that in the short run considerations such as capital gains or corporate tax rates shouldn't impact the decision choices of most firms.

In our specification tests, we find that it is reasonable to assume that coefficients are the same across counties for our pooled estimator. We also show that the sign, size, and significance of property and sales taxes remain consistent for each time period in our sample. Finally, we show that when we include an array of state-pair specific fixed effects all of our estimates become insignificant, but our tax variables remain the largest, keeping their sign and relative importance.

Comparably government expenditure variables do not seem to impact firm start up rates. This might be due to the fact that individuals can live in one county that has a preferred public expenditure bundle and still set up a businesses in a neighboring county that has a preferred regulatory policy. This allows for min-maxing of results for aspiring entrepreneurs. in comparison with other studies, our minimum wage and right to work variables do not seem to impact firm start up rates, but compared to studies focusing on those variables we do not restrict our analysis to restaurants, or other predominately low wage or manufacturing sectors.

Going forward, we would like to do more empirical tests for the impacts of taxes on different types of industries, as is common among related literature (Dube et al (2010), Rohlin (2011). However using our border discontinuity approach limits how many observations and makes identification harder. Further, we would like to provide greater robustness checks within our empirical framework. A simple extension would be to find out how our estimates vary as we test counties further away from the border, though the outline in our theoretical section might dissuade such regressions as able to properly identify a treatment effect over possible changes in location specific terms.
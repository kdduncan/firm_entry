\documentclass[12pt,a4paper]{article}

\renewcommand{\baselinestretch}{2} 
\usepackage[margin=1.0in]{geometry}

\usepackage{amsmath}
\usepackage{amssymb}
\usepackage{relsize}
\usepackage{graphicx}
\usepackage{verbatim}
\usepackage{hyperref}
\usepackage{relsize}
\usepackage{amsthm}

\usepackage{pgffor}%
\usepackage{geometry}
\usepackage{pdflscape}

\usepackage[utf8]{inputenc}
\usepackage[english]{babel}

\newtheorem{theorem}{Theorem}[section]
\newtheorem{lemma}[theorem]{Lemma}
\newtheorem{proposition}[theorem]{Proposition}
\newtheorem{corollary}[theorem]{Corollary}
\newtheorem{assumption}[theorem]{Assumption}


\newenvironment{definition}[1][Definition]{\begin{trivlist}
\item[\hskip \labelsep {\bfseries #1}]}{\end{trivlist}}
\newenvironment{example}[1][Example]{\begin{trivlist}
\item[\hskip \labelsep {\bfseries #1}]}{\end{trivlist}}
\newenvironment{remark}[1][Remark]{\begin{trivlist}
\item[\hskip \labelsep {\bfseries #1}]}{\end{trivlist}}

\begin{document}


\title{Impacts of Taxes on Firm Entry Rates Along State Borders}
\author{Kevin D. Duncan \thanks{Address: 80A Heady Hall, Iowa State University, Ames, IA 50011. Email: \url{kdduncan@iastate.edu}, Web: \url{kdduncan.github.io}. I would like to thank Geogeanne Artz, Peter Orazem, David Keiser, and Brent Kreider for helpful comments and discussion.}}
\date{}
\maketitle

\begin{abstract}
This paper uses a regression discontinuity approach to estimate the impacts of taxes on firm entry rates between neighboring states. We utilize matched county pairs as an approximate bandwidth around the discontinuity in state policies imposed at their border. This controls for unobserved location specific determinants of firm entry, as well as policy responses to shocks shared across borders. We then estimate the impacts of property, income, sales, corporate, capital gains, workers compensation, and unemployment insurance top marginal tax rates on the differences in firm entry between counties. Our array of taxes controls for joint changes in tax policy that governments may implement to accomplish policy goals. We estimate this impact using a sample of 107 state-border pairs between 1999 and 2009. Our results indicate that property and sales taxes have the largest negative effect on firm start up rates, and in more recent year's income taxes as well.
\end{abstract}

\newpage

\section{Introduction} 

This paper tests whether or not taxes impact firm entry rates between neighboring states. Estimation of this marginal effect has been historically difficult. Tax and other policy parameters tend to feature prolonged periods of stability, and changes may be endogenous to many common dependent variables, such that changes in GDP, wages, and employment will entice government officials to try and improve economic performance. This has led to time series applications to use narrative approaches to try and identify the impacts of exogenous shocks to tax rates on macroeconomic variables (Romer and Romer (2007), Mertens and Raven (2013)). 

We use a border-differencing technique to establish local estimates of the impacts of taxes on firm entry rates. This method controls for endogeneity of government policy in response to local economic outcomes. For example, high economic activity states may raise their taxes knowing that local agglomeration factors will continue to attract an asymmetrically high amount of new firm startups, while low economic activity states may lower taxes to attract new businesses. This response would upwards-bias the estimate of the impacts of taxes. Using the differences in firm entry rates along state borders controls for local agglomeration factors, and treat differences in policy variables as exogenous.

This technique relies on the assumption that new firms pick entry locations within a local choice set. Recent studies on agglomeration economies seem to support this view, where Rosenthal and Strange (2003, 2005), and Arzaghi and Henderson (2008)) show that entrepreneurs weight locations within a mile of them significantly higher than distances further away. Use of border discontinuity designs started with Holmes (1998), though has quickly been adopted by researchers looking to identify effects of policies across public economics, including minimum wages (Dube et al (2008), Rohlin (2011)), welfare (McKinnish (2005, 2007)), and school quality by Dhar and Ross (2012). It has even been used by recent papers looking at the impacts of taxes on firm start up rates, including Rathelot and Sillard (2008), Duranton et al (2011), and Rohlin, Rosenthal, and Ross (2014).

Our paper builds on a longer literature looking at determinants of firm entry. Early papers such as Carlton (1979, 1983) and Schmenner (1975, 1982) failed to find incidence of taxes on firm entry rates, instead finding that higher taxes could attract more firms. Starting in the 80's methods and data  allowed for cleaner identification, such that authors started to more definitively show that taxes had an impact on business activity, including Wasylenko \& McGuire (1985), Bartick (1985), Papke (1991), and Hines (1996).

The literature studying the impacts of taxes on firm entry behavior has not settled on the best variables to use for identifying the effects of taxes on firm start up rates. Carlton (1983) used top marginal tax rates for corporate and income tax, but weighted them together, as well as property tax rates. Schmenner (1987) uses state and local property tax revenues per dollar of personal income. Helms (1985) used a budget constraint to estimate the impacts of rising tax revenue on explanatory variables. All three versions have modern equivalents and the literature has not settled on a single best practice to recover the proper marginal effects.

Theory indicates that marginal tax rates are what matter to individuals, and measures of average tax burden change due to both fluctuations in wages or profits, as well as to changes in tax rates.Using average tax rates may add endogeneity into models. Also, politicians may alter multiple taxes at once in order to accomplish policy goals, such that excluding taxes may imply omitted variable bias. Therefore, we argue that using top marginal tax rates is the preferred method of estimating marginal effects of taxes.

We use a data set that includes property, income, corporate, capital gains, sales, workers compensation, and unemployment insurance top marginal tax rates. We further add variables for government expenditures, including highways, education, and welfare. This mirrors the balanced budget approach of Helms (1985), Gabe and Bell (2004), and  Ojede and Yamarik (2012). Entrepreneurs sorting along by government expenditures would imply that increasing tax rates to pay for certain services may not have negative effects on firm start up rates.

The paper proceeds in the following manner. First, we review literature relating to estimating the determinants of firm entry. We then provide a model to show how utilizing discontinuities along state borders allow researchers to control for location specific determinants of firm entry. Next, we explain our empirical design, which uses matched county pairs on either side of state borders to identify the effects on taxes on firm start up rates. We then provide estimates of these impacts. We conclude by showing which borders have the largest firm start up discrepancy, and talk about applications of our work.

\section{Literature Review}

Location choice of firms and individuals has a rich history in economics. At its core, the question is what drives sorting behavior of firms into particular communities. One of the earliest models to approach sorting behavior of individuals was Tiebout (1956), who argued that individuals sorted into locations through preferences for prices and public amenities. Tiebout posited that counties as a result of this sorting behavior receive pressure on their provision of services in order to attract individuals.\footnote{Sorting literature similarly gave birth to tax competition among states as over viewed by Wilson (1999). Our paper can be seen as an extension of this literature, where states compete to have preferential tax differentials compared to neighboring states}

As a result Tiebout's model, the early firm entry literature focused on sorting over all available possible markets. These papers used conditional Logit models to explain the probability of agents entering into particular markets.  McFadden (1974) showed a general framework on how to use conditional Logit function to estimate firm entry sorting. Dowding, John, and Biggs (1994) review 200 articles and books that there is evidence that taxes and services affect location decisions of firms and households, and less assuredly, property values as is implied in the traditional Tiebout models.

Guimaraes, Figueirido, and Woodward (2003) showed that the conditional Logit models of firm entry can be estimated by a Poisson distribution under relatively mild assumptions. Gabe and Bell (2004) used Poisson and Negative Binomial regressions show how taxes and government spending on education impact firm location in Maine. They recover the coefficient on taxes by imposing a balanced budget requirement equivalent to Helms (1985). Their results show that increasing tax rates to raise education spending per pupil causes no distortion on firm entry rates. A review was on the differences between discrete choice models and Poisson models was done by Arauzo-Carod et al (2010). They show that as time has gone on, the discrete choice model has lost favor in favor of count data models. In their review they show that agglomeration and market size tend to have a significant positive effect, while wages and taxes act in the opposite direction.

Modern papers on firm entry paper have increasingly used border discontinuity methods. In these models researchers test whether or not local discontinuities in policy induce firms or individuals to change their start up or housing location. The first paper to use a matched county-pair estimator is Holmes (1998). In Holmes' paper, he uses right to work status as a proxy for an unobserved cost of being on either side of a state border imposed by "pro" and "anti" business policies. He then tested whether or not right to work status affected manufacturing employment growth. His estimates found that counties that have right to work status attract more manufacturing firms than states without right to work status.

Since Holmes' work, authors have utilized this technique to test a variety of expanding topics in public economics. A recent paper by Rohlin, Ross, and Rosenthal (2014) mirrors our paper very closely. They estimate a linear probability model of firm entry using a border difference estimator. They use GIS coded data to get a closer bandwidth to the border than our method, and show that increasing the personal income tax differential actually increases the likelihood of firms entering on one side of the border. However, they show that increasing the corporate and sales tax differential can drastically reduce the relative firm entry probability.

Earlier work by Rohlin (2011) looked at the impact of minimum wages on firm start up rates using less aggregated data. By utilizing the Dun and Bradstreet Marketplace data files Rohlin constructed bands around state borders, and then derived estimates on the impact of minimum wage changes on firm start up rates. He showed that increasing the minimum wage decreased new establishment activity in industries that relied heavily on minimum wage workers, but that changes in the minimum wage did not decrease employment in existing establishments.

Chirinko and Wilson (2008) use a border discontinuity technique to estimate the impacts of state investment tax credits on firm start up rates. Rathelot and Sillard (2008) use the border discontinuity technique in a Probit model to show that increasing the total tax rate differential increases the probability of a firm picking a side between 1-5\%. Duranton et al (2011) took the difference in the firm entry rate in neighboring areas to estimate the impacts of taxes on employment. They provide estimates both for traditional OLS estimates without the difference, which estimated a positive relationship between taxes and firm entry rates, but after applying the spatial difference, taxes negatively impacted firm start up rates.

From the literature, we see that on average taxes negatively impact firm start up rates, especially as researchers have gone from studying sorting over all available entry choices, to local choices along policy discontinuities. However, how taxes are calculated and used in studies differs wildly among authors. Various studies have used measures of average tax revenue, added together top marginal tax rates, or included a single available tax rate. As a result, we employ more recent spatial difference techniques to get a clear estimate, while employing a larger array of top marginal tax rates than other authors.
\section{Theory}

As entrepreneurs and firms look to start up a business in a new location they first choose a market to enter. This choice is due to primary considerations such as labor market characteristics, or location preferences of the owner. They then pick among possible locations in that market. Our model looks at choice of firm entry across state borders, such that individuals have mobility across the border. As a result, firms treat location specific determinants of profit as the same on both side of the border. This process leaves policy drivers as remaining difference in expected profits. We formalize the conditions for this process below.

Assume there exists a spatial equilibrium where wages and capital costs are adjusted to local tax and location specific variables affecting firm level productivity. If markets are competitive firms will make zero economic profit in the long run, but in the short run demand or policy shocks can leave short run profits. We expect that if a regime changes its taxes over time, higher production costs and lower profits exist in that county, and that market will deter a relative amount of firms from entering. Since firms will bid up or down prices relative to taxes, those prices can be proxied by the tax rates directly. Firms make decisions based on information from the previous year, as governments might concurrently change policy along with market entry and there may exist costs to establishing a business.

\begin{assumption}
Assume that a firms' profit can be expressed as a linear function, for a given location, state, and time pair denoted $(i,j,t)$,
\begin{equation}
\pi_{i,j,t} =  \gamma+\beta_{i}+\beta_{j}+X_{i,t-1}\beta_{1}+X_{j,t-1}\beta_{2}+\epsilon_{i,j,t}
\end{equation}
\begin{equation}
E[\epsilon_{ijt}] = 0
\end{equation}
$X_{i,t-1}$ is a $1 \times K_{1}$ row vector of location specific terms, and $X_{j,t=1}$ is a $1 \times K_{2}$ row vector of state specific terms, and $\beta_{i}, \beta_{j}$ are location and state specific fixed effects.
\end{assumption}

Location specific variables are any variable that is specific to a location, such as local agglomeration figures, education attainment, and other variables driven by the distribution of labor and productive factors in each regime. Variables at the regime level include taxes, regulatory policies, and government expenditures. Both sets of variables are allowed to evolve over time. Therefore this assumption simply states that our policy variables have to enter directly into the profit function, and that it is shared across all firm types.

Now let us focus on a market that is defined by the interval $[-1,1]$, such that for $i \in [-1,0)$ a firm is in state $A$, and for $i \in [0,1]$, they are in state $B$. Therefore, if a firm has two choices, $y \in [-1,0)$ and $\hat y \in [0,1]$, then the firm chooses $y$ over $\hat y$ if
\begin{equation}\label{diff}
E[\pi_{y,A,t}-\pi_{\hat y,B,t}] > 0
\end{equation}

\begin{assumption}\label{cont}
$\beta_{i}$ and $X_{i,t-1}$ are continuous locally on $[-1,1]$, such that for any $\epsilon > 0$, where  $|\beta_{i}-\beta_{j}| < \frac{\epsilon}{K+1}$, and $|(X_{y,t-1,k}-X_{\hat y,t-1,k})| < \frac{\epsilon}{(K+1)|\beta_{k}|} \forall k \in \{1,...,K_{1}\}$, then there exists a $\delta$ such that $|y - \hat y| < \delta$
\end{assumption}

This statates that as the locations firms choose between get asymptotically close to the border, the difference between unobserved location specific fixed effects and observed location specific variables converge to zero. This is a technical illustration of labor and capital mobility in close geographic areas. As the distance between the two locations increases this may no longer be the case, as illustrated in Holmes (1998).

Therefore, conditional on firms choosing locations $(y, \hat y)$ arbitrarily close to the border, the profit function becomes,

\begin{equation}\label{prof}
E[\pi_{y,A,t}-\pi_{\hat y, B, t}] =  \beta_{A}-\beta_{B}+(X_{A,t-1}-X_{B,t-1})\beta_{2}
\end{equation}

As we move away from the border location characteristics might dominate the policy effect, especially when we expect policy effects to be small. This theory favors the use of regression discontinuity techniques for estimating policy treatment effects, especially when location specific drivers of firm entry might be unknown or unobserved.
\section{Variables and Data}

\subsection{Matching Process}

Our matching procedure is as follows. We first obtained Census county adjacency files.\footnote{\url{https://www.census.gov/geo/reference/county-adjacency.html}}, then used it to construct county-pairs by generating all pairs of counties that have adjacent counties in a neighboring state. From this matching we also tracked state FIPS codes to create a list of state pairs. For each state-pair we assigned one side of a border to be either a subject ($sub$) or neighbor ($nbr$) side of the border, which we use in our data construction. This matching generates 1213 matched county-pairs with 107 state-pairs in each year.

We then generated an extended border match. For this process we matched each subject county to each of its neighbor's neighbor, then excluded from any county in the original matching set. We provide a graphical representation of these matching processes in Figure \ref{rb}. This extended match connects 1549 county-pairs across 107 state pairs each year.

\subsection{Firm Entry Data}

Our primary variable of interest were county level firm start up rates for all firms in a year. This data set was procured at the Census Bureau's Business Dynamic Statistics program.\footnote{\url{http://www.census.gov/ces/dataproducts/bds/overview.html}} The data included the number of firm births, deaths, expansions, and contractions for each year from 1999 to 2013. It also provided these figures for  broad NAICS coded industries. As a result, our main variable of interest, $births\_ratio$ is calculated as,

\begin{equation} births\_ratio_{t} = \ln(n_{sub,t})-\ln(n_{nbr,t})\end{equation}

\subsection{Tax Data}

We included top marginal tax rates of seven taxes from 1977 to 2008. In all cases we used a one period lagged difference in top marginal values. For each tax rate $\tau_{i}$ and state pair $(A,B)$, at time $t$ the tax ratio was calculated 

\begin{equation} tax\_ratio_{A,B,t} = \tau_{A,t}-\tau_{B,t} \end{equation}

State marginal income tax and long term capital gains tax rates were obtained from The National Bureau of Economic Research. For income tax rates we used the highest marginal tax rates available, as this is the rate most applied to small business and S corporations. When not available, we calculated the highest implied tax rate. \footnote{\url{http://users.nber.org/~taxsim/allyup/} \url{http://users.nber.org/~taxsim/marginal-tax-rates/} \url{http://users.nber.org/~taxsim/state-marginal/}}

Corporate and sales tax rates were compiled from The Council of State Governments Book of States\footnote{\url{http://knowledgecenter.csg.org/kc/category/content-type/content-type/book-states}}. We used the highest marginal state tax rates on business corporations. Where rates differ between banks and non-banks, we use the non-bank rate, and we restrict to sales tax rates levied on general merchandise, rather than food, clothing, or medicine. 

Property taxes were calculated from household level data provided by the Minnesota Population Center's Integrated Public Use Micro-data Series (IPUMS).\footnote{\url{https://usa.ipums.org/usa/}} Workers compensation was calculated from Thomason et al (2001) between 1977 and 1995, with data afterwards provided by the Oregon Department of Consumer and Business Services. 

Finally, The top marginal unemployment insurance tax rates were provided by the US Department of Labor. To calculate, they multiplied the top marginal tax rate, $\tau_{u,it}^{max}$, by the maximum wage level to which the rate is applied, $W_{it}^{max}$. They normalized this figure by the average wage in a state in a current year, $\bar W_{it}^{max}$. Then the unemployment insurance tax was calculated as;
\begin{equation} \tau_{u,it} = \frac{\tau_{u,it}^{max}W_{it}^{max}}{\bar W_{it}^{max}}\end{equation}

\subsection{Government Expenditures}

We compiled log state governments expenditures on highways, education, and welfare per capita using Census data on State Government Finances.\footnote{\url{https://www.census.gov/govs/state/}} We used expenditures on "Education" for our education value, welfare sums up expenditures on "Public Welfare", "Hospitals," and "Health," while highways is calculated from "Highways" expenditures pulled from annual historical data accounts. To calculate per capita terms we divided each figure by Census state population estimates,\footnote{\url{http://www.census.gov/popest/}} and then took logs. For each of our expenditure figures, the state differenced variable for two states and time t was calculated as,

\begin{equation} exp\_percap\__{A,B,t} = \log(exp_{A}/pop_{A}) - \log(exp_{B}/pop_{B}) \end{equation}

\subsection{Additional Controls}

As a final series of controls, we included state level variables for percent of workforce unionized, log real fuel prices, population density, percent of industry manufacturing, and percent of population with high school education. This data was collected from "Union Membership and Coverage Database from the CPS."\footnote{\url{http://www.unionstats.com/}}

Lastly, amenity data was acquired from the USDA.\footnote{\url{http://www.ers.usda.gov/data-products/natural-amenities-scale.aspx}} We used normalized values of hours of sunlight in January, temperature in July, humidity in July, topology score, and percent of county that is water. After normalization each amenity variable is normal with approximate mean zero and standard deviation 1. As a result, interpretation of these terms should be done in terms of deviations from the mean. Again, we take difference in county level Z-scores, and it is the only county level data we include in our empirical estimates.

\subsection{Preliminary Analysis}

Summary statistics are provided in Table (\ref{--summary}).

We test the hypothesis that states use taxes jointly to accomplish policy goals. We plot simple cross correlations between our differenced tax variables in Table \ref{pairs} as a heuristic test. Between 1998 and 2008, income tax and capital gains tax rates exhibit highly positively correlation, the simple correlation between values is 0.64. We further see that sales, payroll, workers compensation, and unemployment insurance tax rates have low rates of correlation with other tax rates. 

The presence of simple correlations indicate policy makers might have shifted taxes jointly to accomplish policy goals and tried to advantageously shift tax incidence. Thus, modeling firm entry using a larger set of top marginal tax rates will improve estimates of tax incidence on firm start up rates.
\section{Empirical Design}

Our parameters of interest are the coefficients on our top marginal tax rates. We provide two approaches to identify these parameters. We first estimate traditional count data models using Poisson and Negative Binomial Maximum Likelihood Estimation. We then explain an alternative approach to identification using a regression discontinuity approach at state border.

The count data model works by showing that under a spatial equilibrium firm entry behaves like a Conditional Logit Model of entry, where there is a probability of a firm entering into particular locations. Other research has shown that by assuming structure on the profit function, we can show that estimating the conditional Logit model provides the same coefficients as estimating a Poisson regression. Our regression discontinuity approach follows directly from our theoretical section. This method takes the difference in variables from counties on either side of a state border. The border provides a sharp discontinuity in policies firms face, thus we treat counties as a closest bandwidth around the discontinuity.

Consistent with most panel data work, major obstacles include several levels of unobserved heterogeneity, and our reduced form estimates might have issues in proper accounting for the incentives to locate in a particular location. Our preferred estimator might solve some of these problems.

\subsection{Count Data Models}

Traditional firm location choice literature is motivated by firms entering across all possible locations in the market using the profit function described in (\ref{profit2}). Our set up follows from McFadden (1974) and Wooldridge (2010, pp 619). This is done in the following fashion. First, we assume a profit function equivalent to equation \ref{profit2}, where we assume $\epsilon_{ijt}$ takes on an extreme-type-value-I distribution Weibuill distributed. Let us have $f = 1,...,F$ number of firms trying to enter in a given time period. Each state in the US is denoted as a regime, $j$, and each county is a location $i$.  Let us index them $ij = 1,...,N$. Then the probability of a firm $f$ locating at point $ij$ in period $t$ is;
\begin{equation}\label{condlogit}
p_{f,ij,t} = \frac{\exp(X_{i,j,t-1}\beta)}{\sum_{ij}\exp(i,j,t-1)}
\end{equation}
Now let $d_{f,ij,t} = 1$ if a firm $f$ enters at point $ij$ in period $t$, and $d_{f,ij,t} = 0$ otherwise. Further, let us assume that there is no time dependence element, such that we can run this as $TN$ independent events. Then the log likelihood becomes
\begin{equation}\label{loglike}
\log L_{cl} = \sum_{ij=1}^{TN}\sum_{ij}d_{f,ij,t}\log p_{f,ij,t}
\end{equation}
Here we are assuming a strong assumption that the vector of parameters $X_{i,j,t-1}$ is the same for all types of firms. Guimaraes, Figueiredo, and Woodward (2003) show that in the case where the profit function depends on the same characteristics across all firms, that (\ref{loglike}) becomes a Poisson distribution consistent up to a constant as long as we believe that firm entry is directly a Poisson distribution as well. This happens as equation (\ref{loglike}) becomes,
\begin{equation}
\log L_{d} = \sum_{ij}n_{j}\log p_{ij}
\end{equation}
With $n_{ij}$ being the number of firms that open up in location $ij$, let us assume;
\begin{equation}\label{pois}
E[n_{ij}] = \exp(X_{i,j,t-1}\beta)
\end{equation}
They show that the log likelihood of the Poisson distribution becomes proportional up to a constant of the conditional Logit. 
Thus estimating a count data Poisson model is equivalent to estimating a full conditional log likelihood. This nice feature allows a fast and easy approach to identify the impacts of our tax variables.  Taking the exponent of (\ref{pois}) gives us the Kernel to a Poisson distribution, with $E[n_{ijt}] = X_{i,j,t-1}\beta$. As a result, the Poisson likelihood takes on the form,
\begin{equation}
L(\theta|X,N) = \mathlarger \Pi_{i=1}^{m}\frac{e^{n_{i}\beta'n_{i}}e^{e^{-\beta'x_{i}}}}{n_{i}!}
\end{equation}

We proceed by estimating a Poisson distribution in Table \ref{table:pois}.\footnote{Currently our estimates utilize just our matched county pair data so limits the firms' decision to just counties on the border of a state. However we plan to extend this estimation procedure over all counties. Due to time considerations leading to the existing draft of the paper this has not yet been done, but will be accomplished shortly.} We find that the the model is over dispersed by using Cameron \& Trivedi (1990) regression based test for over dispersion. As a result, we run Negative Binomial regressions using the same expected value. This gives us space to relax the assumption that $E[n_{ij}] = Var[n_{ij}]$, and allow our errors to take on a more general shape.

\subsection{Regression Discontinuity Approach}

There are several issues with this approach. First, firm entry may be heavily dependent on terms such as population. Similarly, individuals may place preference in areas that have been experiencing large job growth. Finding instruments for these interactions can be difficult. Further, there may be other unobserved heterogeneity at the location level that is unobserved by the researcher. Regression discontinuity techniques are a way to possibly control for these variables.

By our theory we know that location specific terms, an terms shared across observations get canceled out as we take the difference while approaching the borer. Our data does not allow us to get a closer estimation to the true discontinuity than those provided by the borders of the county. The average county in our data set is 1260 square miles, or about 35 miles per side of believed to be approximately square. This distance is slightly longer than more refined approaches such as Rohlin (2011). In practice we match up counties on either side of a state border, let us denote them subject ($sub$) and neighbor ($nbr$), and their states $stA$ and $stB$. Then, taking differences, we get by applying Theorem \ref{thrm}

\begin{equation}
\ln(n_{sub,stA,t})-\ln(n_{nbr,stB,t}) = \beta_{stA}-\beta_{stB}+(X_{stA,t-1}-X_{stB,t-1})\beta_{2} + \epsilon_{sub,stA,t}-\epsilon_{nbr,stB,t}
\end{equation}

First, let us index each state-pairs $(stA,stB)$ by $g$. Next let us assume that $\beta_{stA}-\beta_{stB} = \beta_{0}$ for all $sub, nbr$ pairs. Since we assign $sub$ and $nbr$ arbitrarily, this implies that $\beta_{stA} = \beta_{stB}$.  Then we make the following definitions.
\begin{equation}
\ddot \ln(n_{i,g,t}) = \ln(n_{sub,stA,t})-\ln(n_{nbr,stB,t})
\end{equation}

\begin{equation}
\ddot X_{g,t-1} = \beta_{0}+(X_{stA,t-1}-X_{stB,t-1})
\end{equation}

\begin{equation}
\ddot \epsilon_{i,g,t} = \epsilon_{sub,stA,t}-\epsilon_{nbr,stB,t}
\end{equation}

Assume $\ddot \epsilon_{i,g,t}$ be an i.i.d white noise draw, then let $\ddot X_{g} = (\ddot X_{g,0}',...,\ddot X_{g,T-1}')'$ be a $T \times (1+K_{j})$ matrix, and $\ddot \epsilon_{ig} = (\ddot \epsilon_{i,g,1},...,\ddot \epsilon_{i,g,T})'$ be a $T \times 1$ vector. Next we assume the traditional OLS moment conditions.

\begin{assumption}\label{noend}
Let  $\ddot X_{g} = (\ddot X_{g,0}', ... ,\ddot X_{g,T-1}')'$ be a $T \times (1+K_{j})$, and $\ddot \epsilon_{i,g} = (\ddot\epsilon_{i,j,1},...,\ddot\epsilon_{i,j,T})'$ a $T \times 1$ vector. Then 
\begin{equation}E[\ddot X'\ddot \epsilon] = 0, \quad \forall i,g\end{equation}
\end{assumption}

\begin{assumption}\label{fullrank}
 \begin{equation}E[\ddot X_{g}'\ddot X_{g}] = 1+K_{j}: \quad \forall g\end{equation}
\end{assumption}

We can estimate a pooled OLS estimator using Assumption's \ref{noend} and \ref{fullrank}. This gives us the POLS estimator;
\begin{equation}\label{pols}
\hat \beta_{2} = \left(\frac{1}{N^{*}} \sum_{k=1}^{T}\sum_{i=1}^{G}\sum_{j=1}^{N_{G}}\ddot X_{g,t-1}'\ddot X_{g,t-1}\right)^{-1}\left(\frac{1}{N^{*}}\sum_{k=1}^{T}\sum_{i=1}^{G}\sum_{j=1}^{N_{G}}\ddot X_{g,t-1}'\ddot \ln(n_{igt})\right)
\end{equation}
\begin{equation}
N^{*} = T(\sum_{g}^{G}N_{g})
\end{equation}

Donald and Lang (2007) show that increasing individual observations for each group doesn't provide better inference. They use a two stage estimator where they first calculate the mean for each side to show asymotitics with respect to the number of groups.Our estimator is a mean weighted version of their two stage estimator. We can rewrite \ref{pols} as:
\begin{equation}\label{pols_2s}
\hat \beta_{2} = \left(\frac{1}{TG} \sum_{t=1}^{T}\sum_{g=1}^{G}\frac{\sum_{i=1}^{N_{g}}\ddot X_{g,t-1}'\ddot X_{g,t-1}}{E[N_{g}]}\right)^{-1}\left(\frac{1}{TG}\sum_{t=1}^{T}\sum_{g=1}^{G}X_{g,t-1}'\frac{\sum_{i=1}^{N_{G}}\ddot \ln(n_{igt})}{E[N_{g}]}\right)
\end{equation}

\begin{equation}
E[N_{g}] = \frac{\sum_{g=1}^{G}N_{g}}{G}
\end{equation}
Compared to Donald and Lang's two stage estimator we underweight observations we observe only a few times compared to their true mean, and overweight observations we see many times compared to their true mean. Increasing $N_{g}$ for some $g$ doesn't improve our estimator, and only increase $E[N_{g}]$. Trying to keep $E[N_{g}]$ static requires making our asymptotics with respect to the number of group-pairings we have, $G$.

When doing inference there may be shocks to the state-pair border, for example the Mississippi river flooding along a border pair, but not shared with all other pairs in the sample. Therefore we use clustered errors on the state pair. Let $\ddot X$ be the $(\sum_{g}^{G}N_{G} \times T) \times (1+K_{j}) $ regressor matrix. Thus our variance co-variance matrix takes on the form
\begin{equation}\label{var}
\hat V =\frac{1}{G-2}\frac{\sum_{g=1}^{G}N_{g}-1}{\sum_{g=1}^{G}N_{g}-2}(\ddot X'\ddot X)^{-1}(\sum_{t}^{T}\sum_{g}^{G}u_{s}u_{s}')(\ddot X'\ddot X)^{-1}
\end{equation}
\begin{equation}\label{error}
u_{s} = \sum_{i}\hat \ddot \epsilon_{i,j,t-1}\ddot X_{g,t-1}
\end{equation}
We assume this lag structure to indicate that firms respond to variables from the previous time period, and as they are starting up government's may choose to alter policies for the current year. In practice though most of our variables are heavily inter-temporally correlated, so no major difference occurs in sign, significance, or fit appears from using different lag structures.

\subsection{Sensitivity Tests}

We subject our estimator to a series of robustness checks. For all of our regressions, we test models with and without amenities. We want to check whether or not our tax and regulatory variables become statistically insignificant once we account for these additions, and in our second model check whether or not they properly become indistinguishable from zero. Next, we test a version of this model where we do not impose the coefficients are the same across borders.
\begin{equation}\label{sense1}
\ddot \ln(n_{g,t}) = X_{stA,t-1}\beta_{sub}+X_{stB,t-1}\beta_{nbr}+ e_{igt} 
\end{equation}

Instead we let coefficients take on their own value in the difference, and do a set of F-tests on whether or not our assumption that $\beta_{i,A} = -\beta_{i,B}$ holds in the difference as assumed. The results of this regression are reported in Table \ref{table:equal}. Corresponding F tests are presented in Table \ref{table:Ftests}. Next we run our regression discontinuity estimator while forcing the coefficients to be the same. We present results for this model in \ref{table:canon}.

In Table \ref{table:stable} we test a set of regressions where we estimate period-specific coefficients and compare them to our pooled estimator to try and estimate of whether or not it is safe to assume that profit parameters are roughly stable over time. 
\begin{equation}\label{sense2}
\ddot \ln(n_{g,t})  = X_{stA,t-1}\beta_{stA}+X_{stB,t-1}\beta_{stB}+ e_{i,g,t}: \quad t = 1999,...,2008
\end{equation}
Which leads to the POLS coefficient;
\begin{equation}
\hat \beta_{2} = \left(\frac{1}{G}\sum_{i=1}^{G}\frac{\ddot X_{g,t-1}'\ddot X_{g,t-1}}{E[N_{g}]}\right)^{-1}\left(\frac{1}{G}\sum_{i=1}^{G}\ddot X_{g,t-1}'\frac{\sum_{j=1}^{N_{G}}\ddot \ln(n_{igt})}{E[N_{g}]})\right)
\end{equation}
Next we test a version of our model that includes a dummy variable for each state-pair in our sample. By construction of our estimator, we are claiming that any county level fixed effects take the form of location specific terms, which have to cancel out when we take the difference but state specific fixed effects may remain. We favor using dummy variables over Fixed Effect or First Difference transformations because our policy variables are incredibly stable over time. States very rarely change tax policies, and correlations with current tax rates with each of five periods of lags shows that taxes even at their weakest are still more than 85\% correlated with each other. Equivalently, right to work status changes once in our sample, and minimum wages rarely alter at the state level as well. Therefore these transformations do not provide enough variation to get valid inference.

Finally, we do not test for general endogeneity where states change taxes in response to the difference in firm entry rates. This is because the aforementioned stability of all of our policy parameters, it seems unlikely that they are responding to comparatively more volatile firm start up rates. Further, there is no reason to assume counties favor one set of borders over any other, unless counties find themselves systemically at a loss compared to neighbors, a corner solution we do not check for.
\section{Results}

Our main results are reported in Table (\ref{--rd}). The first four columns respond to different pooled OLS estimates where we include or exclude our set of control or amenity variables. The last two columns report our fixed effect estimates. Our pooled OLS estimates show that the inclusion of the geographic amenities makes property taxes lose statistical significance. However, the results still economic intuition that most likely the impacts are small and negative across all of our model estimates. Averaging across models would imply that a 1\% increase in the relative property tax difference would decrease firm start up rates by around 0.2\%. The impacts of income and sales tax differentials remain relatively stable across our OLS estimate, such that a 1\% increase in income tax differentials correspond to a 0.8\$ decrease in the relative firm start up rates, and similarly a 1\% increase in sales tax differentials corresponds to a 0.1\% decrease in the relative firm start up rates. Even though capital gains, corporate tax, workers compensation, and unemployment insurance tax rates are individually insignificant, joint F-tests for all seven taxes show they are jointly significant.

We further see evidence that the difference in log welfare spending per capita is also statistically significant, but the coefficient is economically very small, such that a 1\% increase in the difference corresponds to 0.001\% higher firm entry rates. Finally, contrary our assumptions, not all of our county level geographic amenities and state level controls become zero at the border. The difference in log real fuel price remains positive and statistically significant, and both the difference in Temperature in January and Log Area with Water remain significant among the geographic controls.

When we run models with state-pair level fixed effects we fail to obtain any statistically significant results. However, the value of these models are dubious. We first argue that our pooled OLS estimates are most likely the properly specified model as firm start up rates are an already differenced estimate. Thus the inclusion of state pair fixed effect require year to year divergence in expected profit from entry, which shouldn't occur under perfect competition. Rather this still might imply that there are still relevant variables we may be leaving out of our model.

Table \ref{--eb} we estimate the extended bandwidth version of our model. We expect that the longer distance between two locations will make taxes have a smaller impact on firm start up rates, while traditional measures of state or local agglomeration economies will have a larger impact. Consistent with this, we see that our tax rates become less individually statistically significant across model types. Further, our state level controls remain largely insignificant, as do our geographic controls. Thus, the fit of the model at large seems to decrease as the distance between counties increases.

When we run pooled OLS estimates where we do not impose that coefficients we see that for most of our variables remain equal but opposite across the border. Table \ref{--noequality} reports coefficients, while Table \ref{--Ftests} provides F tests for the assumption of the coefficients being the same across borders. We test for each variable that $\beta_{i,sub} = - \beta_{i,nbr}$. The results verify our belief that coefficients are the same and opposite in our design is a valid assumption. The exception is sales tax rates and workers compensation tax rates, for the subject county they are strongly and negatively significant, but for the neighbor they not significant at all. However, given that the rest of them pass, this might be a spurious result due to the number of regressors. We see an equivalent note in the workers compensation figures in our F tests, where for the neighboring county it appears to be significant, but not for the subject county.\footnote{Also, the assignment process here might be driving results. We are not running each coefficient as a fixed effect for each border, but rather across all counties defined as "neighbor" in our sample. However, by using clustered standard errors we do not have the degrees of freedom to run this test for each state-pair.} 

Table \ref{--year} shows regression results for $births\_ratio$ for the every year between 1999 and 2009. We use the model that includes state controls but excludes geographic amenities. We see that property taxes remain consistently negative and statistically significant. Sales tax rates remain negative and statistically significant, and even appears to grow in its deterrance of new entry. Income taxes start off insignifcant, but negative, and become statistically significant from zero. Log highway and welfare expenditures per capita vary in their significance across the sample, but remain positive drivers of firm entry when they appear.

Finally, in Table \ref{naics} we report an estimated model equivalent to Equation \ref{pref}, but where we condition firm entry on specific NAICS subcodes. For our reported estimates we include Agriculture, Fishing, Forestry, and Hunting, Retail Trade, Manufacturing, and Finance and Insurance. We find that our initial results in Table \ref{--rd}, including magnitude and strength. This is somewhat surprising, as we would expect characteristics that drive firm entry to differ across firm types. Namely, property taxes may deter agriculture more than financial firms, however it appears that the correlation between different firm types superscedes this selection.

As a final output of our paper, we compare two different rankings. First we calculate the weighted tax differential by multiplying the tax coefficients from Table \ref{--rd}, column 4 times each states marginal tax values. These are plotted in Figure \ref{weightedtax}. We see that for most states the weighted tax differential is very small, thus the implied impact of taxes on relative firm start up rates is ultimately small. However, for a few counties, this is not the case, and we see clear outliers where more than 1\% of the differential is motivated by the difference in tax rates. 

To calculate How important this effect is still aggregately we provide a table of the difference in the mean number of firm start ups along each state border, as well as the weighted tax differential. Since we calculate these terms in absolute value, we similarly show which side of the border is preferred for the borders with the top 50 largest difference in mean firm start ups. This ranking is provided in Table \ref{taxdifferential}. We seen that 62\% of the time the side with the preferred weighted tax differential also has the higher mean firm start up differential.

\section{Conclusion}

Our paper tests the impact of taxes on firm start up rates, and if firm entry seems to be dependent on government expenditures. We present a model illustrating when using regression discontinuity techniques around the border may provide identification for the impacts of government policies on firm start up rates. We then estimated both count data models and a model where we took the difference in county firm start up rates on opposite sides of a state border in a pseudo-regression discontinuity design. 

In our empirical results, we included an array of state top marginal tax rates, right to work status, and minimum wage as costs, and counterbalance it with spending per capital on education, highways, and welfare. We also included a variety of controls, such as geographic amenities, population density, fuel prices, union rate, and percent of population with a high school degree.

Our Count Data Model estimates show that property, capital gains, and corporate taxes impose a burden on the number of firm start up rates. Surprisingly, both higher minimum wage, and a lack of right to work status, imply higher firm start ups. Also, education and highway spending per cap lower firm start up rates, but welfare spending does not. Thus it is not directly clear when cutting or raising taxes pays for itself in increased public expenditures. The inclusion of scaled amenity variables does not strongly impact the significance or sign of terms, but does tend to drive coefficients to be lower by some margin. These results may be biased by the lack of constricting our current choice to just counties on the border, rather than all counties.

In these models we see a high fit and significance for almost all variables. There may be issues both in endogeneity, as well as unobserved characteristics that entice firms to enter into one market over another. Our border discontinuity design may correct for some of these obstacles. In this model we take the difference between two counties on either side of a state border. In these estimates property taxes, income taxes, and sales taxes have the strongest determining factor on firm start up rates. This coincides with the observation that many companies are small S corporations, such that in the short run considerations such as capital gains or corporate tax rates shouldn't impact the decision choices of most firms.

In our specification tests, we find that it is reasonable to assume that coefficients are the same across counties for our pooled estimator. We also show that the sign, size, and significance of property and sales taxes remain consistent for each time period in our sample. Finally, we show that when we include an array of state-pair specific fixed effects all of our estimates become insignificant, but our tax variables remain the largest, keeping their sign and relative importance.

Comparably government expenditure variables do not seem to impact firm start up rates. This might be due to the fact that individuals can live in one county that has a preferred public expenditure bundle and still set up a businesses in a neighboring county that has a preferred regulatory policy. This allows for min-maxing of results for aspiring entrepreneurs. in comparison with other studies, our minimum wage and right to work variables do not seem to impact firm start up rates, but compared to studies focusing on those variables we do not restrict our analysis to restaurants, or other predominately low wage or manufacturing sectors.

Going forward, we would like to do more empirical tests for the impacts of taxes on different types of industries, as is common among related literature (Dube et al (2010), Rohlin (2011). However using our border discontinuity approach limits how many observations and makes identification harder. Further, we would like to provide greater robustness checks within our empirical framework. A simple extension would be to find out how our estimates vary as we test counties further away from the border, though the outline in our theoretical section might dissuade such regressions as able to properly identify a treatment effect over possible changes in location specific terms.
\renewcommand{\baselinestretch}{1} 

\newpage
\begin{thebibliography}{99}

\bibitem{Tiebout56}
Tiebout, CM, "A Pure Theory of Local Expenditures" Journal of Political Economy 1956

\bibitem{Zodrowetal}
George R. Zodrow, Peter Mieszkowski, "Pigou, Tiebout, property taxation, and the underprovision of local public goods," Journal of Urban Economics 1986

\bibitem{BanzhafWalsh}
Banzhaf, H. Spencer; Walsh, Randall P., "Do People Vote with Their Feet? An Empirical Test of Tiebout's Mechanism," American Economic Review (2008), Vol. 98 No. 3, pp 843-863

\bibitem{Holmes98}
Holmes, Thomas J, "The Effect of State Policies on the Location of Manufacturing: Evidence from State Borders," Journal of Political Economy, Vol. 106, No. 4 (August 1998), pp. 667-705

\bibitem{Dubeetal}
Dube, Arindagit; Lester, T. William; Reich, Michael, "Minimum Wage Effects Across State Borders: Estimates Using Contiguous Counties," The Review of Economics and Statistics, Vol. 92 No. 4, (November 2010) pp. 945-964

\bibitem{McKinnish2005}
McKinnish, Terra. "Importing the Poor Welfare Magnetism and Cross-Border Welfare Migration." Journal of Human Resources, Vol 40, No. 1 (2005): 57-76

\bibitem{McKinnish2007}
McKinnish, Terra. "Welfare-induced migration at state borders: New evidence from Micro Data." Journal of Public economics, Vol 91, N0. 3 (2007): 437-450

\bibitem{Orazemetal}
McPhail, Joseph E.; Orazem, Peter; Singh, Rajesh, "The Poverty of States: Do State Tax Policies Affect State Labor Productivity?" WP \#10010, May 2010, \url{https://www.econ.iastate.edu/research/working-papers/p11552}

\bibitem{McFadden1974}
McFadden, Daniel, "Conditional Logit Analysis of Qualitative Choice Behavior," in P. Zarembka (ed.), Frontiers in Econometrics, 105-142, Academic Press: New York, 1974.

\bibitem{Nevo2000}
Nevo, Aviv; "A Practioner's Guide to Estimation of Random Coefficients Logit Models of Demand," Journal of Economics \& Management Strategy, Vol 9, Num 4, Winter 2000, 513-548

\bibitem{Dowdingetal}
Keith Dowding, Peter John, Stephen Biggs; "Tiebout: A Survey of the Empirical Literature," Urban Studies, Vol 31, Issue 4-5, 1994, pp 767-797

\bibitem{Kuminoffetal}
Kuminoff, Nicolai, Smith, V. Kerry; Timmins, Chris, "The New Economics of Equilibrium Sorting and Policy Evaluation Using Housing Markets"Journal of Economic Literature 51(4), 2013: 1007-1062   

\bibitem{ArauzoCarodetal}
JM Arauzo-Carod, D Liviano-Solis, M Manjon-Antolin; "Empirical Studies in Industrial Location: An Assessment of Their Methods and Results," Journal of Regional Science 2010, 50 (3), 685-711

\bibitem{GabeBell2004}
Gabe, Todd M; Bell, Kathleen P., "Tradeoffs Between Local Taxes and Government Spending as Determinants of Business Location," Journal of Regional Science, Vol. 44, pp. 21-41, February 2004 

\bibitem{Guimareaes2003}
Paulo Guimarães, Octávio Figueirdo, Douglas Woodward, "A Tractable Approach to the Firm Location Decision Problem," The Review of Economics and Statistics, February 2003, Vol. 85, No. 1, Pages 201-204

\bibitem{Brulhartetal}
Marius Brülhart, Mario Jametti and Kurt Schmidheiny, "Do Agglomeration Economics Reduce the Sensitivity of Firm Location to Tax Differencials?" The Economic Journal, Vol. 122, No. 563 (September 2012), pp. 1069-1093

\bibitem{Rohlin2011}
Rohlin, Shawn M. "State Minimum Wages and Business Location: Evidence from a Refined Border Approach," Journal of Urban Economics, Vol. 69 (2011), pp. 103-117

\bibitem{AudretschMahmood}
Audretsch, David B.; Mahmood, Talat; "New Firm Survival: New Results Using a Hazard Function," The Review of Economics and Statistics (Feb, 1995), pp. 97-103

\bibitem{HamiltonJPE}
Hamilton, Barton H. “Does entrepreneurship pay? An empirical analysis of the returns to self-employment.”  Journal of Political Economy 108 (June 2000): 604-631.

\bibitem{HendersonMcNamara}
Jason R. Henderson and Kevin T. McNamara "The Location of Food Manufacturing Plant Investments in Corn Belt Counties" Journal of Agricultural and Resource Economics
Vol. 25, No. 2 (December 2000), pp. 680-697 

\bibitem{Thomason}
Thomason, Terry and John F. Burton, Jr. 2001. “The Effectsof Changes in the Oregon Workers’ Compensation Pro-
gram on Employees’ Benefits and Employers’ Costs,” Workers’ Compensation Policy Review 1, No. 4 (July/August): 7-23, reprinted in Burton, Blum, and Yates (2005): 387-45. 

\bibitem{DonaldLang}
Donald, Stephen G; Lang, Kevin, "Inference with Difference-in-Differences and Other Panel Data," The Review of Economics and Statistics, vol 89 Issue 2. March 2007: pp 221-233

\bibitem{HannToddVann}
Hahn, Jinyong; Todd, Petra; Van der Klaauw, Wilbert, "Identification and Estimation of Treatment Effects with a Regression Discontinuity Design," Econometrica, Vol. 69 No. 1 (Jan 2001), pp 201-209

\bibitem{DharRoss}
Dhar, Paramita; Ross, Stephen L," School Dsicrict Quality and Property Values: Examining Differences Along School District Boundaries," Journal of Urban Economics, Vol 71 (2012), pp 18-25

\bibitem{KahnMansur}
Kahn, Matthew E; Mansur, Erin T, "Do Local Energy Prices and Regulation Affect the Geographic Concentration of Employment?" Journal of Public Economics, Vol 101 (2013), pp 105-114

\end{thebibliography}


\section{Appendix: Figures \& Tables}

\begin{figure}[h]\label{rb}
    \centering
    \textbf{Example of Border Matching}
    \includegraphics[scale = 0.5]{../analysis/output/borders_temp}
    \caption{Red rectangles are subject counties, and blue are neighbor counties. In this example Subject 1 would be only matched to Neighbor 1, while "Subject 2" would be paired with Neighbor 1-3. Similarly, when we broaden the bandwidth, Subject 1 would be matched with Nbr's Nbr 1, whle Subject 2 would be paired with Nbr's Nbr 1 and 2}
\end{figure}

\begin{figure}[h]\label{eb}
    \centering
    \textbf{Original Bandwidth Borders}\par\medskip
    \includegraphics[scale = 0.20]{../analysis/output/rb_picture}
    \caption{Red borders are subject counties, blue borders are neighbor counties}
\end{figure}

\begin{figure}[h]
    \centering
    \textbf{Extended Bandwidth Borders}\par\medskip
    \includegraphics[scale = 0.5]{../analysis/output/eb_picture}
    \caption{Red borders are subject counties, blue borders are neighbor counties}
\end{figure}


% Table created by stargazer v.5.2 by Marek Hlavac, Harvard University. E-mail: hlavac at fas.harvard.edu
% Date and time: Sat, Sep 03, 2016 - 06:48:59 AM
\begin{table}[!htbp] \centering 
  \caption{Summary Table for  Total Firm Births} 
  \label{--summary} 
\begin{tabular}{@{\extracolsep{5pt}}lccccc} 
\\[-1.8ex]\hline 
\hline \\[-1.8ex] 
Statistic & \multicolumn{1}{c}{N} & \multicolumn{1}{c}{Mean} & \multicolumn{1}{c}{St. Dev.} & \multicolumn{1}{c}{Min} & \multicolumn{1}{c}{Max} \\ 
\hline \\[-1.8ex] 
Births Ratio & 13,042 & $-$0.065 & 1.550 & $-$5.670 & 5.328 \\ 
Property Tax Difference & 13,042 & $-$0.097 & 0.500 & $-$1.672 & 1.241 \\ 
Income Tax Difference & 13,042 & 1.224 & 3.984 & $-$9.280 & 9.860 \\ 
Capital Gains Tax Difference & 13,042 & 1.917 & 4.320 & $-$9.280 & 13.420 \\ 
Sales Tax Difference & 13,042 & $-$0.311 & 2.124 & $-$7.000 & 7.250 \\ 
Corp Tax Difference & 13,042 & 1.276 & 3.688 & $-$8.900 & 12.000 \\ 
Workers Comp Tax Difference & 13,042 & 0.025 & 0.666 & $-$2.762 & 2.451 \\ 
Unemp. Tax Difference & 13,042 & 0.027 & 1.325 & $-$4.564 & 16.070 \\ 
Educ Spending Per Cap Diff & 13,042 & 8.951 & 209.455 & $-$807 & 692 \\ 
Highway Spending Per Cap Diff & 13,042 & $-$39.489 & 144.369 & $-$756 & 358 \\ 
Welfare Spending Per Cap Diff & 13,042 & $-$39.459 & 266.905 & $-$1,072 & 953 \\ 
Pct Highschool & 13,042 & 0.262 & 3.757 & $-$10.100 & 12.000 \\ 
Real Fuel Price & 13,042 & 0.320 & 2.348 & $-$7.500 & 8.200 \\ 
Pct Union & 13,042 & 0.633 & 4.670 & $-$14.900 & 16.100 \\ 
Pop Density & 13,042 & 42.623 & 161.167 & $-$746.200 & 901.000 \\ 
Pct Manuf & 13,042 & 0.011 & 0.067 & $-$0.240 & 0.250 \\ 
Jan Temp Z Diff & 13,042 & 0.002 & 0.207 & $-$1.291 & 1.291 \\ 
Jan Sun Z Diff & 13,042 & 0.042 & 0.582 & $-$2.499 & 3.583 \\ 
Jul Temp Z Diff & 13,042 & 0.066 & 0.601 & $-$4.475 & 4.115 \\ 
Jul Hum Z Diff & 13,042 & $-$0.029 & 0.424 & $-$3.697 & 3.081 \\ 
Topog Z Diff & 13,042 & $-$0.024 & 0.647 & $-$2.578 & 2.123 \\ 
Ln Water Z Diff & 13,042 & $-$0.058 & 0.871 & $-$3.456 & 3.155 \\ 
\hline \\[-1.8ex] 
\multicolumn{6}{l}{All variables are for the difference between our subject and neighbor counties. At the state level, this is 1177 observations. Further, all tax variables are scaled to be between 0 and 100 rather than 0 and 1. For each variable we observe them 13,115 times when not accounting for positive or negative infinify values in firm start up rates.} \\ 
\end{tabular} 
\end{table} 


\begin{figure}[h]\label{pairs}
    \centering
    \includegraphics[scale = 0.5]{../analysis/output/_--_pairs}
\end{figure}

\newgeometry{margin=1cm}
\begin{landscape}

% Table created by stargazer v.5.2 by Marek Hlavac, Harvard University. E-mail: hlavac at fas.harvard.edu
% Date and time: Sat, Sep 03, 2016 - 06:49:03 AM
\begin{table}[!htbp] \centering 
  \caption{Regression Discontinuity Models for  Total Firm Births} 
  \label{--rd} 
\footnotesize 
\begin{tabular}{@{\extracolsep{5pt}}lcccccc} 
\\[-1.8ex]\hline 
\hline \\[-1.8ex] 
 & \multicolumn{6}{c}{\textit{Dependent variable:}} \\ 
\cline{2-7} 
\\[-1.8ex] & \multicolumn{6}{c}{births ratio} \\ 
 & OLS & OLS & OLS & OLS & FE & FE \\ 
\\[-1.8ex] & (1) & (2) & (3) & (4) & (5) & (6)\\ 
\hline \\[-1.8ex] 
 Property Tax Difference & $-$0.199 & $-$0.363$^{**}$ & $-$0.129 & $-$0.291$^{*}$ & $-$0.008 & $-$0.020 \\ 
  & (0.151) & (0.147) & (0.148) & (0.150) & (0.120) & (0.123) \\ 
  Income Tax Difference & $-$0.094$^{***}$ & $-$0.086$^{***}$ & $-$0.088$^{***}$ & $-$0.076$^{***}$ & $-$0.024 & $-$0.023 \\ 
  & (0.027) & (0.025) & (0.028) & (0.026) & (0.035) & (0.035) \\ 
  Capital Gains Tax Difference & 0.017 & 0.009 & 0.028 & 0.019 & 0.0001 & 0.001 \\ 
  & (0.023) & (0.023) & (0.024) & (0.024) & (0.012) & (0.013) \\ 
  Sales Tax Difference & $-$0.117$^{***}$ & $-$0.106$^{***}$ & $-$0.112$^{***}$ & $-$0.089$^{***}$ & 0.005 & 0.004 \\ 
  & (0.029) & (0.029) & (0.028) & (0.031) & (0.040) & (0.041) \\ 
  Corp Tax Difference & 0.023 & 0.018 & 0.015 & 0.011 & $-$0.008 & $-$0.008 \\ 
  & (0.020) & (0.018) & (0.020) & (0.019) & (0.026) & (0.026) \\ 
  Workers Comp Tax Difference & 0.009 & 0.095 & $-$0.001 & 0.055 & 0.022 & 0.021 \\ 
  & (0.111) & (0.107) & (0.095) & (0.104) & (0.069) & (0.071) \\ 
  Unemp. Tax Difference & 0.009 & 0.013 & $-$0.002 & $-$0.005 & $-$0.001 & $-$0.002 \\ 
  & (0.041) & (0.037) & (0.044) & (0.039) & (0.018) & (0.019) \\ 
  Educ Spending Per Cap Diff & $-$0.0002 & $-$0.0002 & $-$0.0001 & $-$0.0002 & $-$0.0002 & $-$0.0001 \\ 
  & (0.0002) & (0.0003) & (0.0003) & (0.0003) & (0.0002) & (0.0002) \\ 
  Highway Spending Per Cap Diff & 0.0004 & 0.0004 & 0.0002 & 0.0003 & 0.0001 & 0.0001 \\ 
  & (0.0004) & (0.0004) & (0.0004) & (0.0004) & (0.0002) & (0.0002) \\ 
  Welfare Spending Per Cap Diff & 0.001$^{**}$ & 0.001$^{**}$ & 0.001$^{**}$ & 0.0004$^{*}$ & $-$0.0001 & $-$0.0001 \\ 
  & (0.0002) & (0.0003) & (0.0003) & (0.0003) & (0.0001) & (0.0001) \\ 
  Constant & $-$0.053 & $-$0.064 & $-$0.041 & $-$0.051 &  &  \\ 
  & (0.084) & (0.085) & (0.087) & (0.086) &  &  \\ 
 \hline \\[-1.8ex] 
controls & Yes & Yes & No & No & Yes & Yes \\ 
amenities & Yes & No & Yes & No & Yes & No \\ 
\hline \\[-1.8ex] 
Observations & 13,042 & 13,042 & 13,042 & 13,042 & 13,042 & 13,042 \\ 
R$^{2}$ & 0.094 & 0.055 & 0.080 & 0.037 & 0.243 & 0.205 \\ 
\hline 
\hline \\[-1.8ex] 
\textit{Note:}  & \multicolumn{6}{r}{$^{*}$p$<$0.1; $^{**}$p$<$0.05; $^{***}$p$<$0.01} \\ 
 & \multicolumn{6}{r}{The first four columns are estimated with OLS and clustered standard} \\ 
 & \multicolumn{6}{r}{ errors at the state-pair level. Columns 5 and 6 are estimated with} \\ 
 & \multicolumn{6}{r}{a fixed effect estimator at the state-pair level with homoskedastic} \\ 
 & \multicolumn{6}{r}{standard errors.} \\ 
\end{tabular} 
\end{table} 

\end{landscape}
\restoregeometry

\newgeometry{margin=1cm}
\begin{landscape}

% Table created by stargazer v.5.2 by Marek Hlavac, Harvard University. E-mail: hlavac at fas.harvard.edu
% Date and time: Fri, Jan 15, 2016 - 10:30:13 AM
\begin{table}[!htbp] \centering 
  \caption{Extended Bandwidth Discontinuity Models for  Total Firm Births} 
  \label{--rd} 
\begin{tabular}{@{\extracolsep{5pt}}lcccccc} 
\\[-1.8ex]\hline 
\hline \\[-1.8ex] 
 & \multicolumn{6}{c}{\textit{Dependent variable:}} \\ 
\cline{2-7} 
\\[-1.8ex] & \multicolumn{6}{c}{births ratio} \\ 
 & OLS & OLS & OLS & OLS & FE & FE \\ 
\\[-1.8ex] & (1) & (2) & (3) & (4) & (5) & (6)\\ 
\hline \\[-1.8ex] 
 Property Tax Difference & 0.039 & $-$0.019 & 0.104 & 0.074 & 0.007 & 0.006 \\ 
  & (0.147) & (0.152) & (0.143) & (0.148) & (0.112) & (0.114) \\ 
  Income Tax Difference & $-$0.054 & $-$0.063$^{*}$ & $-$0.043 & $-$0.050 & 0.008 & 0.012 \\ 
  & (0.035) & (0.036) & (0.038) & (0.037) & (0.033) & (0.034) \\ 
  Capital Gains Tax Difference & 0.039 & 0.048$^{*}$ & 0.043 & 0.053$^{*}$ & $-$0.013 & $-$0.013 \\ 
  & (0.029) & (0.028) & (0.033) & (0.030) & (0.012) & (0.012) \\ 
  Sales Tax Difference & $-$0.040 & $-$0.042 & $-$0.051 & $-$0.041 & 0.018 & 0.020 \\ 
  & (0.049) & (0.054) & (0.052) & (0.055) & (0.037) & (0.038) \\ 
  Corp Tax Difference & 0.006 & $-$0.001 & 0.004 & 0.002 & $-$0.024 & $-$0.024 \\ 
  & (0.026) & (0.025) & (0.027) & (0.025) & (0.024) & (0.024) \\ 
  Workers Comp Tax Difference & 0.180 & 0.300$^{**}$ & 0.139 & 0.216 & $-$0.008 & $-$0.007 \\ 
  & (0.126) & (0.152) & (0.142) & (0.178) & (0.066) & (0.068) \\ 
  Unemp. Tax Difference & $-$0.113$^{*}$ & $-$0.110$^{*}$ & $-$0.111 & $-$0.109 & 0.011 & 0.011 \\ 
  & (0.062) & (0.064) & (0.068) & (0.071) & (0.018) & (0.019) \\ 
  Educ Spending Per Cap Diff & 0.0001 & 0.0002 & 0.0002 & 0.0003 & $-$0.0001 & $-$0.0001 \\ 
  & (0.0005) & (0.001) & (0.0005) & (0.001) & (0.0002) & (0.0002) \\ 
  Highway Spending Per Cap Diff & 0.0002 & 0.0001 & $-$0.0002 & $-$0.0003 & 0.0001 & 0.00005 \\ 
  & (0.0005) & (0.001) & (0.0005) & (0.001) & (0.0002) & (0.0002) \\ 
  Welfare Spending Per Cap Diff & 0.001 & 0.001$^{*}$ & 0.001$^{*}$ & 0.001$^{*}$ & $-$0.00003 & $-$0.00004 \\ 
  & (0.0004) & (0.0004) & (0.0004) & (0.0004) & (0.0001) & (0.0001) \\ 
  Constant & $-$0.033 & $-$0.017 & $-$0.026 & 0.002 &  &  \\ 
  & (0.100) & (0.111) & (0.105) & (0.113) &  &  \\ 
 \hline \\[-1.8ex] 
controls & Yes & Yes & No & No & Yes & Yes \\ 
amenities & Yes & No & Yes & No & Yes & No \\ 
\hline \\[-1.8ex] 
Observations & 16,245 & 16,245 & 16,245 & 16,245 & 16,245 & 16,245 \\ 
R$^{2}$ & 0.097 & 0.038 & 0.081 & 0.023 & 0.298 & 0.267 \\ 
\hline 
\hline \\[-1.8ex] 
\textit{Note:}  & \multicolumn{6}{r}{$^{*}$p$<$0.1; $^{**}$p$<$0.05; $^{***}$p$<$0.01} \\ 
\end{tabular} 
\end{table} 

\end{landscape}
\restoregeometry


% Table created by stargazer v.5.2 by Marek Hlavac, Harvard University. E-mail: hlavac at fas.harvard.edu
% Date and time: Tue, Sep 08, 2015 - 12:16:41 PM
\begin{table}[!htbp] \centering 
  \caption{Pseudo-RD for Stability of Terms over Borders for  --} 
  \label{} 
\begin{tabular}{@{\extracolsep{5pt}}lcc} 
\\[-1.8ex]\hline 
\hline \\[-1.8ex] 
 & \multicolumn{2}{c}{\textit{Dependent variable:}} \\ 
\cline{2-3} 
\\[-1.8ex] & \multicolumn{2}{c}{ } \\ 
\\[-1.8ex] & (1) & (2)\\ 
\hline \\[-1.8ex] 
 ptax\_sub & $-$0.056 & $-$0.369$^{**}$ \\ 
  & (0.183) & (0.174) \\ 
  ptax\_nbr & 0.208 & 0.350$^{**}$ \\ 
  & (0.166) & (0.147) \\ 
  inctax\_sub & $-$0.145$^{***}$ & $-$0.121$^{***}$ \\ 
  & (0.053) & (0.045) \\ 
  inctax\_nbr & 0.076$^{**}$ & 0.055$^{*}$ \\ 
  & (0.038) & (0.032) \\ 
  capgntax\_sub & 0.034 & 0.022 \\ 
  & (0.035) & (0.032) \\ 
  capgntax\_nbr & $-$0.067$^{**}$ & $-$0.045 \\ 
  & (0.033) & (0.031) \\ 
  salestax\_sub & $-$0.153$^{***}$ & $-$0.145$^{***}$ \\ 
  & (0.044) & (0.039) \\ 
  salestax\_nbr & 0.037 & 0.006 \\ 
  & (0.046) & (0.045) \\ 
  corptax\_sub & 0.026 & 0.029 \\ 
  & (0.026) & (0.026) \\ 
  corptax\_nbr & 0.010 & 0.002 \\ 
  & (0.024) & (0.024) \\ 
  wctaxfixed\_sub & $-$0.149 & $-$0.121 \\ 
  & (0.133) & (0.117) \\ 
  wctaxfixed\_nbr & $-$0.125 & $-$0.229 \\ 
  & (0.147) & (0.147) \\ 
  uitaxrate\_sub & $-$1.762 & $-$5.908 \\ 
  & (4.359) & (4.492) \\ 
  uitaxrate\_nbr & $-$1.272 & $-$2.129 \\ 
  & (7.629) & (5.669) \\ 
  educ\_pc\_L1\_sub & $-$0.0002 & $-$0.001 \\ 
  & (0.0004) & (0.0004) \\ 
  educ\_pc\_L1\_nbr & 0.0001 & 0.0001 \\ 
  & (0.0004) & (0.0004) \\ 
  hwy\_pc\_L1\_sub & 0.0004 & 0.001 \\ 
  & (0.001) & (0.001) \\ 
  hwy\_pc\_L1\_nbr & $-$0.0005 & $-$0.0004 \\ 
  & (0.001) & (0.001) \\ 
  welfare\_pc\_L1\_sub & 0.001$^{**}$ & 0.001$^{**}$ \\ 
  & (0.0003) & (0.0003) \\ 
  welfare\_pc\_L1\_nbr & $-$0.0005 & $-$0.0003 \\ 
  & (0.0003) & (0.0003) \\ 
  JAN.TEMP...Z\_sub & 0.903$^{***}$ &  \\ 
  & (0.349) &  \\ 
  JAN.TEMP...Z\_nbr & $-$0.866$^{***}$ &  \\ 
  & (0.336) &  \\ 
  JAN.SUN...Z\_sub & 0.137 &  \\ 
  & (0.138) &  \\ 
  JAN.SUN...Z\_nbr & $-$0.065 &  \\ 
  & (0.141) &  \\ 
  JUL.TEMP...Z\_sub & 0.135 &  \\ 
  & (0.121) &  \\ 
  JUL.TEMP...Z\_nbr & 0.038 &  \\ 
  & (0.127) &  \\ 
  JUL.HUM...Z\_sub & $-$0.088 &  \\ 
  & (0.173) &  \\ 
  JUL.HUM...Z\_nbr & 0.073 &  \\ 
  & (0.198) &  \\ 
  TOPOG...Z\_sub & 0.097 &  \\ 
  & (0.085) &  \\ 
  TOPOG...Z\_nbr & $-$0.128$^{*}$ &  \\ 
  & (0.074) &  \\ 
  LN.WATER..AREA...Z\_sub & 0.222$^{**}$ &  \\ 
  & (0.087) &  \\ 
  LN.WATER..AREA...Z\_nbr & $-$0.299$^{***}$ &  \\ 
  & (0.080) &  \\ 
  Constant & 1.115 & 1.701$^{**}$ \\ 
  & (0.852) & (0.748) \\ 
 \hline \\[-1.8ex] 
\hline 
\hline \\[-1.8ex] 
\textit{Note:}  & \multicolumn{2}{r}{$^{*}$p$<$0.1; $^{**}$p$<$0.05; $^{***}$p$<$0.01} \\ 
\end{tabular} 
\end{table} 


% Table created by stargazer v.5.2 by Marek Hlavac, Harvard University. E-mail: hlavac at fas.harvard.edu
% Date and time: Mon, Feb 15, 2016 - 09:58:18 AM
\begin{table}[!htbp] \centering 
  \caption{F-Tests for Symmetry of Coefficients for Total Firm Start Ups} 
  \label{--Ftests} 
\begin{tabular}{@{\extracolsep{5pt}} ccc} 
\\[-1.8ex]\hline 
\hline \\[-1.8ex] 
Test & F-Stat & P(\textgreater F) \\ 
\hline \\[-1.8ex] 
ptax\_sub = -ptax\_nbr & 0.0064 & 0.9361 \\ 
inctax\_sub = -inctax\_nbr & 1.7426 & 0.1868 \\ 
capgntax\_sub = -capgntax\_nbr & 0.3873 & 0.5337 \\ 
salestax\_sub = -salestax\_nbr & 4.5658 & 0.0326 \\ 
corptax\_sub = -corptax\_nbr & 0.6824 & 0.4088 \\ 
wctaxfixed\_sub = -wctaxfixed\_nbr & 3.2369 & 0.072 \\ 
uitaxrate\_sub = -uitaxrate\_nbr & 1.8872 & 0.1695 \\ 
\hline \\[-1.8ex] 
\end{tabular} 
\end{table} 


\newgeometry{margin=1cm}
\begin{landscape}

% Table created by stargazer v.5.2 by Marek Hlavac, Harvard University. E-mail: hlavac at fas.harvard.edu
% Date and time: Tue, Mar 29, 2016 - 09:26:56 PM
\begin{table}[!htbp] \centering 
  \caption{Psuedo-RD for Stability over Time for  Total Firm Births} 
  \label{--year} 
\small 
\begin{tabular}{@{\extracolsep{5pt}}lccccccccccc} 
\\[-1.8ex]\hline 
\hline \\[-1.8ex] 
 & \multicolumn{11}{c}{\textit{Dependent variable:}} \\ 
\cline{2-12} 
\\[-1.8ex] & \multicolumn{11}{c}{births ratio} \\ 
 & 1999 & 2000 & 2001 & 2002 & 2003 & 2004 & 2005 & 2006 & 2007 & 2008 & 2009 \\ 
\\[-1.8ex] & (1) & (2) & (3) & (4) & (5) & (6) & (7) & (8) & (9) & (10) & (11)\\ 
\hline \\[-1.8ex] 
 Prop Tax Diff & $-$0.411 & $-$0.371 & $-$0.426 & $-$0.390 & $-$0.320 & $-$0.479 & $-$0.344 & $-$0.364 & $-$0.396 & $-$0.311 & $-$0.351$^{***}$ \\ 
  & ($-$0.411) & ($-$0.426) & ($-$0.390) & ($-$0.320) & ($-$0.479) & ($-$0.344) & ($-$0.364) & ($-$0.396) & ($-$0.311) & ($-$0.351) & (0.116) \\ 
  Inc Tax Diff & $-$0.025 & $-$0.026 & $-$0.066 & $-$0.061 & $-$0.047 & $-$0.055 & $-$0.063 & $-$0.136 & $-$0.127 & $-$0.123 & $-$0.117$^{***}$ \\ 
  & ($-$0.025) & ($-$0.066) & ($-$0.061) & ($-$0.047) & ($-$0.055) & ($-$0.063) & ($-$0.136) & ($-$0.127) & ($-$0.123) & ($-$0.117) & (0.026) \\ 
  Cap Tax Diff & $-$0.045 & $-$0.040 & $-$0.025$^{***}$ & $-$0.006 & $-$0.018 & $-$0.032 & $-$0.029 & 0.054 & 0.036 & 0.032 & 0.028 \\ 
  & ($-$0.045) & ($-$0.025) & ($-$0.006) & ($-$0.018) & ($-$0.032) & ($-$0.029) & (0.054) & (0.036) & (0.032) & (0.028) & (0.023) \\ 
  Sal Tax Diff & $-$0.083 & $-$0.097 & $-$0.104 & $-$0.106 & $-$0.095 & $-$0.119 & $-$0.136 & $-$0.102 & $-$0.110 & $-$0.140 & $-$0.132$^{***}$ \\ 
  & ($-$0.083) & ($-$0.104) & ($-$0.106) & ($-$0.095) & ($-$0.119) & ($-$0.136) & ($-$0.102) & ($-$0.110) & ($-$0.140) & ($-$0.132) & (0.026) \\ 
  Corp Tax Diff & $-$0.015 & 0.011 & 0.010 & 0.007 & 0.035 & 0.030 & 0.037$^{*}$ & 0.019$^{***}$ & 0.004 & 0.014$^{**}$ & $-$0.007 \\ 
  & ($-$0.015) & (0.010) & (0.007) & (0.035) & (0.030) & (0.037) & (0.019) & (0.004) & (0.014) & ($-$0.007) & (0.018) \\ 
  Work Comp Diff & 0.309 & 0.225 & 0.201$^{***}$ & 0.018 & 0.029 & 0.071 & 0.066 & 0.142 & 0.102 & 0.086 & 0.089 \\ 
  & (0.309) & (0.201) & (0.018) & (0.029) & (0.071) & (0.066) & (0.142) & (0.102) & (0.086) & (0.089) & (0.092) \\ 
  Unemp. Tax Diff & $-$0.045 & 0.0001 & 0.015 & 0.027 & $-$0.022 & 0.062$^{***}$ & 0.003 & $-$0.014 & $-$0.034$^{*}$ & 0.020 & 0.070$^{*}$ \\ 
  & ($-$0.045) & (0.015) & (0.027) & ($-$0.022) & (0.062) & (0.003) & ($-$0.014) & ($-$0.034) & (0.020) & (0.070) & (0.039) \\ 
  Ln Educ Diff & $-$0.0001 & $-$0.0002 & $-$0.0003 & $-$0.0002 & $-$0.0002 & $-$0.001$^{**}$ & $-$0.0003$^{***}$ & 0.0001 & $-$0.0002 & $-$0.0001 & $-$0.0002 \\ 
  & ($-$0.0001) & ($-$0.0003) & ($-$0.0002) & ($-$0.0002) & ($-$0.001) & ($-$0.0003) & (0.0001) & ($-$0.0002) & ($-$0.0001) & ($-$0.0002) & (0.0002) \\ 
  Ln Hwy Diff & 0.001 & 0.002 & 0.001$^{***}$ & 0.0002 & 0.0004 & 0.0004$^{***}$ & 0.0001 & 0.0002$^{*}$ & 0.0001 & $-$0.0002 & $-$0.0004 \\ 
  & (0.001) & (0.001) & (0.0002) & (0.0004) & (0.0004) & (0.0001) & (0.0002) & (0.0001) & ($-$0.0002) & ($-$0.0004) & (0.0003) \\ 
  Ln Welf. Diff & 0.001 & 0.001 & 0.001 & 0.001 & 0.001 & 0.0005 & 0.001 & 0.001 & 0.001 & 0.001 & 0.001$^{***}$ \\ 
  & (0.001) & (0.001) & (0.001) & (0.001) & (0.0005) & (0.001) & (0.001) & (0.001) & (0.001) & (0.001) & (0.0002) \\ 
  Constant & $-$0.034 & $-$0.026$^{*}$ & $-$0.013 & $-$0.057$^{***}$ & 0.007 & $-$0.042$^{***}$ & $-$0.015 & $-$0.097 & $-$0.072 & $-$0.086 & $-$0.075 \\ 
  & ($-$0.034) & ($-$0.013) & ($-$0.057) & (0.007) & ($-$0.042) & ($-$0.015) & ($-$0.097) & ($-$0.072) & ($-$0.086) & ($-$0.075) & (0.056) \\ 
 \hline \\[-1.8ex] 
controls & Yes & Yes & Yes & Yes & Yes & Yes & Yes & Yes & Yes & Yes & Yes \\ 
amenities & No & No & No & No & No & No & No & No & No & No & No \\ 
\hline \\[-1.8ex] 
Observations & 1,193 & 1,188 & 1,191 & 1,195 & 1,189 & 1,188 & 1,191 & 1,194 & 1,199 & 1,196 & 1,191 \\ 
R$^{2}$ & 0.068 & 0.059 & 0.066 & 0.050 & 0.052 & 0.068 & 0.064 & 0.062 & 0.069 & 0.067 & 0.077 \\ 
\hline 
\hline \\[-1.8ex] 
\textit{Note:}  & \multicolumn{11}{r}{$^{*}$p$<$0.1; $^{**}$p$<$0.05; $^{***}$p$<$0.01} \\ 
 & \multicolumn{11}{r}{All models are estimated with Ordinary Least Squares and clustered standard errors at the state-pair level.} \\ 
\end{tabular} 
\end{table} 

\end{landscape}
\restoregeometry

\newgeometry{margin=1cm}
\begin{landscape}

% Table created by stargazer v.5.2 by Marek Hlavac, Harvard University. E-mail: hlavac at fas.harvard.edu
% Date and time: Tue, Apr 19, 2016 - 09:28:33 AM
\begin{table}[!htbp] \centering 
  \caption{Results for Firm Entry across NAICS Subcodes for } 
  \label{naics} 
\small 
\begin{tabular}{@{\extracolsep{5pt}}lcccccccc} 
\\[-1.8ex]\hline 
\hline \\[-1.8ex] 
 & \multicolumn{8}{c}{\textit{Dependent variable:}} \\ 
\cline{2-9} 
\\[-1.8ex] & \multicolumn{8}{c}{births ratio} \\ 
 & Farming & Farming & Manuf & Manuf & Retail & Retail & Finance & Finance \\ 
\\[-1.8ex] & (1) & (2) & (3) & (4) & (5) & (6) & (7) & (8)\\ 
\hline \\[-1.8ex] 
 Property Tax Difference & $-$0.367$^{**}$ & $-$0.300$^{**}$ & $-$0.365$^{**}$ & $-$0.294$^{**}$ & $-$0.354$^{**}$ & $-$0.282$^{*}$ & $-$0.375$^{**}$ & $-$0.302$^{**}$ \\ 
  & (0.144) & (0.147) & (0.145) & (0.149) & (0.148) & (0.152) & (0.146) & (0.149) \\ 
  Income Tax Difference & $-$0.083$^{***}$ & $-$0.073$^{***}$ & $-$0.081$^{***}$ & $-$0.071$^{***}$ & $-$0.082$^{***}$ & $-$0.073$^{***}$ & $-$0.085$^{***}$ & $-$0.075$^{***}$ \\ 
  & (0.025) & (0.026) & (0.026) & (0.026) & (0.026) & (0.026) & (0.026) & (0.026) \\ 
  Capital Gains Tax Difference & 0.008 & 0.019 & 0.006 & 0.017 & 0.005 & 0.017 & 0.009 & 0.020 \\ 
  & (0.024) & (0.024) & (0.024) & (0.024) & (0.024) & (0.024) & (0.024) & (0.024) \\ 
  Sales Tax Difference & $-$0.102$^{***}$ & $-$0.087$^{***}$ & $-$0.100$^{***}$ & $-$0.085$^{***}$ & $-$0.107$^{***}$ & $-$0.091$^{***}$ & $-$0.105$^{***}$ & $-$0.090$^{***}$ \\ 
  & (0.029) & (0.030) & (0.030) & (0.031) & (0.030) & (0.032) & (0.030) & (0.032) \\ 
  Corp Tax Difference & 0.020 & 0.012 & 0.017 & 0.011 & 0.019 & 0.011 & 0.017 & 0.010 \\ 
  & (0.018) & (0.018) & (0.018) & (0.018) & (0.018) & (0.019) & (0.018) & (0.019) \\ 
  Workers Comp Tax Difference & 0.086 & 0.047 & 0.094 & 0.053 & 0.086 & 0.048 & 0.088 & 0.046 \\ 
  & (0.106) & (0.103) & (0.108) & (0.104) & (0.110) & (0.106) & (0.107) & (0.104) \\ 
  Unemp. Tax Difference & 0.011 & $-$0.006 & 0.011 & $-$0.006 & 0.013 & $-$0.007 & 0.013 & $-$0.005 \\ 
  & (0.035) & (0.038) & (0.036) & (0.038) & (0.037) & (0.039) & (0.036) & (0.038) \\ 
  Educ Spending Per Cap Diff & $-$0.0003 & $-$0.0002 & $-$0.0002 & $-$0.0002 & $-$0.0002 & $-$0.0002 & $-$0.0002 & $-$0.0002 \\ 
  & (0.0003) & (0.0003) & (0.0003) & (0.0003) & (0.0003) & (0.0003) & (0.0003) & (0.0003) \\ 
  Highway Spending Per Cap Diff & 0.0004 & 0.0003 & 0.0003 & 0.0002 & 0.0003 & 0.0003 & 0.0003 & 0.0002 \\ 
  & (0.0004) & (0.0004) & (0.0004) & (0.0004) & (0.0004) & (0.0004) & (0.0004) & (0.0004) \\ 
  Welfare Spending Per Cap Diff & 0.001$^{**}$ & 0.0004 & 0.001$^{**}$ & 0.0005$^{*}$ & 0.001$^{**}$ & 0.0005$^{*}$ & 0.001$^{**}$ & 0.0004$^{*}$ \\ 
  & (0.0003) & (0.0003) & (0.0003) & (0.0003) & (0.0003) & (0.0003) & (0.0003) & (0.0003) \\ 
  Constant & $-$0.062 & $-$0.053 & $-$0.058 & $-$0.049 & $-$0.057 & $-$0.049 & $-$0.064 & $-$0.054 \\ 
  & (0.084) & (0.085) & (0.084) & (0.085) & (0.086) & (0.087) & (0.085) & (0.086) \\ 
 \hline \\[-1.8ex] 
controls & Yes & No & Yes & No & Yes & No & Yes & No \\ 
amenities & No & No & No & No & No & No & No & No \\ 
\hline \\[-1.8ex] 
Observations & 12,550 & 12,550 & 12,998 & 12,998 & 13,119 & 13,119 & 12,984 & 12,984 \\ 
R$^{2}$ & 0.054 & 0.036 & 0.053 & 0.036 & 0.055 & 0.036 & 0.055 & 0.037 \\ 
\hline 
\hline \\[-1.8ex] 
\textit{Note:}  & \multicolumn{8}{r}{$^{*}$p$<$0.1; $^{**}$p$<$0.05; $^{***}$p$<$0.01} \\ 
 & \multicolumn{8}{r}{All models are estimated with Ordinary Least Squares} \\ 
 & \multicolumn{8}{r}{and clustered standard errors at the state-pair level.} \\ 
\end{tabular} 
\end{table} 

\end{landscape}
\restoregeometry

\begin{figure}[h]\label{weightedtax}
    \centering
    \includegraphics[scale = 0.5]{../analysis/output/_--_weightedtax}
\end{figure}


% Table created by stargazer v.5.2 by Marek Hlavac, Harvard University. E-mail: hlavac at fas.harvard.edu
% Date and time: Mon, Oct 19, 2015 - 09:27:35 PM
\begin{table}[!htbp] \centering 
  \caption{Result Comparison for Total Firm Births} 
  \label{} 
\tiny 
\begin{tabular}{@{\extracolsep{5pt}} ccccccc} 
\\[-1.8ex]\hline 
\hline \\[-1.8ex] 
mean firm entry & preffered side & abs weighted tax & preferred side & same? & sub state & nbr state \\ 
\hline \\[-1.8ex] 
$2.591$ & nbr & $0.010$ & sub & different & kansas & nebraska \\ 
$2.260$ & nbr & $0.016$ & nbr & same & maryland & west virginia \\ 
$2.194$ & sub & $0.294$ & sub & same & alabama & georgia \\ 
$2.126$ & sub & $0.205$ & nbr & different & minnesota & wisconsin \\ 
$1.808$ & sub & $0.097$ & nbr & different & ohio & pennsylvania \\ 
$1.743$ & sub & $0.555$ & sub & same & colorado & kansas \\ 
$1.568$ & nbr & $0.105$ & nbr & same & arizona & nevada \\ 
$1.513$ & nbr & $0.256$ & sub & different & idaho & utah \\ 
$1.477$ & sub & $0.119$ & sub & same & oklahoma & texas \\ 
$1.376$ & nbr & $0.015$ & nbr & same & kentucky & west virginia \\ 
\hline \\[-1.8ex] 
\end{tabular} 
\end{table} 


\end{document}
\documentclass{article}

\usepackage[ampersand]{easylist}
\usepackage{float}
\usepackage{amsmath}

\usepackage{geometry}
\usepackage{pdflscape}

\begin{document}

\begin{easylist}[itemize]
& MVEA Discussant
&& Results for employment
&&& Might have used total employment, rather than change in employment.
& SEA Discussant
&& Agrawal, "Local Fiscal Competition: An Application to Sales Taxation with Multiple Federations," JUE (forthcoming)
&& Estimates for MSA's
&& Estimates for Urban/Rural areas
&& Estimates for each Region
&& Estimates for Recipricol Agreements
&& some clarifications (taxes, assignment procedure
& Georgeanne Edits
&& COMPLETED: Lit Review, Theory, Data, Conclusions
&& TODO: empirical design, results, introduction
\end{easylist}

\newpage
\begin{landscape}

% Table created by stargazer v.5.2 by Marek Hlavac, Harvard University. E-mail: hlavac at fas.harvard.edu
% Date and time: Sat, Sep 03, 2016 - 06:49:03 AM
\begin{table}[!htbp] \centering 
  \caption{Regression Discontinuity Models for  Total Firm Births} 
  \label{--rd} 
\footnotesize 
\begin{tabular}{@{\extracolsep{5pt}}lcccccc} 
\\[-1.8ex]\hline 
\hline \\[-1.8ex] 
 & \multicolumn{6}{c}{\textit{Dependent variable:}} \\ 
\cline{2-7} 
\\[-1.8ex] & \multicolumn{6}{c}{births ratio} \\ 
 & OLS & OLS & OLS & OLS & FE & FE \\ 
\\[-1.8ex] & (1) & (2) & (3) & (4) & (5) & (6)\\ 
\hline \\[-1.8ex] 
 Property Tax Difference & $-$0.199 & $-$0.363$^{**}$ & $-$0.129 & $-$0.291$^{*}$ & $-$0.008 & $-$0.020 \\ 
  & (0.151) & (0.147) & (0.148) & (0.150) & (0.120) & (0.123) \\ 
  Income Tax Difference & $-$0.094$^{***}$ & $-$0.086$^{***}$ & $-$0.088$^{***}$ & $-$0.076$^{***}$ & $-$0.024 & $-$0.023 \\ 
  & (0.027) & (0.025) & (0.028) & (0.026) & (0.035) & (0.035) \\ 
  Capital Gains Tax Difference & 0.017 & 0.009 & 0.028 & 0.019 & 0.0001 & 0.001 \\ 
  & (0.023) & (0.023) & (0.024) & (0.024) & (0.012) & (0.013) \\ 
  Sales Tax Difference & $-$0.117$^{***}$ & $-$0.106$^{***}$ & $-$0.112$^{***}$ & $-$0.089$^{***}$ & 0.005 & 0.004 \\ 
  & (0.029) & (0.029) & (0.028) & (0.031) & (0.040) & (0.041) \\ 
  Corp Tax Difference & 0.023 & 0.018 & 0.015 & 0.011 & $-$0.008 & $-$0.008 \\ 
  & (0.020) & (0.018) & (0.020) & (0.019) & (0.026) & (0.026) \\ 
  Workers Comp Tax Difference & 0.009 & 0.095 & $-$0.001 & 0.055 & 0.022 & 0.021 \\ 
  & (0.111) & (0.107) & (0.095) & (0.104) & (0.069) & (0.071) \\ 
  Unemp. Tax Difference & 0.009 & 0.013 & $-$0.002 & $-$0.005 & $-$0.001 & $-$0.002 \\ 
  & (0.041) & (0.037) & (0.044) & (0.039) & (0.018) & (0.019) \\ 
  Educ Spending Per Cap Diff & $-$0.0002 & $-$0.0002 & $-$0.0001 & $-$0.0002 & $-$0.0002 & $-$0.0001 \\ 
  & (0.0002) & (0.0003) & (0.0003) & (0.0003) & (0.0002) & (0.0002) \\ 
  Highway Spending Per Cap Diff & 0.0004 & 0.0004 & 0.0002 & 0.0003 & 0.0001 & 0.0001 \\ 
  & (0.0004) & (0.0004) & (0.0004) & (0.0004) & (0.0002) & (0.0002) \\ 
  Welfare Spending Per Cap Diff & 0.001$^{**}$ & 0.001$^{**}$ & 0.001$^{**}$ & 0.0004$^{*}$ & $-$0.0001 & $-$0.0001 \\ 
  & (0.0002) & (0.0003) & (0.0003) & (0.0003) & (0.0001) & (0.0001) \\ 
  Constant & $-$0.053 & $-$0.064 & $-$0.041 & $-$0.051 &  &  \\ 
  & (0.084) & (0.085) & (0.087) & (0.086) &  &  \\ 
 \hline \\[-1.8ex] 
controls & Yes & Yes & No & No & Yes & Yes \\ 
amenities & Yes & No & Yes & No & Yes & No \\ 
\hline \\[-1.8ex] 
Observations & 13,042 & 13,042 & 13,042 & 13,042 & 13,042 & 13,042 \\ 
R$^{2}$ & 0.094 & 0.055 & 0.080 & 0.037 & 0.243 & 0.205 \\ 
\hline 
\hline \\[-1.8ex] 
\textit{Note:}  & \multicolumn{6}{r}{$^{*}$p$<$0.1; $^{**}$p$<$0.05; $^{***}$p$<$0.01} \\ 
 & \multicolumn{6}{r}{The first four columns are estimated with OLS and clustered standard} \\ 
 & \multicolumn{6}{r}{ errors at the state-pair level. Columns 5 and 6 are estimated with} \\ 
 & \multicolumn{6}{r}{a fixed effect estimator at the state-pair level with homoskedastic} \\ 
 & \multicolumn{6}{r}{standard errors.} \\ 
\end{tabular} 
\end{table} 

\end{landscape}

\newpage

\section{MVEA}

\input{../analysis/output/_--_emp_rd_results.tex}

\newpage
\section{SEA}


% Table created by stargazer v.5.2 by Marek Hlavac, Harvard University. E-mail: hlavac at fas.harvard.edu
% Date and time: Wed, Jan 13, 2016 - 01:45:06 PM
\begin{table}[!htbp] \centering 
  \caption{Regional Estates for  Total Firm Births} 
  \label{} 
\begin{tabular}{@{\extracolsep{5pt}}lcccc} 
\\[-1.8ex]\hline 
\hline \\[-1.8ex] 
 & \multicolumn{4}{c}{\textit{Dependent variable:}} \\ 
\cline{2-5} 
\\[-1.8ex] & \multicolumn{4}{c}{births ratio} \\ 
 & Northeast & Midwest & South & West \\ 
\\[-1.8ex] & (1) & (2) & (3) & (4)\\ 
\hline \\[-1.8ex] 
 Property Tax Difference & 0.069 & 0.008 & $-$0.200 & $-$1.324$^{***}$ \\ 
  & (0.202) & (0.186) & (0.280) & (0.397) \\ 
  Income Tax Difference & 0.103$^{**}$ & $-$0.016 & $-$0.125$^{***}$ & $-$0.222$^{**}$ \\ 
  & (0.051) & (0.037) & (0.039) & (0.104) \\ 
  Capital Gains Tax Difference & $-$0.149$^{***}$ & 0.010 & 0.076$^{*}$ & 0.035 \\ 
  & (0.050) & (0.029) & (0.040) & (0.039) \\ 
  Sales Tax Difference & $-$0.110$^{*}$ & $-$0.334$^{***}$ & $-$0.166$^{*}$ & $-$0.090$^{*}$ \\ 
  & (0.060) & (0.095) & (0.086) & (0.052) \\ 
  Corp Tax Difference & 0.256$^{***}$ & $-$0.010 & $-$0.038 & 0.225$^{***}$ \\ 
  & (0.063) & (0.024) & (0.041) & (0.068) \\ 
  Workers Comp Tax Difference & $-$0.183 & 0.219$^{*}$ & 0.276 & 0.008 \\ 
  & (0.202) & (0.114) & (0.202) & (0.240) \\ 
  Unemp. Tax Difference & $-$0.094 & 0.041 & $-$0.018 & $-$0.119 \\ 
  & (0.092) & (0.059) & (0.046) & (0.086) \\ 
  Educ Spending Per Cap Diff & $-$0.001$^{**}$ & 0.0003 & $-$0.001 & $-$0.001 \\ 
  & (0.0005) & (0.0004) & (0.0005) & (0.001) \\ 
  Highway Spending Per Cap Diff & 0.001 & 0.001 & $-$0.001 & 0.0004 \\ 
  & (0.001) & (0.001) & (0.001) & (0.001) \\ 
  Welfare Spending Per Cap Diff & 0.001$^{***}$ & 0.001$^{*}$ & 0.0004 & 0.00002 \\ 
  & (0.0002) & (0.0004) & (0.0004) & (0.0004) \\ 
  Constant & 0.502$^{***}$ & 0.040 & $-$0.119 & $-$0.680 \\ 
  & (0.104) & (0.091) & (0.147) & (0.437) \\ 
 \hline \\[-1.8ex] 
Observations & 1,060 & 4,374 & 5,205 & 2,476 \\ 
R$^{2}$ & 0.184 & 0.060 & 0.055 & 0.119 \\ 
\hline 
\hline \\[-1.8ex] 
\textit{Note:}  & \multicolumn{4}{r}{$^{*}$p$<$0.1; $^{**}$p$<$0.05; $^{***}$p$<$0.01} \\ 
\end{tabular} 
\end{table} 

\newpage

% Table created by stargazer v.5.2 by Marek Hlavac, Harvard University. E-mail: hlavac at fas.harvard.edu
% Date and time: Wed, Jan 13, 2016 - 01:43:17 PM
\begin{table}[!htbp] \centering 
  \caption{MSA Estates for  Total Firm Births} 
  \label{} 
\begin{tabular}{@{\extracolsep{5pt}}lcccc} 
\\[-1.8ex]\hline 
\hline \\[-1.8ex] 
 & \multicolumn{4}{c}{\textit{Dependent variable:}} \\ 
\cline{2-5} 
\\[-1.8ex] & \multicolumn{4}{c}{births ratio} \\ 
 & In a MSA & Same MSA & Jointly Urban & Jointly Rural \\ 
\\[-1.8ex] & (1) & (2) & (3) & (4)\\ 
\hline \\[-1.8ex] 
 Property Tax Difference & $-$0.339 & $-$0.153 & $-$0.205 & $-$0.390$^{**}$ \\ 
  & (0.418) & (0.614) & (0.215) & (0.174) \\ 
  Income Tax Difference & $-$0.183$^{***}$ & $-$0.309$^{***}$ & $-$0.124$^{***}$ & $-$0.041 \\ 
  & (0.068) & (0.097) & (0.042) & (0.039) \\ 
  Capital Gains Tax Difference & 0.117$^{*}$ & 0.228$^{***}$ & 0.074$^{*}$ & $-$0.019 \\ 
  & (0.063) & (0.077) & (0.039) & (0.026) \\ 
  Sales Tax Difference & $-$0.132 & $-$0.253$^{***}$ & $-$0.125$^{***}$ & $-$0.069 \\ 
  & (0.086) & (0.086) & (0.048) & (0.053) \\ 
  Corp Tax Difference & 0.020 & 0.031 & $-$0.037 & 0.058$^{**}$ \\ 
  & (0.048) & (0.073) & (0.028) & (0.026) \\ 
  Workers Comp Tax Difference & 0.425$^{**}$ & 0.438 & 0.149 & $-$0.109 \\ 
  & (0.182) & (0.293) & (0.131) & (0.163) \\ 
  Unemp. Tax Difference & 0.098$^{*}$ & 0.084 & 0.031 & $-$0.070 \\ 
  & (0.060) & (0.062) & (0.048) & (0.054) \\ 
  Educ Spending Per Cap Diff & $-$0.001 & $-$0.0004 & $-$0.0001 & $-$0.001$^{*}$ \\ 
  & (0.001) & (0.001) & (0.0004) & (0.0004) \\ 
  Highway Spending Per Cap Diff & $-$0.002$^{*}$ & $-$0.001 & $-$0.00002 & 0.001$^{**}$ \\ 
  & (0.001) & (0.001) & (0.001) & (0.001) \\ 
  Welfare Spending Per Cap Diff & 0.0001 & $-$0.0001 & 0.0002 & 0.001$^{*}$ \\ 
  & (0.001) & (0.001) & (0.0003) & (0.0004) \\ 
  Constant & $-$0.248 & $-$0.507$^{*}$ & $-$0.329$^{***}$ & 0.381$^{***}$ \\ 
  & (0.214) & (0.261) & (0.113) & (0.101) \\ 
 \hline \\[-1.8ex] 
Observations & 2,223 & 1,383 & 8,180 & 4,935 \\ 
R$^{2}$ & 0.117 & 0.168 & 0.050 & 0.089 \\ 
\hline 
\hline \\[-1.8ex] 
\textit{Note:}  & \multicolumn{4}{r}{$^{*}$p$<$0.1; $^{**}$p$<$0.05; $^{***}$p$<$0.01} \\ 
\end{tabular} 
\end{table} 

\newpage

% Table created by stargazer v.5.2 by Marek Hlavac, Harvard University. E-mail: hlavac at fas.harvard.edu
% Date and time: Wed, Jan 13, 2016 - 01:44:29 PM
\begin{table}[!htbp] \centering 
  \caption{Counties with Income Tax Agreements for  Total Firm Births} 
  \label{--rd} 
\begin{tabular}{@{\extracolsep{5pt}}lcccc} 
\\[-1.8ex]\hline 
\hline \\[-1.8ex] 
 & \multicolumn{4}{c}{\textit{Dependent variable:}} \\ 
\cline{2-5} 
\\[-1.8ex] & \multicolumn{4}{c}{births ratio} \\ 
 & OLS & OLS & OLS & OLS \\ 
\\[-1.8ex] & (1) & (2) & (3) & (4)\\ 
\hline \\[-1.8ex] 
 Property Tax Difference & 0.272 & 0.283 & 0.105 & 0.100 \\ 
  & (0.297) & (0.314) & (0.278) & (0.286) \\ 
  Income Tax Difference & $-$0.116 & $-$0.193$^{**}$ & 0.012 & 0.017 \\ 
  & (0.081) & (0.076) & (0.126) & (0.129) \\ 
  Capital Gains Tax Difference & 0.071$^{*}$ & 0.147$^{**}$ & 0.002 & $-$0.003 \\ 
  & (0.037) & (0.068) & (0.072) & (0.074) \\ 
  Sales Tax Difference & $-$0.014 & $-$0.090 & 0.043 & 0.044 \\ 
  & (0.064) & (0.088) & (0.076) & (0.078) \\ 
  Corp Tax Difference & 0.080$^{*}$ & 0.059 & $-$0.008 & $-$0.007 \\ 
  & (0.043) & (0.036) & (0.040) & (0.041) \\ 
  Workers Comp Tax Difference & 0.392$^{***}$ & 0.050 & 0.075 & 0.071 \\ 
  & (0.126) & (0.185) & (0.166) & (0.171) \\ 
  Unemp. Tax Difference & $-$0.083 & $-$0.016 & 0.023 & 0.021 \\ 
  & (0.071) & (0.088) & (0.049) & (0.050) \\ 
  Educ Spending Per Cap Diff & 0.0004 & 0.00003 & $-$0.0001 & $-$0.0001 \\ 
  & (0.0005) & (0.001) & (0.0004) & (0.0005) \\ 
  Highway Spending Per Cap Diff & $-$0.001 & $-$0.001 & $-$0.0001 & $-$0.0001 \\ 
  & (0.001) & (0.001) & (0.0005) & (0.001) \\ 
  Welfare Spending Per Cap Diff & 0.001$^{**}$ & 0.0003 & $-$0.00003 & $-$0.00003 \\ 
  & (0.0003) & (0.0005) & (0.0003) & (0.0003) \\ 
  Constant & $-$0.086 & $-$0.221 &  &  \\ 
  & (0.226) & (0.169) &  &  \\ 
 \hline \\[-1.8ex] 
controls & Yes & No & Yes & Yes \\ 
amenities & Yes & No & Yes & No \\ 
\hline \\[-1.8ex] 
Observations & 2,963 & 2,963 & 2,963 & 2,963 \\ 
R$^{2}$ & 0.150 & 0.061 & 0.226 & 0.178 \\ 
\hline 
\hline \\[-1.8ex] 
\textit{Note:}  & \multicolumn{4}{r}{$^{*}$p$<$0.1; $^{**}$p$<$0.05; $^{***}$p$<$0.01} \\ 
\end{tabular} 
\end{table} 


\section{Georgeanne Edits}

\section{Resampling Procedure}

First, let $i = 1,...,N$ be all of our matched county pairs, and $t = 1,...,T$ be each time period. We can then write our model as,

$y_{it} = X_{it}\beta + e_{it}$
Where $y_{it}$ is the difference in logged firm start up rates, and $X_{it}$ is our differenced independent variables.

Then, our model becomes the usual POLS estimator.

$$\hat \beta = (\frac{1}{TN}\sum_{t=1}^{T}\sum_{i=1}^{N}x_{it}'x_{it})^{-1}(\frac{1}{TN}\sum_{t=1}^{T}\sum_{i=1}^{N}x_{it}'y_{it})$$

Let us assume for simplicitly we only have state level independent variables (which is true for most of our estimated models). Then, let $g = 1,...,G$ be the number of \textit{state-pairs} in our sample, and $k = 1,...,N_{g}$ be the state-pair specific number of matched county pairs. Letting $\bar G = \sum_{g=1}^{G}N_{g}/G$ be the average number of state-pair matched county pairs, note that $N = \sum_{g=1}^{G}N_{g}$. Therefore, $N = \frac{\sum_{g=1}^{G}N_{g}}{G}G = \bar G G$

we can rewrite our modeol to be, $y_{kgt} = X_{gt}\beta+e_{igt}$, and our estimator to be
$$\hat \beta = (\frac{1}{TG}\sum_{t=1}^{T}\sum_{g=1}^{G}\frac{N_{g}}{\bar G}x_{gt}'x_{gt})^{-1}(\frac{1}{TG}\sum_{t=1}^{T}\sum_{g=1}^{G}x_{gt}'\frac{(\sum_{k=1}^{N_{g}}y_{kgt})}{\bar G})$$

Now, we can plug back in our original model,
\begin{align*}\hat \beta &= (\frac{1}{TG}\sum_{t=1}^{T}\sum_{g=1}^{G}\frac{N_{g}}{\bar G}x_{gt}'x_{gt})^{-1}(\frac{1}{TG}\sum_{t=1}^{T}\sum_{g=1}^{G}x_{gt}'\frac{(\sum_{k=1}^{N_{g}}x_{gt}\beta+e_{kgt})}{\bar G}) \\
 &=  (\frac{1}{TG}\sum_{t=1}^{T}\sum_{g=1}^{G}\frac{N_{g}}{\bar G}x_{gt}'x_{gt})^{-1}(\frac{1}{TG}\sum_{t=1}^{T}\sum_{g=1}^{G}\frac{N_{g}}{\bar G}x_{gt}'x_{gt}\beta +x'_{gt}\frac{(\sum_{k=1}^{N_{g}}e_{kgt})}{\bar G}) \\
 &= \beta + (\frac{1}{TG}\sum_{t=1}^{T}\sum_{g=1}^{G}\frac{N_{g}}{\bar G}x_{gt}'x_{gt})^{-1}(\frac{1}{TG}\sum_{t=1}^{T}\sum_{g=1}^{G}x'_{gt}\frac{(\sum_{k=1}^{N_{g}}e_{kgt})}{\bar G}) \end{align*}

The result of this is that we see that our estimator appears to be a weighted estimate of the data in our sample. In particular, counties that have more than $\bar G$ worth of observations get underweighted compared to their mean, and counties that have less than $\bar G$ worth of matched county pairs get overweighted compared to their mean.

An example of this would be, imagine we only had 4 states in our sample, Texas-Oklahoma, and Maryland-Delaware. TX-OK has 35 matched county pairs per year, and MD-DL has 2. As a result, the mean is 17.5. Then, when we go to compute our estimate, the estimator over emphasizes interactions along TX-OK's border, and underweights interactions along MD-DL's border. (We can see this when I went to do the Rural v Urban estimates, and how property tax appeared. Many of the states with higher observation counts appear in more rural counties.

As a result, I (starting last spring) was also calculating a Donald and Lang (2007) two-stage estimator. The first stage is to simply take averages along each border, and then take the difference. This leads to the alternative estimator,

Now note that $E(y_{tg})$ is the average for each side of the border. By construction, this is equal to,

$$E(y_{tg}) = x_{tg}\beta+E(e_{tg})$$

Thus this model requires the same assumptions we make for our first model. We can then compute the estimator as follows.

\begin{align*}\hat \beta' &= (\frac{1}{TG}\sum_{t=1}^{T}\sum_{g=1}^{G}x_{tg}'x_{tg})^{-1}(\frac{1}{TG}\sum_{t=1}^{T}\sum_{g=1}^{G}x_{tg}'E(y_{tg})) \\ 
&= (\frac{1}{TG}\sum_{t=1}^{T}\sum_{g=1}^{G}x_{tg}'x_{tg})^{-1}(\frac{1}{TG}\sum_{t=1}^{T}\sum_{g=1}^{G}x_{tg}' x_{tg}\beta+x_{tg}'E(e_{tg})) \\
&= \beta + (\frac{1}{TG}\sum_{t=1}^{T}\sum_{g=1}^{G}x_{tg}'x_{tg})^{-1}(\frac{1}{TG}\sum_{t=1}^{T}\sum_{g=1}^{G}x_{tg}'E(e_{tg})) \end{align*}

\end{document}

Taxes are a major lever that policy makers use to bring about change in communities, where many attempt to either spur economic growth or raise revenue for new initiatives. Estimating the impacts of taxes on economic activity provides large value to lawmakers looking to understand and assess how and when to raise or lower taxes. For many states this impact is doubly important as they are required to have balanced operating budgets at the end of each fiscal year. Because of this, states cannot use deficit spending to make up for short run slow losses in tax revenue, and are forced to navigate a careful balance between promoting employment and wage growth while maintaining yearly revenue. Knowing how tax policies impact economic activity provides policy makers more knowledge on the real costs of implementing tax policy changes, particularly over short time periods.

One of the major ways in which taxes may impact economic activity is through deterring new firm start ups. Firms provide new employment, capital, and innovation into economies, while still bringing new tax revenue to state coffers. Many tax cuts are carried out under the assumption that increased growth will quickly return government tax revenue back to their original level, or, to those looking to raise taxes, that hikes will not have a large distortionary effect on economic growth. Providing estimated values for the impacts of taxes and expenditures on firm entry might better allow State and Federal government's the ability to properly account for tax incidence.

Accordingly this paper tests whether or not taxes impact firm entry rates. This topic has been constantly examined by economists over the years. One of the major unanswered questions is accounting for joint changes in tax policy when estimating these impacts. Traditionally researchers have only estimated a few taxes at once, while the levers of policy actions extend across a large array of tax rates. We add value to the literature by including the longest array of top marginal tax rates used to date. This includes property, income, corporate, capital gains, sales, workers compensation, and unemployment insurance top marginal tax rates.

These tax rates cover the vast majority of existing tax rates that state policy makers use. Many governments may opt to change tax rates jointly. An example of a policy that would cause such a joint movement may be lowering corporate taxes but keep revenue neutral by raising income taxes. This allows us to both track changes in tax policy that alters state expenditures as well as policy changes that are meant to change tax incidence. Much of the existing literature includes a much smaller array of tax rates, which has the potential to create omitted variable bias especially if governments attempt to hide the true burden of taxes by shifting the tax incidence. Our longer array helps capture the full impact of these changes on economic activity.

The paper proceeds in the following manner. First, we provide a model to show how utilizing discontinuities along state borders allow researchers to control for location specific determinants of firm entry when the full location choice of firms may be unknown. Next, we explore characteristics of state tax structure, relative firm entry, and frequency of joint tax changes. Then we explain our empirical design, which uses matched county pairs on either side of state borders to identify the effects on taxes on firm start up rates.

We provide estimates for how differences in state level tax and expenditures per capita impact relative firm entry rates into counties on either side of the border. This includes estimates for a sliding scale of estimates for matched urban and rural communities, and year specific effects. We conclude by providing an estimate for how large the aggregate impact of taxes is on relative firm entry along US states based both on ranking by existing discrepancy in mean firm entry rates, and by the predicted difference.

The results of this paper aim to provide clear, well identified, estimates of the impacts of top marginal tax rates on firm entry. This estimate may be of value to policy makers looking to judge the efficacy of tax cuts or hikes on local economic activity and state tax revenue better than existing estimates.
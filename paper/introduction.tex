
\section{Introduction}

Each year states and counties try to incentivize new business start-ups in their locale over alternative choices. The most visible cases include governments offering firm's temporary reprieve from tax burdens, or deals to build infrastructure to support new entrants. There has been a growing literature addressing the efficacy of these exemptions both to state revenue and public welfare. States may also take a long run approach to incentive new firm start-ups by altering their tax and regulatory codes for all market participants. 

This paper tests whether or not taxes impact firm entry rates between neighboring states. We also explore whether or not firms place preferences over government expenditures in their entry decision. This later situation would imply that increasing taxes to pay for certain services may not have distortionary effects on firm start up rates. The paper tests the impacts of seven top marginal tax rates, including property, income, corporate, capital gains, sales, workers compensation, as well as unemployment insurance, and state expenditures per capita on welfare, highways, and education.

We provide two methods to address these problems. We first estimate count data models of firm entry as a benchmark against existing literature in the field. We then outline our preferred estimator, which uses matched county pairs on either side of a state border. By matching counties, we get an approximate bandwidth around the sharp discontinuity in policies at state borders. The process may control for unobserved variables in count data models, as well as interaction terms between policy and location specific components of firm entry. This gives us the ability to identify just the policy effect.

This county-pair difference estimator is a generalization of the traditional difference in difference estimator extended over a larger panel of agents and time periods. It has been used to explore how different policy regimes affect local migration between geographic entities (McKinnish (2005) (2007)), as well as how differences in policies impact outcomes across borders (Holmes (1998), Rohlin (2011), Dube et al (2010), McPhail, Orazem, Singh (2010), Dhar and Ross (2012),  Kahn and Mansur (2013)). Many of these papers focus on how firms or individuals respond to imposed opportunity costs in location from neighbor's policy action.

Broader economic theory provides mixed outcomes to if a higher number of firm start ups increases welfare. Cases where more firms are welfare decreasing appear when assuming fixed costs to entry, such that too many firms entering can exceed their marginal benefit. Comparably if consumers have preferences over differentiated products, increasing the number of firms ('varieties') can be welfare increasing. The theory we develop here works outside of these considerations. Instead our results can help governments achieve higher employment growth and government revenues, as well as extending existing determinants of firm sorting.

The paper proceeds in the following manner. First, we review literature relating to empirical firm sorting and entry and how regression discontinuity techniques have been used in related studies. We then provide a model to show why state borders may allow us to control for location specific terms of firm entry. Next, we explain our empirical design, which uses both count data models  and a regression discontinuity technique using matched county pairs. Finally, we estimate reduced form models of the impact of various policies on firm start up rates. We then provide estimates for the impacts of taxes and government expenditures on firm entry into US counties. We conclude by showing which borders have the largest firm start up discrepancy, and talk about applications and limitations for our work.
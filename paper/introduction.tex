
\section{Introduction} 

This paper tests whether or not taxes impact firm entry rates. This topic has been constantly examined by economists over the years. However there remain unanswered questions, and sometimes conflicting answers in the literature. As a result such studies remain useful to help clear up previous work, and provide policy makers understanding on how new rules may impact a economies.

One of the major unanswered questions...

 Among the conflicting answers, we see that estimates are sometimes inconsistent across papers. This is due to tax variables not having a consistent measurement to estimate marginal effects.


 variables seem sensitive to what other variables are included, and how  is what the relevant variables are to use to estimate the impacts of taxes on firm entry. Different papers have utilized a plethora of different variables to estimate the impacts of taxes on firm entry rate. We argue that top marginal tax rates are the proper identifying variable. Further, most of the existing studies have focused on a small array of tax variables. This leaves space for us to introduce a longer array of top marginal tax rates, 

Further, estimation of  marginal effect has been historically difficult due to both theoretical and empirical concerns. Among theoretical concerns, there is clearly endogeneity between economic conditions and policy at a variety of different levels. At the empirical level, devising empirical strategies that accounts for the proper marginal incentives, while properly accounting for engogeneities has taken some time to develop. 

This paper builds on recent work that utilizes regression discontinuity techniques in order to control for a variety of endogeneity. It further extends the literature by including a larger array of top marginal tax variables, that includes property, income, corporate, capital gains, sales, workers compensation, and unemployment insurance top marginal tax rates.

The paper proceeds in the following manner. First, we review literature relating to estimating the determinants of firm entry. We then provide a model to show how utilizing discontinuities along state borders allow researchers to control for location specific determinants of firm entry. Next, we explain our empirical design, which uses matched county pairs on either side of state borders to identify the effects on taxes on firm start up rates. We then provide estimates of these impacts. We conclude by showing which borders have the largest firm start up discrepancy, and talk about applications of our work.

\section{Introduction} 

This paper tests whether or not taxes impact firm entry rates between neighboring states. Estimation of this marginal effect has been historically difficult. Tax and other policy parameters tend to feature prolonged periods of stability, and changes may be endogenous to many common dependent variables, such that changes in GDP, wages, and employment will entice government officials to try and improve economic performance. This has led to time series applications to use narrative approaches to try and identify the impacts of exogenous shocks to tax rates on macroeconomic variables (Romer and Romer (2007), Mertens and Raven (2013)). 

We use a border-differencing technique to establish local estimates of the impacts of taxes on firm entry rates. This method controls for endogeneity of government policy in response to local economic outcomes. For example, high economic activity states may raise their taxes knowing that local agglomeration factors will continue to attract an asymmetrically high amount of new firm startups, while low economic activity states may lower taxes to attract new businesses. This response would upwards-bias the estimate of the impacts of taxes. Using the differences in firm entry rates along state borders controls for local agglomeration factors, and treat differences in policy variables as exogenous.

This technique relies on the assumption that new firms pick entry locations within a local choice set. Recent studies on agglomeration economies seem to support this view, where Rosenthal and Strange (2003, 2005), and Arzaghi and Henderson (2008)) show that entrepreneurs weight locations within a mile of them significantly higher than distances further away. Use of border discontinuity designs started with Holmes (1998), though has quickly been adopted by researchers looking to identify effects of policies across public economics, including minimum wages (Dube et al (2008), Rohlin (2011)), welfare (McKinnish (2005, 2007)), and school quality by Dhar and Ross (2012). It has even been used by recent papers looking at the impacts of taxes on firm start up rates, including Rathelot and Sillard (2008), Duranton et al (2011), and Rohlin, Rosenthal, and Ross (2014).

Our paper builds on a longer literature looking at determinants of firm entry. Early papers such as Carlton (1979, 1983) and Schmenner (1975, 1982) failed to find incidence of taxes on firm entry rates, instead finding that higher taxes could attract more firms. Starting in the 80's methods and data  allowed for cleaner identification, such that authors started to more definitively show that taxes had an impact on business activity, including Wasylenko \& McGuire (1985), Bartick (1985), Papke (1991), and Hines (1996).

The literature studying the impacts of taxes on firm entry behavior has not settled on the best variables to use for identifying the effects of taxes on firm start up rates. Carlton (1983) used top marginal tax rates for corporate and income tax, but weighted them together, as well as property tax rates. Schmenner (1987) uses state and local property tax revenues per dollar of personal income. Helms (1985) used a budget constraint to estimate the impacts of rising tax revenue on explanatory variables. All three versions have modern equivalents and the literature has not settled on a single best practice to recover the proper marginal effects.

Theory indicates that marginal tax rates are what matter to individuals, and measures of average tax burden change due to both fluctuations in wages or profits, as well as to changes in tax rates.Using average tax rates may add endogeneity into models. Also, politicians may alter multiple taxes at once in order to accomplish policy goals, such that excluding taxes may imply omitted variable bias. Therefore, we argue that using top marginal tax rates is the preferred method of estimating marginal effects of taxes.

We use a data set that includes property, income, corporate, capital gains, sales, workers compensation, and unemployment insurance top marginal tax rates. We further add variables for government expenditures, including highways, education, and welfare. This mirrors the balanced budget approach of Helms (1985), Gabe and Bell (2004), and  Ojede and Yamarik (2012). Entrepreneurs sorting along by government expenditures would imply that increasing tax rates to pay for certain services may not have negative effects on firm start up rates.

The paper proceeds in the following manner. First, we review literature relating to estimating the determinants of firm entry. We then provide a model to show how utilizing discontinuities along state borders allow researchers to control for location specific determinants of firm entry. Next, we explain our empirical design, which uses matched county pairs on either side of state borders to identify the effects on taxes on firm start up rates. We then provide estimates of these impacts. We conclude by showing which borders have the largest firm start up discrepancy, and talk about applications of our work.
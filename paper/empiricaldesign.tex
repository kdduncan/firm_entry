\section{Empirical Design}

Our parameters of interest are the coefficients on our top marginal tax rates. Estimation of this marginal effect is historically difficult. Tax and other policy parameters tend to feature prolongued periods of stability, and changes may be endogenous to many common dependent variables, such that changes in GDP, wages, and employment will entice government officials to try and improve economic performance. This has led to time series applications to use narrative approaches to try and identify the impacts of exogenous shocks to tax rates on macroeconomic variables (Romer and Romer (2008), Mertens and Raven (2012)). 

We estimate a pseudo-regression discontinuity approach as a way of trying to remove local drivers of firm entry identified in count data models, as well as shared responses to larger macroeconomic shocks. In the limit as firms approach borders from either side leads to the policy discontinuity to be locally exogenous....

\subsection{Regression Discontinuity Approach}

By our theory we know that location specific terms, an terms shared across observations get canceled out as we take the difference while approaching the borer. Our data does not allow us to get a closer estimation to the true discontinuity than those provided by the borders of the county. The average county in our data set is 1260 square miles, or about 35 miles per side of believed to be approximately square. This distance is slightly longer than more refined approaches such as Rohlin (2011). In practice we match up counties on either side of a state border, let us denote them subject ($sub$) and neighbor ($nbr$), and their states $stA$ and $stB$. Then, taking differences, we get by applying Theorem \ref{thrm}

\begin{equation}
\ln(n_{sub,stA,t})-\ln(n_{nbr,stB,t}) = \beta_{stA}-\beta_{stB}+(X_{stA,t-1}-X_{stB,t-1})\beta_{2} + \epsilon_{sub,stA,t}-\epsilon_{nbr,stB,t}
\end{equation}

First, let us index each state-pairs $(stA,stB)$ by $g$. Next let us assume that $\beta_{stA}-\beta_{stB} = \beta_{0}$ for all $sub, nbr$ pairs. Since we assign $sub$ and $nbr$ arbitrarily, this implies that $\beta_{stA} = \beta_{stB}$.  Then we make the following definitions.
\begin{equation}
\ddot \ln(n_{i,g,t}) = \ln(n_{sub,stA,t})-\ln(n_{nbr,stB,t})
\end{equation}

\begin{equation}
\ddot X_{g,t-1} = 1+(X_{stA,t-1}-X_{stB,t-1})
\end{equation}

\begin{equation}
\ddot \epsilon_{i,g,t} = \epsilon_{sub,stA,t}-\epsilon_{nbr,stB,t}
\end{equation}

Assume $\ddot \epsilon_{i,g,t}$ be an i.i.d white noise draw, then let $\ddot X_{g} = (\ddot X_{g,0}',...,\ddot X_{g,T-1}')'$ be a $T \times (1+K_{j})$ matrix, and $\ddot \epsilon_{ig} = (\ddot \epsilon_{i,g,1},...,\ddot \epsilon_{i,g,T})'$ be a $T \times 1$ vector. Next we assume the traditional OLS moment conditions.

\begin{assumption}\label{noend}
Let  $\ddot X_{g} = (\ddot X_{g,0}', ... ,\ddot X_{g,T-1}')'$ be a $T \times (1+K_{j})$, and $\ddot \epsilon_{i,g} = (\ddot\epsilon_{i,j,1},...,\ddot\epsilon_{i,j,T})'$ a $T \times 1$ vector. Then 
\begin{equation}E[\ddot X'\ddot \epsilon] = 0, \quad \forall i,g\end{equation}
\end{assumption}

\begin{assumption}\label{fullrank}
 \begin{equation}E[\ddot X_{g}'\ddot X_{g}] = 1+K_{j}: \quad \forall g\end{equation}
\end{assumption}

We can estimate a pooled OLS estimator using Assumption's \ref{noend} and \ref{fullrank}. This gives us the POLS estimator;
\begin{equation}\label{pols}
\hat \beta_{2} = \left(\frac{1}{N^{*}} \sum_{k=1}^{T}\sum_{i=1}^{G}\sum_{j=1}^{N_{G}}\ddot X_{g,t-1}'\ddot X_{g,t-1}\right)^{-1}\left(\frac{1}{N^{*}}\sum_{k=1}^{T}\sum_{i=1}^{G}\sum_{j=1}^{N_{G}}\ddot X_{g,t-1}'\ddot \ln(n_{igt})\right)
\end{equation}
\begin{equation}
N^{*} = T(\sum_{g}^{G}N_{g})
\end{equation}

We can rewrite \ref{pols} as:
\begin{equation}\label{pols_2s}
\hat \beta_{2} = \left(\frac{1}{TG} \sum_{t=1}^{T}\sum_{g=1}^{G}\frac{\sum_{i=1}^{N_{g}}\ddot X_{g,t-1}'\ddot X_{g,t-1}}{E[N_{g}]}\right)^{-1}\left(\frac{1}{TG}\sum_{t=1}^{T}\sum_{g=1}^{G}X_{g,t-1}'\frac{\sum_{i=1}^{N_{G}}\ddot \ln(n_{igt})}{E[N_{g}]}\right)
\end{equation}

\begin{equation}
E[N_{g}] = \frac{\sum_{g=1}^{G}N_{g}}{G}
\end{equation}

When doing inference there may be shocks to the state-pair border, for example the Mississippi river flooding along a border pair, but not shared with all other pairs in the sample. Therefore we use clustered errors on the state pair. Let $\ddot X$ be the $(\sum_{g}^{G}N_{G} \times T) \times (1+K_{j}) $ regressor matrix. Thus our variance co-variance matrix takes on the form
\begin{equation}\label{var}
\hat V =\frac{1}{G-2}\frac{\sum_{g=1}^{G}N_{g}-1}{\sum_{g=1}^{G}N_{g}-2}(\ddot X'\ddot X)^{-1}(\sum_{t}^{T}\sum_{g}^{G}u_{s}u_{s}')(\ddot X'\ddot X)^{-1}
\end{equation}
\begin{equation}\label{error}
u_{s} = \sum_{i}\hat \ddot \epsilon_{i,j,t-1}\ddot X_{g,t-1}
\end{equation}
We assume this lag structure to indicate that firms respond to variables from the previous time period, and as they are starting up government's may choose to alter policies for the current year. In practice though most of our variables are heavily inter-temporally correlated, so no major difference occurs in sign, significance, or fit appears from using different lag structures.

\subsection{Sensitivity Tests}

\subsubsection{Added Controls}

We first test our regressions with a variety of controls. We first report our benchmark model, which includes just our seven top marginal tax rates, and our three sources of government expenditures. We then have two sets of controls, county level geographic amenities, and state level economic controls. We estimate models that include and exclude one of each, and then include both. We want to check whether or not our tax and regulatory variables become statistically insignificant once we account for these additions, and in our second model check whether or not they properly become indistinguishable from zero. 

\subsubsection{State-Pair Fixed Effects}
Then, we test a version of our model that includes a state-pair fixed effects. By construction of our estimator, we are claiming that any county level fixed effects take the form of location specific terms, which have to cancel out when we take the difference but state specific fixed effects may remain. This might include unaccounted government policies, such as usury laws, or other unobserved state specific characteristics.

\subsubsection{IV}

We use the difference in Log Water Area Z score as an instrument for population...

\subsubsection{Extended Bandwidth}

We then extend the bandwidth of our estimator. This is done by matching every "subject" county with every neighbor's neighbor that they weren't previously matched with. This estimate extends the distance between each of our observations, such that we expect state tax differentials to play a less important role, and local economic activity controls become more important.

\subsubsection{Relaxing Coefficient Symmetry}
We then test a version of this model where we do not impose the coefficients are the same across borders.
\begin{equation}\label{sense1}
\ddot \ln(n_{g,t}) = X_{stA,t-1}\beta_{sub}+X_{stB,t-1}\beta_{nbr}+ e_{igt} 
\end{equation}

Instead we let coefficients take on their own value in the difference, and do a set of F-tests on whether or not our assumption that $\beta_{i,A} = -\beta_{i,B}$ holds in the difference as assumed. The results of this regression are reported in Table \ref{table:equal}. Corresponding F tests are presented in Table \ref{table:Ftests}. Next we run our regression discontinuity estimator while forcing the coefficients to be the same. We present results for this model in \ref{table:canon}.

\subsubsection{Period Specific Cross Section Analysis}
Third, we test a set of regressions where we estimate period-specific coefficients and compare them to our pooled estimator to try and estimate of whether or not it is safe to assume that profit parameters are roughly stable over time. 
\begin{equation}\label{sense2}
\ddot \ln(n_{g,t})  = X_{stA,t-1}\beta_{stA}+X_{stB,t-1}\beta_{stB}+ e_{i,g,t}: \quad t = 1999,...,2008
\end{equation}
Which leads to the POLS coefficient;
\begin{equation}
\hat \beta_{2} = \left(\frac{1}{G}\sum_{i=1}^{G}\frac{\ddot X_{g,t-1}'\ddot X_{g,t-1}}{E[N_{g}]}\right)^{-1}\left(\frac{1}{G}\sum_{i=1}^{G}\ddot X_{g,t-1}'\frac{\sum_{j=1}^{N_{G}}\ddot \ln(n_{igt})}{E[N_{g}]})\right)
\end{equation}

\subsubsection{Industry Subcodes}
As a final control, we further run our main regression for a variety of NAICS subcodes. Specifically, we test for the stability of our coefficients across, agriculture, fishing, forestry, and hunting; retail trade; manufacturing; and finance and insurance.


\subsubsection{Two-Stage Estimator}
Another similar estimator to ours is common in the literature. Donald and Lang (2007) suggest an alternative version of our estimator, where they first calculate $E[n_{jt}] = \frac{\sum_{i=1}^{N_{g}}n_{ijt}}{N_{g}}$, then letting $\ddot y_{gt} = E[n_{A,t}]-E[n_{B,t}]$. Then they suggest the usual POLS estimator
\begin{equation}\label{dl}
\hat \beta_{DL} = \left(\frac{1}{TG} \sum_{t=1}^{T}\sum_{g=1}^{G}\ddot X_{g,t-1}'\ddot X_{g,t-1}\right)^{-1}\left(\frac{1}{TG}\sum_{t=1}^{T}\sum_{g=1}^{G}\ddot X_{g,t-1}'\ddot y_{gt}\right)
\end{equation}

Compared to Donald and Lang's two stage estimator our estimator underweights observations we observe only a few times compared to their true mean, and overweight observations we see many times compared to their true mean. We therefore run all of our estimates for both our original estimator, as well as the Donald and Lang two-stage estimator to see if there is any divergence in their results.


\subsubsection{Endogeneity along the Borders}
Finally, we do not test for general endogeneity where states change taxes in response to the difference in firm entry rates. This is because the aforementioned stability of all of our policy parameters, it seems unlikely that they are responding to comparatively more volatile firm start up rates. Further, there is no reason to assume counties favor one set of borders over any other, unless counties find themselves systemically at a loss compared to neighbors, a corner solution we do not check for.



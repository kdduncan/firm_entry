\subsection{Results}

We report results from our count data models in Table \ref{table:pois}. The first and second columns report the results for our Poisson regression. Consistent with the literature, we confirm they are overdispered by running a Cameron \& Trivedi (1990) test for over-dispersion. Despite this, we see that for most of our tax variables we find they are negative and significant. The exceptions for this are corporate and workers compensation taxes. Our coefficients for unemployment tax are incredibly large, which is given that due to the weighting process the calculated tax rates are relatively small, such that a 1\% increase in the effective top marginal tax rate is almost a 10x increase in the current rates. Equivalent logic holds for why the property tax rate coefficient is particularly large.

Interesting, we find that higher percent of workforce with a high school education decreases firm start ups, higher minimum wage and right to work status increase firm start ups, and most government expenditures on highways and education decrease firm start up rates, while expenditures on welfare increases firm start up rates.

Columns 3 and 4 provide results for a Negative Binomial regression, where Column 3 includes amenity variables and Column 4 omits them. By letting the error structure vary more, the sign and significance switch for income and unemployment insurance tax rates. Equivalently, now higher minimum wages decrease firm start up rates, and both corporate tax rates and right to work status become statistically insignificant. We continue to see that higher education spending and higher percent of the populace with a high school education reduces firm start up rates.

In Table \ref{table:equal} we report coefficients for our matched county pair design where we do not impose that coefficients are the same on either side of the border. Columns 1 and 2 take the raw difference in firm start ups, with the first column having amenity controls and the second excluding them. Columns 3 and 4 use the difference in log firm start ups, with the same structure on including amenity controls. Visually we see for the most part the coefficients are in fact the equal and opposite from our regressions. Same with the count data models, higher corporate tax rates imply a higher number of new firm start ups that is statistically significant. The difference in state minimum wage rates and right to work status are both indistinguishable from zero. Of note though is that many of the papers that have found positive sign on these restrict their start up data to a finer selection of predominately small firm industries, which we do not do in our sample. In contrast to the count data model estimates, all the government expenditure figures are not statistically significant below the 5\% threshold, and further as expected, all the amenity figures are not statistically significant.

Table \ref{table:Ftests} provides F tests for the assumption of the coefficients being the same across borders. We test for each variable that $\beta_{i,sub} = - \beta_{i,nbr}$. The results verify our belief that coefficients are the same and opposite in our design is a valid assumption. The exception is sales tax rates, for the subject county they are strongly and negatively significant, but for the neighbor they not significant at all. However, given that the rest of them pass, this might be a spurious result due to the number of regressors. We see an equivalent note in the workers compensation figures in our F tests, where for the neighboring county it appears to be significant, but not for the subject county.

Table \ref{table:canon} presents our main results. While in previous regressions by not imposing equality or taking differences our estimations have presented some positive and significant estimates for taxes, by imposing equality in the difference, we see that only tax variables that are negative and significant remain. This includes property taxes, income taxes, and sales taxes as the biggest driver of the tax differential. Weirdly, we see that a higher difference in the fuel price also imposes higher firm start up rates. In our differenced count regressions, only property taxes and right to work status seem to impact the number of firm start ups. And finally, as we saw in Table \ref{table:canon}, all the amenity variables become statistically insignificant in the difference as is expected by construction of our design.

Table \ref{table:stable} shows regression results for $births\_ratio$ for the years 1999, 2002, 2006, and 2008 to be relatively equally spaced. Property taxes continue to be strongly negatively significant for all 4 time periods, with it getting slightly weaker near the end. Conversely, income taxes go from largely irrelevant to strongly significant in the early 2000's. Sales tax rates remain relatively constant in sign, magnitude, and significance across our sample as well. Alternatively, 1999 seems to have a variety of variables that appear to be significant, including right to work laws, highway spending per capita, welfare spending per capita, and workers compensation, all of which are strongly positively significant. In later years these terms become statistically insignificant of the regression, which is a feature shared in Table \ref{table:canon}'s results as well. Thus, while our tax variables seem to be the most stable among our terms in sign and significance, the assumption that terms do not change over time does not seem to hold exactly.

Finally, Table \ref{table:fe} reports a model with dummy variables for each state-pair interaction. When we make this transformation, we find that none of our variables are significant. However, this is somewhat to be expected. For certain state-pairs we only have 10 observations, and for others we have 350. Further, for each of our previous regressions, the $R^{2}$ is quite small, leading to a lot of additional variation that might be being soaked up by these terms. Despite this, we see that the coefficients with the highest t-test remain property, sales, and income, all of which are negative.

As a final output of our paper, we compare two different rankings. First we calculate the weighted tax differential by multiplying the tax coefficients from Table \ref{table:canon}, column 4 times each states marginal tax values. These are reported in Table \ref{table:taxdifferential}. We then calculate the mean number of firm start up differential over the entire state border by taking a pooled average across counties for each time period in our sample. In both cases we rank them from highest to smallest, but report only the top 10, where results are reported in Table \ref{table:meanfirmstart}.

We see that for most states the weighted tax differential is very small, especially given that taxes are on a scale of 1 to 100. Thus the implied impact of taxes on relative firm start up rates is ultimately unsurprisingly small. However, for a few counties, this is not the case, and we see clear outliers where more than 1\% of the differential is motivated by the difference in tax rates. Since the coefficient on unemployment insurance tax is so large, we also tested a version of the weighted tax differential with it removed, but nothing changes, and inspection on the data itself implies that the weighting process to calculate the term makes it on average very small.

Comparably, we can look at what the empirical difference is in the log number of firms. Thus we can get a comparably idea to measure the relative impacts of these tax policies on firm start up rates, as well as show which borders currently have the largest discrepancy. Below I present the top 10 states with the largest discrepancy in their relative weighted tax impact.

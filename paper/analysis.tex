\section{Results}

Our main results are reported in Table (\ref{--rd}). The first four columns report the pooled OLS estimates with and without our sets of additional control variables. The last two columns report our fixed effect estimates. Higher relative income taxes and sales taxes deter firm entry.  This result is robust to the addition of controls, and in fact, the estimated effects become slightly stronger (more negative) with the added measures.  While statistically significant the effects are economically small.  A 1 percent increase in income tax differentials correspond to a 0.8 percent decrease in the relative firm start up rates, and similarly a 1 percent increase in sales tax differentials corresponds to a 0.1 percent decrease in the relative firm start up rates.  Higher relative property tax rates also exert a negative influence on firm births, although this effect becomes statistically insignificant when amenity measures are included in the model. Whilecapital gains, corporate tax, workers compensation, and unemployment insurance tax rates are individually insignificant, the set of seven tax rates are jointly  significant.

Of the three expenditure measures included in the model, only the difference in log welfare spending per capita is statistically significant.The coefficient is economically very small, such that a 1\% increase in the difference corresponds to 0.001\% higher firm entry rates. Finally, contrary our assumptions, not all of our county level geographic amenities and state level controls become zero at the border. The difference in log real fuel price remains positive and statistically significant, and both the difference in Temperature in January and Log Area with Water remain significant among the geographic controls. 

When we run models with state-pair level fixed effects we fail to obtain any statistically significant results. However, the value of these models are dubious. We first argue that our pooled OLS estimates are most likely the properly specified model as firm start up rates are an already differenced estimate. Thus the inclusion of state pair fixed effect require year to year divergence in expected profit from entry, which shouldn’t occur under perfect competition. Rather this still might imply that there are still relevant variables we may be leaving out of our model.

Table \ref{--eb} reports the estimates for the extended bandwidth version of our model. We expect that the increased distance between the two locations, and the increased distance from the border, will diminish the impact taxes have on firm start up rates.  Meanwhile we would expect the  measures of state and local factors to have a larger impact.  Our results are consistent with these expectations. The income and sales tax rates lose statistical significance across model types. Further, our state level controls remain largely insignificant, as do our geographic controls. Thus, the fit of the model at large seems to decrease as the distance between counties increases. 

When we relax the assumption that  that coefficients are equal on either side of the border, we find that for most of our variables, the effects remain equal but opposite across the border. Table \ref{--noequality} reports coefficients, while Table \ref{--Ftests} provides F-tests of the hypothesis that the coefficients are equal, that for each variable, $\beta_{i,sub} = - \beta_{i,nbr}$. We test for each variable that $\beta_{i,sub} = - \beta_{i,nbr}$. The results verify our belief that the coefficients are of equal magnitude and opposite sign. The exceptions are sales tax rates and workers compensation tax rates. For the subject county sales taxes are strongly and negatively significant, but for the neighbor they are insignificant. However, given that the rest of them pass, this might be a spurious result due to the number of regressors. We see an equivalent note in the workers compensation figures in our F tests, where for the neighboring county it appears to be significant, but not for the subject county.\footnote{Also, the assignment process here might be driving results. We are not running each coefficient as a fixed effect for each border, but rather across all counties defined as "neighbor" in our sample. However, by using clustered standard errors we do not have the degrees of freedom to run this test for each state-pair.} Hover, given the rest of taxes pass this test, this finding might be a spurious result due to the number of regressors.

Table \ref{--year} shows regression results for each  year between 1999 and 2009. We include state controls but exclude geographic amenities. Property taxes remain consistently negative and statistically significant over the time period. Likewsie, sales tax rates remain negative and statistically significant, with the effect becoming somewhat larger over time.  Income taxes are insignificant, but negative, at the beginning of the time period, but  become statistically significant and larger in the later years. Log highway and welfare expenditures per capita are positive drivers of firm entry, but the effects are inconsistently significant across the time periods and the magnitudes are very small.

Finally, Table \ref{naics} reports the estimates by NAICS sub codes,  Agriculture, Fishing, Forestry, and Hunting; Retail Trade; Manufacturing; and Finance and Insurance. These results are very consistent with the results for all firms in Table \ref{--rd}. Property, income and sales taxes are significantly negative in all specifications, and furthermore, the magnitudes are very similar across the industries. . This is somewhat surprising, as we would expect characteristics that drive firm entry to differ across firm types. For example, we might expect that higher property taxes would deter agricultural firm entry more than entry of financial firms.  However, it appears that the correlation between different firm types supersedes this selection. 

As a final output of our paper, we compare two different rankings to identify which borders are most (or least) disadvantaged with regard to tax differentials. First we calculate the weighted tax differential by multiplying the tax coefficients from Table \ref{--rd}, column 4 by each states marginal tax values. This measures....  These are plotted in Figure \ref{weightedtax}. For most states the weighted tax differential is very small, thus the implied impact of taxes on relative firm start up rates is ultimately small. However, for a few counties, this is not the case, and we see clear outliers where more than 1\% of the differential is motivated by the difference in tax rates. 

To illustrate how important this effect is for firm entry we rank the county-pairs by the absolute difference in the mean number of firm startups, and compare this to the weighted tax differential. Table \ref{taxdifferential} reports the top 50 largest mean differences in firm start-ups.  Since we calculate these terms in absolute value, we report which side of the border has  the advantage for firm startups in column 2 and which has the advantage in terms for tax rate differences in column 4.  Sixty-two percent of the time, the side with the more advantageous weighted tax differential also has the higher mean firm start up differential.
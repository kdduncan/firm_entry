\section{Results}

Our main results are reported in Table (\ref{--rd}). The first four columns respond to different pooled OLS estimates where we include or exclude our set of control or amenity variables. The last two columns report our fixed effect estimates. Our pooled OLS estimates show that the inclusion of the geographic amenities makes property taxes lose statistical significance. However, the results still economic intuition that most likely the impacts are small and negative across all of our model estimates. Averaging across models would imply that a 1\% increase in the relative property tax difference would decrease firm start up rates by around 0.2\%. The impacts of income and sales tax differentials remain relatively stable across our OLS estimate, such that a 1\% increase in income tax differentials correspond to a 0.8\$ decrease in the relative firm start up rates, and similarly a 1\% increase in sales tax differentials corresponds to a 0.1\% decrease in the relative firm start up rates. Even though capital gains, corporate tax, workers compensation, and unemployment insurance tax rates are individually insignificant, joint F-tests for all seven taxes show they are jointly significant.

We further see evidence that the difference in log welfare spending per capita is also statistically significant, but the coefficient is economically very small, such that a 1\% increase in the difference corresponds to 0.001\% higher firm entry rates. Finally, contrary our assumptions, not all of our county level geographic amenities and state level controls become zero at the border. The difference in log real fuel price remains positive and statistically significant, and both the difference in Temperature in January and Log Area with Water remain significant among the geographic controls.

When we run models with state-pair level fixed effects we fail to obtain any statistically significant results. However, the value of these models are dubious. We first argue that our pooled OLS estimates are most likely the properly specified model as firm start up rates are an already differenced estimate. Thus the inclusion of state pair fixed effect require year to year divergence in expected profit from entry, which shouldn't occur under perfect competition. Rather this still might imply that there are still relevant variables we may be leaving out of our model.

Table \ref{--eb} we estimate the extended bandwidth version of our model. We expect that the longer distance between two locations will make taxes have a smaller impact on firm start up rates, while traditional measures of state or local agglomeration economies will have a larger impact. Consistent with this, we see that our tax rates become less individually statistically significant across model types. Further, our state level controls remain largely insignificant, as do our geographic controls. Thus, the fit of the model at large seems to decrease as the distance between counties increases.

When we run pooled OLS estimates where we do not impose that coefficients we see that for most of our variables remain equal but opposite across the border. Table \ref{--noequality} reports coefficients, while Table \ref{--Ftests} provides F tests for the assumption of the coefficients being the same across borders. We test for each variable that $\beta_{i,sub} = - \beta_{i,nbr}$. The results verify our belief that coefficients are the same and opposite in our design is a valid assumption. The exception is sales tax rates and workers compensation tax rates, for the subject county they are strongly and negatively significant, but for the neighbor they not significant at all. However, given that the rest of them pass, this might be a spurious result due to the number of regressors. We see an equivalent note in the workers compensation figures in our F tests, where for the neighboring county it appears to be significant, but not for the subject county.\footnote{Also, the assignment process here might be driving results. We are not running each coefficient as a fixed effect for each border, but rather across all counties defined as "neighbor" in our sample. However, by using clustered standard errors we do not have the degrees of freedom to run this test for each state-pair.} 

Table \ref{--year} shows regression results for $births\_ratio$ for the every year between 1999 and 2009. We use the model that includes state controls but excludes geographic amenities. We see that property taxes remain consistently negative and statistically significant. Sales tax rates remain negative and statistically significant, and even appears to grow in its deterrence of new entry. Income taxes start off insignificant, but negative, and become statistically significant from zero. Log highway and welfare expenditures per capita vary in their significance across the sample, but remain positive drivers of firm entry when they appear.

Finally, in Table \ref{naics} we report an estimated model equivalent to Equation \ref{pref}, but where we condition firm entry on specific NAICS sub codes. For our reported estimates we include Agriculture, Fishing, Forestry, and Hunting, Retail Trade, Manufacturing, and Finance and Insurance. We find that our initial results in Table \ref{--rd}, including magnitude and strength. This is somewhat surprising, as we would expect characteristics that drive firm entry to differ across firm types. Namely, property taxes may deter agriculture more than financial firms, however it appears that the correlation between different firm types supersedes this selection.

As a final output of our paper, we compare two different rankings. First we calculate the weighted tax differential by multiplying the tax coefficients from Table \ref{--rd}, column 4 times each states marginal tax values. These are plotted in Figure \ref{weightedtax}. We see that for most states the weighted tax differential is very small, thus the implied impact of taxes on relative firm start up rates is ultimately small. However, for a few counties, this is not the case, and we see clear outliers where more than 1\% of the differential is motivated by the difference in tax rates. 

To calculate how important this effect is still aggregately we provide a table of the difference in the mean number of firm startups along each state border, as well as the weighted tax differential. Since we calculate these terms in absolute value, we similarly show which side of the border is preferred for the borders with the top 50 largest difference in mean firm start ups. This ranking is provided in Table \ref{taxdifferential}. We see that 62\% of the time the side with the preferred weighted tax differential also has the higher mean firm start up differential.

\section{Theory}

As entrepreneurs and firms look to start up a business in a new location they first choose a market to enter. This choice is due to primary considerations such as labor market characteristics, or location preferences of the owner. They then pick among possible locations in that market. Our model looks at choice of firm entry across state borders, such that individuals have mobility across the border. As a result, firms treat location specific determinants of profit as the same on both side of the border. This process leaves policy drivers as remaining difference in expected profits. We formalize the conditions for this process below.

Assume there exists a spatial equilibrium where wages and capital costs are adjusted to local tax and location specific variables affecting firm level productivity. If markets are competitive firms will make zero economic profit in the long run, but in the short run demand or policy shocks can leave short run profits. We expect that if a regime changes its taxes over time, higher production costs and lower profits exist in that county, and that market will deter a relative amount of firms from entering. Since firms will bid up or down prices relative to taxes, those prices can be proxied by the tax rates directly. Firms make decisions based on information from the previous year, as governments might concurrently change policy along with market entry and there may exist costs to establishing a business.

\begin{assumption}
Assume that a firms' profit can be expressed as a linear function, for a given location, state, and time pair denoted $(i,j,t)$,
\begin{equation}
\pi_{i,j,t} =  \gamma+\beta_{i}+\beta_{j}+X_{i,t-1}\beta_{1}+X_{j,t-1}\beta_{2}+\epsilon_{i,j,t}
\end{equation}
\begin{equation}
E[\epsilon_{ijt}] = 0
\end{equation}
$X_{i,t-1}$ is a $1 \times K_{1}$ row vector of location specific terms, and $X_{j,t=1}$ is a $1 \times K_{2}$ row vector of state specific terms, and $\beta_{i}, \beta_{j}$ are location and state specific fixed effects.
\end{assumption}

Location specific variables are any variable that is specific to a location, such as local agglomeration figures, education attainment, and other variables driven by the distribution of labor and productive factors in each regime. Variables at the regime level include taxes, regulatory policies, and government expenditures. Both sets of variables are allowed to evolve over time. Therefore this assumption simply states that our policy variables have to enter directly into the profit function, and that it is shared across all firm types.

Now let us focus on a market that is defined by the interval $[-1,1]$, such that for $i \in [-1,0)$ a firm is in state $A$, and for $i \in [0,1]$, they are in state $B$. Therefore, if a firm has two choices, $y \in [-1,0)$ and $\hat y \in [0,1]$, then the firm chooses $y$ over $\hat y$ if
\begin{equation}\label{diff}
E[\pi_{y,A,t}-\pi_{\hat y,B,t}] > 0
\end{equation}

\begin{assumption}\label{cont}
$\beta_{i}$ and $X_{i,t-1}$ are continuous locally on $[-1,1]$, such that for any $\epsilon > 0$, where  $|\beta_{i}-\beta_{j}| < \frac{\epsilon}{K+1}$, and $|(X_{y,t-1,k}-X_{\hat y,t-1,k})| < \frac{\epsilon}{(K+1)|\beta_{k}|} \forall k \in \{1,...,K_{1}\}$, then there exists a $\delta$ such that $|y - \hat y| < \delta$
\end{assumption}

This statates that as the locations firms choose between get asymptotically close to the border, the difference between unobserved location specific fixed effects and observed location specific variables converge to zero. This is a technical illustration of labor and capital mobility in close geographic areas. As the distance between the two locations increases this may no longer be the case, as illustrated in Holmes (1998).

Therefore, conditional on firms choosing locations $(y, \hat y)$ arbitrarily close to the border, the profit function becomes,

\begin{equation}\label{prof}
E[\pi_{y,A,t}-\pi_{\hat y, B, t}] =  \beta_{A}-\beta_{B}+(X_{A,t-1}-X_{B,t-1})\beta_{2}
\end{equation}

As we move away from the border location characteristics might dominate the policy effect, especially when we expect policy effects to be small. This theory favors the use of regression discontinuity techniques for estimating policy treatment effects, especially when location specific drivers of firm entry might be unknown or unobserved.
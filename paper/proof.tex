\documentclass{article}

\usepackage{amsmath}
\usepackage{amssymb}
\usepackage{amsthm}

\begin{document}

\section{Proof of Equation (4)}

et us define a country by a closed, connected set $\Theta \in \mathbb{R}^{2}$, which is broken up into a countable collection of open sets $\theta_{i},\ i = 1,...,n$, such that for any $i \neq j,\ \theta_{i} \cap \theta_{j} = \emptyset$, and letting the closure of $\theta_{i}$ be denoted, $\bar \theta_{i}$, $\cup_{i=1}^{k}\bar \theta_{i} = C$.\footnote{The point of this is to think of $\Theta$ as a country, and $\theta$ as a collection of states. This process will be iterated into local counties as well. I state that they are all open under the perception that the border is not in any particular object, but exists in the limit. Finally, $\mathcal{F}$ defines the data generating process on the set of points around each border.}

Moreover, two sets $\theta_{i}, \theta_{j}$ are denoted \textit{adjacent} if $\bar \theta_{i} \cap \bar \theta_{j} \neq 0$, for $i \neq j$, and that the cardinality, $\# (\bar \theta_{i} \cap \bar \theta_{j}) = \infty$.\footnote{i.e. state pairs have to be connected at more than a finite number of points} Then for two adjacent counties, $i,j$, we denote their border by $B_{ij} = \bar A_{i} \cap A_{j}$. Further, they are \textit{cornered} if  $\bar A_{i} \cap \bar A_{j} \neq 0,\ i \neq j$, and $\#(\bar \theta_{i} \cap \bar \theta_{j}) < \infty$.

We now make the following definitions.\footnote{The $B$ take all the border points into a set and a state-pair respectively. The $H$, take all the points in each state that fit a fixed bandwidth for each point along the borders.}

\begin{equation} B = \{ b \in \bar \theta_{i} \cap \theta_{j}, \forall i, j,\ i \neq j\} \end{equation}

\begin{equation} B_{ij} = \{ b \in \bar \theta_{i} \cap \bar \theta_{j},\ i \neq j\} \end{equation}

\begin{equation} H(\epsilon) = \{\cup_{b \in B}\left((\mathcal{B}_{\epsilon}(b)\cap\theta_{i})\cup(\mathcal{B}_{\epsilon}(b)\cap\theta_{j})\right),\ \forall i,j,\ i \neq j\} \end{equation}

Where, $\mathcal{B}_{\epsilon}(b)$ denotes the epsilon ball around a point $b$.

Now, let us have a set of $K$ covariates, where each one is in $\mathbb{R}$ that we are interested in over $T$ time periods, such that we have the random matrix $\{\Omega_{\alpha},\epsilon_{\alpha}: \alpha \in \Theta\}$ be a set of random matrices such that $\Omega_{\alpha}$ is a $T \times K$ matrix, and $\epsilon$ is a $T \times 1$ vector, and we believe that firm entry rate for each location is determined by,

$$y_{\alpha} = \Omega_{\alpha}\Gamma_{\alpha}+\epsilon_{\alpha}$$
  
  Where, let us split $\Omega_{\alpha}$ into a partitioned matrix, $\Omega_{\alpha} = [Z_{\alpha}\ X_{\alpha}]$, and $\Gamma_{\alpha} = [\gamma_{\alpha} \beta_{\alpha}]$. Then,

$$y_{\alpha} = Z_{\alpha}\gamma_{\alpha}+X_{\alpha}\beta_{\alpha}+e_{\alpha}$$
  
  Then, for any $\alpha \in \theta_{i},\ \alpha \in \theta_{j}$ such that $\bar \theta_{i} \cap \bar \theta_{j} \not = \emptyset$,
  
 

$$y_{\alpha}-y_{\alpha'} = Z_{\alpha}\gamma_{\alpha}-Z_{\alpha'}\gamma_{\alpha'}+X_{\alpha}\beta_{\alpha}-X_{\alpha'}\beta_{\alpha'}+e_{\alpha}-e_{\alpha'}$$
  
  Now, we need that, $E[Z_{\alpha}\gamma_{\alpha}-Z_{\alpha'}\gamma_{\alpha'}] \to 0$ as $\alpha,\alpha' \to b = \{\lambda \alpha + (1-\lambda)\alpha': \forall \lambda \in (0,1)\}\cap \{\theta_{i} \cap \theta_{j}\}$. Then,

$$Z_{\alpha}\gamma_{\alpha}-Z_{\alpha'}\gamma_{\alpha'} = (Z_{\alpha} - Z_{\alpha'})(\gamma_{\alpha}-\gamma_{\alpha'}) + (Z_{\alpha} - Z_{\alpha'})\gamma_{\alpha'} + (\gamma_{\alpha}-\gamma_{\alpha'})\beta_{\alpha'}$$
                                                                                                                                                                         
                                                                                                                                                                         But, as $\alpha, \alpha' \to b,\ ||\alpha,\alpha'||_{2} < \delta \implies (Z_{\alpha} - Z_{\alpha'}) < \epsilon$, and $((\gamma_{\alpha}-\gamma_{\alpha'}) < \epsilon $. Thus,
                                                                                                                                                                       
                                                                                                                                                                       $$Z_{\alpha}\gamma_{\alpha}-Z_{\alpha'}\gamma_{\alpha'} = \epsilon^{2} + 2\epsilon\gamma_{b}$$
                                                                                                                                                                         
                                                                                                                                                                         Since $\epsilon$ is arbitrary, $Z_{\alpha}\gamma_{\alpha}-Z_{\alpha'}\gamma_{\alpha'} \to 0$, and
                                                                                                                                                                       
                                                                                                                                                                       $$\lim_{\alpha, \alpha' \to b} y_{\alpha}-y_{\alpha'} = X_{\alpha}-X_{\alpha'}\beta_{b}+e_{\alpha}-e_{\alpha'} $$
                                                                                                                                                                       
                                                                                                                                                                        \end{document}
                                                                                                                                                                    
                                                                                                                                                                         
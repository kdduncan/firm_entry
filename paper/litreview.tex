
\section{Literature Review}

Location choice of firms and individuals has a rich history in economics. At its core, the question is what drives sorting behavior of firms into particular communities. One of the earliest models to approach sorting behavior of individuals was Tiebout (1956), who argued that individuals sorted into locations through preferences for prices and public amenities. Tiebout posited that counties as a result of this sorting behavior receive pressure on their provision of services in order to attract individuals.\footnote{Sorting literature similarly gave birth to tax competition among states as over viewed by Wilson (1999). Our paper can be seen as an extension of this literature, where states compete to have preferential tax differentials compared to neighboring states}

As a result Tiebout's model, the early firm entry literature focused on sorting over all available possible markets. These papers used conditional Logit models to explain the probability of agents entering into particular markets.  McFadden (1974) showed a general framework on how to use conditional Logit function to estimate firm entry sorting. Dowding, John, and Biggs (1994) review 200 articles and books that there is evidence that taxes and services affect location decisions of firms and households, and less assuredly, property values as is implied in the traditional Tiebout models.

Guimaraes, Figueirido, and Woodward (2003) showed that the conditional Logit models of firm entry can be estimated by a Poisson distribution under relatively mild assumptions. Gabe and Bell (2004) used Poisson and Negative Binomial regressions show how taxes and government spending on education impact firm location in Maine. They recover the coefficient on taxes by imposing a balanced budget requirement equivalent to Helms (1985). Their results show that increasing tax rates to raise education spending per pupil causes no distortion on firm entry rates. A review was on the differences between discrete choice models and Poisson models was done by Arauzo-Carod et al (2010). They show that as time has gone on, the discrete choice model has lost favor in favor of count data models. In their review they show that agglomeration and market size tend to have a significant positive effect, while wages and taxes act in the opposite direction.

Modern papers on firm entry paper have increasingly used border discontinuity methods. In these models researchers test whether or not local discontinuities in policy induce firms or individuals to change their start up or housing location. The first paper to use a matched county-pair estimator is Holmes (1998). In Holmes' paper, he uses right to work status as a proxy for an unobserved cost of being on either side of a state border imposed by "pro" and "anti" business policies. He then tested whether or not right to work status affected manufacturing employment growth. His estimates found that counties that have right to work status attract more manufacturing firms than states without right to work status.

Since Holmes' work, authors have utilized this technique to test a variety of expanding topics in public economics. A recent paper by Rohlin, Ross, and Rosenthal (2014) mirrors our paper very closely. They estimate a linear probability model of firm entry using a border difference estimator. They use GIS coded data to get a closer bandwidth to the border than our method, and show that increasing the personal income tax differential actually increases the likelihood of firms entering on one side of the border. However, they show that increasing the corporate and sales tax differential can drastically reduce the relative firm entry probability.

Earlier work by Rohlin (2011) looked at the impact of minimum wages on firm start up rates using less aggregated data. By utilizing the Dun and Bradstreet Marketplace data files Rohlin constructed bands around state borders, and then derived estimates on the impact of minimum wage changes on firm start up rates. He showed that increasing the minimum wage decreased new establishment activity in industries that relied heavily on minimum wage workers, but that changes in the minimum wage did not decrease employment in existing establishments.

Chirinko and Wilson (2008) use a border discontinuity technique to estimate the impacts of state investment tax credits on firm start up rates. Rathelot and Sillard (2008) use the border discontinuity technique in a Probit model to show that increasing the total tax rate differential increases the probability of a firm picking a side between 1-5\%. Duranton et al (2011) took the difference in the firm entry rate in neighboring areas to estimate the impacts of taxes on employment. They provide estimates both for traditional OLS estimates without the difference, which estimated a positive relationship between taxes and firm entry rates, but after applying the spatial difference, taxes negatively impacted firm start up rates.

From the literature, we see that on average taxes negatively impact firm start up rates, especially as researchers have gone from studying sorting over all available entry choices, to local choices along policy discontinuities. However, how taxes are calculated and used in studies differs wildly among authors. Various studies have used measures of average tax revenue, added together top marginal tax rates, or included a single available tax rate. As a result, we employ more recent spatial difference techniques to get a clear estimate, while employing a larger array of top marginal tax rates than other authors.
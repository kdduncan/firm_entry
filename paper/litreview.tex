
\section{Literature Review}

Location choice of firms and individuals has a rich history in economics. At its core, the question is what drives households and firms to choose to locate in particular communities. Tiebout (1956) argued that individuals sorted into locations based on their preferences for prices and public amenities. He posited that, because households can “vote with their feet,” counties have incentives to adjust their provision of services in order to attract residents.\footnote{Sorting literature similarly gave birth to tax competition among states as over viewed by Wilson (1999). Our paper can be seen as an extension of this literature, where states compete to have preferential tax differentials compared to neighboring states}

Guided by Tiebout’s model, the early firm entry literature focused on sorting over all available possible markets.  McFadden (1974) provided a general framework for using the conditional Logit function to estimate firm entry choices over all available possible markets. Early papers such as Carlton (1979, 1983) and Schmenner (1975, 1982) failed to find incidence of taxes on firm entry rates, instead finding that higher taxes could attract more firms. Starting in the 80's methods and data allowed for cleaner identification, such that authors started to more definitively show that taxes had an impact on business activity, including Wasylenko \& McGuire (1985), Bartick (1985), Papke (1991), and Hines (1996).

Researchers have continued to estimate models of firms sorting over a large number of counties. Gabe and Bell (2004) used Poisson and Negative Binomial regressions show how taxes and government spending on education impact firm location in Maine. Their results show that increasing tax rates to raise education spending per pupil causes no distortion on firm entry rates. A review of these sort of sorting estimates was done by Arauzo-Carod et al (2010). In their review they show that agglomeration and market size tend to have a significant positive effect, while wages and taxes act in the opposite direction. Further, the findings on the effect of property values as is implied in the traditional Tiebout models is even weaker (see Dowding, John, and Biggs (1994) for a comprehensive review of Tiebout model estimates).

Increasingly researchers have utilized border-difference technique to establish local estimates of the impacts of taxes on firm entry rates. This method controls for endogeneity of government policy in response to local economic outcomes. For example, high economic activity states may raise their taxes knowing that local agglomeration factors will continue to attract an asymmetrically high amount of new firm start ups, while low economic activity states may lower taxes to attract new businesses.\footnote{Further, tax and other policy parameters tend to feature prolonged periods of stability, and changes may be endogenous to many common dependent variables, such that changes in GDP, wages, and employment will entice government officials to try and improve economic performance. This has led to time series applications to use narrative approaches to try and identify the impacts of exogenous shocks to tax rates on macroeconomic variables. This is why narrative approaches are currently common in the macoreconometrics literature as a way of estimating the impacts of taxes see Romer and Romer (2007), and Mertens and Ravn (2013).} This response would upwards-bias the estimate of the impacts of taxes. Using the differences in firm entry rates along state borders controls for local agglomeration factors, and treat differences in policy variables as exogenous.

This technique relies on the assumption that new firms pick entry locations within a local choice set. Recent studies on agglomeration economies seem to support this view.  Rosenthal and Strange (2003, 2005), and Arzaghi and Henderson (2008)) show that entrepreneurs weight potential locations within a mile of their current location significantly higher than distances further away. Use of border discontinuity designs started with Holmes’ (1998) analysis of right to work laws on manufacturing employment growth. In Holmes' paper, he uses right to work status as a proxy for an unobserved cost of being on either side of a state border imposed by "pro" and "anti" business policies. He then tested whether or not right to work status affected manufacturing employment growth. His estimates found that counties that have right to work status attract more manufacturing firms than states without right to work status.

Since Holms's study, this technique has been adopted by researchers looking to identify effects of additional state policies, including minimum wages (Dube et al, 2008; Rohlin, 2011), welfare (McKinnish, 2005; 2007), and school quality (Dhar and Ross, 2012). Recent papers looking at the impacts of taxes on firm start up rates, including Rathelot and Sillard (2008), Duranton et al (2011), and Rohlin, Rosenthal, and Ross (2014). 

Rohlin (2011) looked at the impact of minimum wages on firm start up rates using aggregated data. By utilizing the Dun and Bradstreet Marketplace data files Rohlin constructed bands around state borders, and then derived estimates on the impact of minimum wage changes on firm start up rates. He showed that increasing the minimum wage decreased new establishment activity in industries that relied heavily on minimum wage workers, but that changes in the minimum wage did not decrease employment in existing establishments. 

Chirinko and Wilson (2008) use a border discontinuity technique to estimate the impact of state investment tax credits on firm start up rates. Rathelot and Sillard (2008) use the border discontinuity technique in a Probit model to show that increasing the total tax rate differential increases the probability of a firm picking a side between 1-5\%. Duranton, et al (2011)  difference firm entry rates in neighboring areas to estimate the impact of taxes on employment. While their  traditional OLS estimates (without the spatial difference)  show a positive relationship between taxes and firm entry rates, after applying the spatial difference, taxes negatively impact firm start up rates. 

A recent paper by Rohlin, Ross, and Rosenthal (2014) mirrors our paper very closely. They estimate a linear probability model of firm entry using a border difference estimator. They use GIS coded data to get a closer bandwidth to the border than our method, and show that increasing the personal income tax differential actually increases the likelihood of firms entering on one side of the border. However, they show that increasing the corporate and sales tax differential can drastically reduce the relative firm entry probability.

Rohlin et al utilize a measure of state-level government expenditures per capita, and utilize Tax Foundation data on top marginal sales, corporate, and personal income tax rates from 2000 to 2003. They estimate a linear probability model of the chance that a firm enters onto one side of the border. They then use reciprocal agreements on where individuals pay income taxes based on location of work rather than location of residence to try to control for proper allocation of tax burdens on each side of the state, and to provide additional strength in identification. Finally, they then use zip code level data to estimate average entry along each side of the border. Both with and without the reciprocal agreements in place, they show that there is a negative impact of increasing the tax differential between states on the probability of firm entry.

Our paper differs by having a considerably larger number of tax policy variables, thus better controlling for other tax policies that may be impact business activity. Moreover, we also include a longer time series than Rohlin et al, providing additional variation in state level tax policies over our window. We differ in only having county level data, rather than the finer zip code level data that Rohlin et al use. This provides a much finer bandwidth to identify the impacts of changes on.

A major issue with the existing literature is the failure to settle on the best variables to use for identifying the effects of taxes on firm start up rates. Carlton (1983) used top marginal tax rates for corporate and income tax, but weighted them together, as well as property tax rates. Schmenner (1987) uses state and local property tax revenues per dollar of personal income. Helms (1985) used a budget constraint to estimate the impacts of rising tax revenue on explanatory variables. All three versions have modern equivalents and the literature has not settled on a single best practice to recover the proper marginal effects.

Theory indicates that marginal tax rates are what matter to individuals, and measures of average tax burden change due to both fluctuations in wages or profits, as well as to changes in tax rates. Using average tax rates may add endogeneity into models. Also, politicians may alter multiple taxes at once in order to accomplish policy goals, such that excluding taxes may imply omitted variable bias. Therefore, we argue that using top marginal tax rates is the preferred method of estimating marginal effects of taxes. 

From the literature, we see that on average taxes negatively impact firm start up rates, especially as researchers have gone from studying sorting over all available entry choices, to local choices along policy discontinuities. However, how taxes are calculated and used in studies differs wildly among authors. Various studies have used measures of average tax revenue, added together top marginal tax rates, or included a single available tax rate. As a result, we employ more recent spatial difference techniques to get a clear estimate, while employing a larger array of top marginal tax rates than other authors.
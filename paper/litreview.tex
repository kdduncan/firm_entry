
\section{Literature Review}

Location choice of firms and individuals has a rich history in economics. Describing how and why agents sort among locations is important both for policy evaluation and understanding economic geography. One of the earliest models to approach sorting behavior of individuals was Tiebout (1956), who argued that individuals sorted into locations through preferences for prices and public amenities. Tiebout posited that counties as a result of this sorting behavior receive pressure on their provision of services through other counties threatening a better mix.

The canonical model of the Tiebout world is Zodrow and Mieszkowski (1986), where they showed how governments that faced a constrained head taxes led to under provision of public goods compared to the unconstrained social planners problem. Thus competition for consumers might lead to governments having lower head taxes, and because consumers do not see the full equilibrium effect end up getting a lower provision of public and private services in those counties. The model exhibited exogenously determined number of governments, and shared production technology that doesn't change over time, thus the model lacked serious dynamics. More modern theoretical attempts include Epple and Romer (1991).

Estimating empirical Tiebout models has been common throughout economics' history. Dowding, John, and Biggs (1994) show in a review of almost 200 articles and books that there is evidence that taxes and services affect location decisions of firms and households, and less assuredly, property values as is implied in the traditional Tiebout models. More recently Banzhaf and Walsh (2008) develop a model where agents move over a variety of governments that choose optimal price and spending levels subject to a single crossing constraint. This tests whether or not individuals move between counties in general as a result of different government expenditure and land values. 

Many of these papers use conditional Logit models to explain the probability of agents entering into particular markets.  McFadden (1974) showed how we could estimate location choice using a conditional Logit function, a process now common to industrial organization and discrete choice models in general (Nevo (2000)).  A more modern review of sorting models includes Kuminoff, Smith, and Timmins (2013), which mixes structural considerations into a discrete choice model structure, and covers both Tiebout and firm entry models.

However, the conditional Logit is not the only way to model such behavior, as Guimaraes, Figueirido, and Woodward (2003) have shown that the conditional Logit model becomes a Poisson distribution if we assume firms enter by a Poisson distribution under relatively mild assumptions. As a result, we motivate our empirical strategy by claiming that firms enter markets according to Tiebout-style considerations, and include a full set of taxes, regulatory policies, and government spending as our set of covariates.

In most sorting models researchers describe and try to estimate models that characterize the individuals choice over all possible locations. A broader discussion of empirical studies in industrial location is covered in Arauzo-Carod et al (2010). Their piece covers the use of discrete choice and count data models. The authors note that in papers that use Poisson regressions location data tends to reject the assumption that the conditional variance and means are the same. Therefore Poisson results are often presented to be more illustrative in comparison, and researchers compare results to Negative Binomial regressions. The authors also point out that count data models are more commonly applied to panel data applications than discrete choice models. 

Gabe and Bell (2004) using Poisson and Negative Binomial regressions show how taxes and government spending on education impact firm location in Maine. They recover the coefficient on taxes by imposing a balanced budget requirement and including the full non-tax revenue and expenses in Maine to recover the parameter of interest. Their results show that increasing tax rates to raise education spending per pupil causes no distortion on firm entry rates. Similarly, Brülhart et al (2012) use a series of Poisson, OLS, and 2SLS estimation strategies to test whether economies of agglomeration weaken the impacts of tax changes. They show that taxes have a statistically significant impact on firm start up rates, but that increasing agglomeration tax interaction terms weaken this effect.

Some modern approaches to sorting models have utilized border discontinuity methods. In these models individuals researchers test whether or not local discontinuities in policy induce firms or individuals to change their start up or housing location. The first paper to use a matched county-pair estimator is Holmes (1998). In Holmes' paper, he uses right to work status as a proxy for an unobserved cost of being on either side of a state border. He then tested whether or not right to work status affected both the percent of non-agricultural employment that was manufacturing, and manufacturing employment growth from 1947 to 1992. His estimates found that counties that have right to work status attract more manufacturing firms than states without right to work status.His justification was claiming the border was where the highest opportunity cost exists in firm location, and at the limit where we expect the see the largest pure treatment effect.

Since then a variety of authors have utilized this technique to test a variety of expanding topics. Most related to our paper is Rohlin (2011). Rohlin looked at the impact of minimum wages on firm start up rates using less aggregated data. By utilizing the Dun and Bradstreet Marketplace data files he was able to construct bands around state borders, and then derived estimates on the impact of minimum wage changes on firm start up rates. Rohlin showed that new and existing firms are impacted by an increase in the minimum wage differently. He showed that increasing the minimum wage did decrease new establishment activity in industries that rely heavily on minimum wage workers, but that such changes are not enough to decrease employment in existing establishments. In his study 96\% of firms are already established, thus supporting conclusions that minimum wages have little effect on total business. Equivalently, Dube et al (2010) looked at how minimum wage changes impacted employment, and found that along their contiguous border sample that increasing the minimum wage increased earnings in the restaurant sector, and no unemployment effect. 

Using this matched county-pair technique to estimate the impacts of policies on an economy, McPhail, Orazem, and Singh (2010) use top marginal tax variables to test for the impact of taxes on state labor productivity. They find that increasing top marginal rates on property, capital, and sales lowers equilibrium output per worker. Finally, McKinnish (2005, 2007) look at how changes in welfare programs induce people to change to different sides of a state border in order to apply for those benefits. McKinnish wants to see if individuals respond to more favorable welfare benefits on opposite side of a state border, and finds that counties with large benefit differentials have higher welfare expenditures per capita relative to the interior counties of their state, and border counties on the low-benefit side of state borders have lower welfare expenditures per capita relative to their interior. Thus there appears to be migration from the low benefit side to the high benefit side as a response to the policy discontinuity in welfare benefits. 

Dhar and Ross (2012) use repeat cross-sections of housing transactions near school district boundaries in Connecticut. They test for the impacts of test scores on housing values, and find significant impacts that are notably smaller than OLS and OLS estimates that include a boundary fixed effect.  Finally Kahn and Mansur (2013) use local energy prices and regulation to test the geographic concentration of employment. They show that energy-intensive industries concentrate in low electricity price counties, and labor-intensive industries avoid pro-union counties.

From the above we make the following observations; The Tiebout literature implies that models have to include both the costs and benefits of moving to a particular location to be well specified. Thus our set of regressors is driven by this consideration to include tax, regulatory, and expenditure values. Further, while historically location entry models have been estimated by discrete choice and count data models, as researchers have focused on specific treatment effect identification a natural experimental design lies in testing differences on the border between different policy regimes.

% Table created by stargazer v.5.2 by Marek Hlavac, Harvard University. E-mail: hlavac at fas.harvard.edu
% Date and time: Mon, Feb 15, 2016 - 10:40:12 AM
\begin{table}[!htbp] \centering 
  \caption{F-Tests for Density Joint Tax and Expenditure Effects for Accommodation and foodservices Firm Start Ups} 
  \label{72Ftests} 
\begin{tabular}{@{\extracolsep{5pt}} ccc} 
\\[-1.8ex]\hline 
\hline \\[-1.8ex] 
Test & F-Stat & P(\textgreater F) \\ 
\hline \\[-1.8ex] 
In MSA Taxes & 0.3678 & 0.5443 \\ 
In MSA Exp & 0.716 & 0.3975 \\ 
Same MSA Taxes & 0.0029 & 0.9571 \\ 
Same MSA Exp & 1.3659 & 0.2427 \\ 
Jointly Urban Taxes & 0.955 & 0.3285 \\ 
Jointly Urban Exp & 0.0123 & 0.9116 \\ 
Jointly Rural Taxes & 5.9195 & 0.015 \\ 
Jointly Rural Exp & 3.4479 & 0.4221 \\ 
\hline \\[-1.8ex] 
\end{tabular} 
\end{table} 
